\documentclass{article}
\usepackage{amsmath,amsthm,amssymb,mathtools}

\title{A Note on Linear Independence and Bases}
\author{William DeMeo}
\date{\today}

\newcommand{\R}{\ensuremath{\mathbb{R}}}
\newcommand{\vN}{\ensuremath{\operatorname{N}}}
\newcommand{\Span}{\ensuremath{\operatorname{Span}}}
\newcommand{\sV}{\ensuremath{\mathcal{V}}}
\newcommand{\sW}{\ensuremath{\mathcal{W}}}
\newcommand{\sA}{\ensuremath{\mathcal{A}}}
\newcommand{\sB}{\ensuremath{\mathcal{B}}}
\newcommand{\vx}{\ensuremath{\mathbf{x}}}
\newcommand{\vy}{\ensuremath{\mathbf{y}}}
\newcommand{\vw}{\ensuremath{\mathbf{w}}}
\newcommand{\vv}{\ensuremath{\mathbf{v}}}
\newcommand{\vb}{\ensuremath{\mathbf{b}}}
\newcommand{\va}{\ensuremath{\mathbf{a}}}
\newcommand{\vzero}{\ensuremath{\mathbf{0}}}


\begin{document}

\maketitle

\noindent {\bf Lemma 1:}
If $A \in \R^{m\times n}$ and $\vN(A)= \{\vzero\}$, then
$m\geq n$.
\begin{proof}
If $\vN(A)= \{\vzero\}$, then by definition the only solution to
$A \vx = \vzero$ is the trivial solution $\vzero$.  Equivalently, the system 
$A\vx = \vzero$ has no free variables, so any echelon form of $A$ has a 
pivot in each of its $n$ columns. 
Since there are $m$ rows, there can be at most $m$ pivots, so
$m\geq n$.
\end{proof}

\noindent {\bf Lemma 2:}
If $V$ is a subspace of $\R^n$ with basis 
$\sV = \{\mathbf{v}_1,\dots, \mathbf{v}_k\}$, then
any linearly independent subset of $V$ has at most $k$ vectors.
\begin{proof}
%% Recall that, by definition, $\dim V$ is the number of vectors in any basis.  So,
%% in the present case, $\dim V= k$.
  Let $\{\vx_1, \vx_2, \dots, \vx_m\}$ be a linearly independent subset of $V$.
  We will prove $m\leq k$.
  Each $\vx_i$ is a linear combination of the basis vectors in $\sV$.
  That is, there exist scalars $a_{ij}\in \R$ such that
  \begin{align}
    \label{eq:1}
    \vx_1 &= a_{11}\vv_1+a_{21}\vv_2+\cdots +a_{k1}\vv_k\nonumber\\
    \vx_2 &= a_{12}\vv_1+a_{22}\vv_2+\cdots +a_{k2}\vv_k\\
    &\vdots \nonumber\\
    \vx_m &= a_{1m}\vv_1+a_{2m}\vv_2+\cdots +a_{km}\vv_k\nonumber
  \end{align}
  Note that, in the first equation above, the right-hand side is a matrix-vector
  product, so we can write this equation as follows:
  \[
  \vx_1=
  \begin{bmatrix*}
    | & | & & |\\
    \vv_1& \vv_2 &\hdots & \vv_k\\
    | & | & & |
  \end{bmatrix*}
  \begin{bmatrix*}
    a_{11}\\
    a_{21}\\
    \vdots\\
    a_{k1}
  \end{bmatrix*}.
  \]
  Of course, the same is true of the other equations in (\ref{eq:1}), so we can
  write them all simultaneously as a the following system of equations:
  \begin{equation}
    \label{eq:2}
  \begin{bmatrix*}
    | & | & & |\\
    \vx_1& \vx_2 &\hdots & \vx_m\\
    | & | & & |
  \end{bmatrix*}=
  \begin{bmatrix*}
    | & | & & |\\
    \vv_1& \vv_2 &\hdots & \vv_k\\
    | & | & & |
  \end{bmatrix*}
  \begin{bmatrix*}
    a_{11}  & a_{12} & \cdots &a_{1m}\\
    a_{21}  & a_{22} & \cdots &a_{2m}\\
    \vdots &       & \ddots & \vdots\\
    a_{k1}  & a_{k2} & \cdots & a_{km}
  \end{bmatrix*}
  \end{equation}
  Let $X$ denote the matrix on the left-hand side of~(\ref{eq:2}), and 
  let $A$ denote the second matrix on the right-hand side of~(\ref{eq:2}).
  Since $\{\vx_1, \dots, \vx_m\}$ is linearly independent, we have
  $\vN(X) =\{\vzero\}$, and this implies $\vN(A) =\{\vzero\}$.
  (Reason: if $A \vy = \vzero$, then $X\vy = \vzero$ by~(\ref{eq:2}).)
  Finally, if $\vN(A) =\{\vzero\}$, we know that $A$ must have at least as
  many rows as columns, by Lemma 1.  Therefore $m\leq k$, as desired.
\end{proof}

\noindent {\bf Proposition:}
Let $W \subseteq V$ be subspaces of $\R^n$.
Suppose $\sV = \{\mathbf{v}_1,\dots, \mathbf{v}_k\}$ is a basis for
$V$ and $\sW = \{\mathbf{w}_1,\dots, \mathbf{w}_\ell\}$ is a basis for $W$.
Then $W = V$ if and only if $\ell = k$.

\begin{proof}\
  \begin{itemize}
  \item[($\Rightarrow$)]
  If $W = V$, then $\sW$ is a basis for $V$ so $\ell = k$, since (as we proved
  earlier) all bases for a given subspace have the same number of vectors.

  \item[($\Leftarrow$)] We want to prove $\ell=k$ implies $W = V$.
  We will prove the (equivalent) contrapositive statement, which is
  $W \neq V$ implies $\ell\neq k$.

  %% Since $\sW$ is a linearly independent subset of $V$, we know by
  %% Lemma 2 that $\ell \leq k$.  We want to show that $\ell < k$.

  If $W \neq V$, then there is a vector $\vv \in V$ that does not belong to $W$.
  Therefore, $\vv \notin \Span \{\vw_1, \vw_2, \dots, \vw_\ell\}$, so
  $\{\vw_1, \vw_2, \dots, \vw_\ell, \vv\}$ is a set of $\ell + 1$ linearly independent
  vectors in $V$.  By Lemma 2 then, $\ell + 1 \leq k$, so $\ell < k$, as desired.
  
  \end{itemize}
\end{proof}

\noindent {\bf Corollary:} If $W \subseteq V$ are subspaces of $\R^n$ and $\dim W = \dim V$, then $W = V$.

\begin{proof}
  This follows directly from the Proposition above and the definition of dimension.
\end{proof}

\end{document}


\subset \mathbf{E}(\mu)$$ is linearly independent. Prove that $$\{\mathbf{v}_1,\dots, \mathbf{v}_k, \mathbf{w}_1,\dots, \mathbf{w}_\ell\}$$ is linearly independent.

Farouq asked a good question about this.  He suggested that it might suffice to show that for each $$i$$ the set $$\{\mathbf{v}_i, \mathbf{w}_1, \dots, \mathbf{w}_\ell\}$$ is linearly independent.  I told him that, while this might work in special cases, in general this approach can lead to mistakes.  I promised him I would post an example if he asked about this on Piazza.  Well, time passed, and Farouq didn't post so neither did I.

But I wanted to resolve this by giving a simple example that shows what can go wrong.  Suppose $$k=2$$ and $$\{\mathbf{v}_1, \mathbf{v}_2\} = \{(1,0), (0,1)\}$$, and suppose $$\ell=1$$ and $$\{\mathbf{w}_1\} = \{(1,1)\}$$.  Then clearly $$\{\mathbf{v}_1, \mathbf{w}_1\}$$ and $$\{\mathbf{v}_2, \mathbf{w}_1\}$$ are both linearly independent sets, but $$\{\mathbf{v}_1, \mathbf{v}_2, \mathbf{w}_1\}$$ is not linearly independent.

So you see that just showing each vector from the first set cannot be written as a linear combination of vectors from the second set does not guarantee that when you take the union of the two sets it will still be linearly independent.
