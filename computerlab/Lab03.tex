\documentclass[fleqn,11pt]{paper}

%% This is the Homework LaTeX template.  Use this file to fill in your solutions. 
%%
%% Notes: 
%%    1. If possibly, try to write your answers inside a \begin{solution}...\end{solution}
%%    environment.
%%
%%    2. If you enter your answers into this Homework*.tex source file, you
%%    will have to compile it into a pdf document.  There are a number of ways to
%%    do that. Probably the easiest is to use the website called ShareLaTeX.com.
%%    Alternatively,
%%
%%       Mac OS X users: you might try MacTeX. 
%%       Linux users: most come with TeX; otherwise do a full install of TeXLive.
%%       Windows users: ...proTeXt maybe? (or switch to a better operating system) 
%%
%%       There is a Makefile in this directory, so (on Linux) you could just 
%%       enter `make` to compile all the Homework*.tex files at once (assuming
%%       you have everything set up properly).
%%
%%    3. Please don't hesitate to inform the professor if you have trouble, or open
%%       a ``New issue'' on GitHub or post to Piazza or ask in lecture.
%%
%%    4. Update the title and date if necessary.
         \title{Math 317: Computer Lab 3}
         \date{Due: 25 March 2016, 2pm}


%%%%%%%%%%%%%%%%%%%%%%%%%%%%%%%%%%%%%%%%
% Basic packages
%%%%%%%%%%%%%%%%%%%%%%%%%%%%%%%%%%%%%%%%
\usepackage[letterpaper,top=3cm,bottom=3cm,left=2.5cm,right=2.5cm]{geometry}
\usepackage{tikz-cd}
\usepackage{scalefnt}
\usepackage{amsmath,amsthm,amssymb}
\usepackage{mathtools}
\usepackage{etoolbox}
\usepackage{fancyhdr}
\usepackage{xcolor}
\usepackage[colorlinks=true,urlcolor=black,linkcolor=black,citecolor=black]{hyperref}
\usepackage{xspace}
\usepackage{comment}
\usepackage{sagetex}
\usepackage{url} % for url in bib entries
\usepackage{mathrsfs}
\usepackage[T1]{fontenc}
\usepackage[utf8]{inputenc}
\usepackage{tikz}
\usetikzlibrary{shadows}

\newcommand*\keystroke[1]{%
  \tikz[baseline=(key.base)]
    \node[%
      draw,
      fill=white,
      drop shadow={shadow xshift=0.25ex,shadow yshift=-0.25ex,fill=black,opacity=0.75},
      rectangle,
      rounded corners=2pt,
      inner sep=1pt,
      line width=0.5pt,
      font=\scriptsize\sffamily
    ](key) {#1\strut}
  ;
}



\theoremstyle{remark}
\newtheorem{theorem}{Theorem}
\newtheorem{exercise}{Exercise}
\newtheorem*{prop}{Proposition}
\newtheorem{problem}{Problem}
\newtheorem*{prob}{Problem}
\newtheorem*{solution}{{\bf Solution}}
\newtheorem*{hint}{{\it Hint}}
\newtheorem*{ex}{Exercise}


%%%%%%%%%%%%%%%%%%%%%%%%%%%%%%%%%%%%%%%%%%%%%%%%%%
%% Surround the problem and solution with 
%% \begin{ProbBox}  and   \end{ProbBox}
%% to prevent pagebreaks.
\newenvironment{ProbBox}{\noindent\begin{minipage}{\linewidth}}{\end{minipage}}

%%%%%%%%%%%%%%%%%%%%%%%%
% Fancy page style     %
%%%%%%%%%%%%%%%%%%%%%%%%
\pagestyle{fancy}
\newcommand{\metadata}[2]{
  \lhead{}
  \chead{}
  \rhead{\bfseries Math 317: Linear Algebra}
  \lfoot{#1}
  \cfoot{#2}
  \rfoot{\thepage}
}
\renewcommand{\headrulewidth}{0.4pt}
\renewcommand{\footrulewidth}{0.4pt}


%%%%%%%%%%%%%%%%%%%%%%%%%%%%%%%%%%
% Customize list enviroonments   %
%%%%%%%%%%%%%%%%%%%%%%%%%%%%%%%%%%
% package to customize three basic list environments: enumerate, itemize and description.
%% \usepackage{enumitem}
%% \setitemize{noitemsep, topsep=0pt, leftmargin=*}
%% \setenumerate{noitemsep, topsep=0pt, leftmargin=*}
%% \setdescription{noitemsep, topsep=0pt, leftmargin=*}

\usepackage{enumerate}

%%%%%%%%%%%%%%%%%%%%%%%%%%%%
%% Space between problems  %
%%%%%%%%%%%%%%%%%%%%%%%%%%%%
\newrobustcmd*{\probskip}{\vskip1cm}

%%    Try to use standard notation, as used in class and in the textbook.
%%    For the most basic symbols, you may wish to use LaTeX macros to keep 
%%    the conventions you use consistent and easy to remember.

%%%%%%%%%%%%%%%%%%%%%%%%%%
%%    Math shortcuts     %
%%%%%%%%%%%%%%%%%%%%%%%%%%
\newcommand\join{\ensuremath{\vee}}
\newcommand\meet{\ensuremath{\wedge}}
\newcommand\R{\fld{R}}
\newcommand\Z{\fld{Z}}
\newcommand\C{\fld{C}}
\newcommand\Q{\fld{Q}}
\newcommand\proj{\ensuremath{\operatorname{proj}}}
\newcommand\End{\ensuremath{\operatorname{End}}}
\newcommand\Aut{\ensuremath{\operatorname{Aut}}}
\newcommand\Hom{\ensuremath{\operatorname{Hom}}}
\newcommand{\Aff}{\ensuremath{\operatorname{Aff}}}
\newcommand{\ann}[1]{\ensuremath{\operatorname{ann}(#1)}}
\newcommand{\id}{\ensuremath{\operatorname{id}}}
\newcommand{\nulity}[1]{\ensuremath{\operatorname{null}(#1)}}
\renewcommand{\ker}[1]{\ensuremath{\operatorname{ker}(#1)}}
\renewcommand{\dim}[1]{\ensuremath{\operatorname{dim}(#1)}}
\newcommand\im[1]{\ensuremath{\operatorname{im}(#1)}}
\newcommand{\rank}[1]{\ensuremath{\operatorname{rank}(#1)}}
\newcommand{\trace}[1]{\ensuremath{\operatorname{trace}(#1)}}
\renewcommand{\phi}{\ensuremath{\varphi}}

\renewcommand{\vec}[1]{\mathbf{#1}}
         \newcommand\alg[1]{\ensuremath{\mathbf{#1}}}
         \newcommand{\<}{\ensuremath{\langle}}
         \renewcommand{\>}{\ensuremath{\rangle}}
         \newcommand\fld[1]{\ensuremath{\mathbb{#1}}}

%%       To make a boldface vector, use backslash v in front of the 
%        letter and add a new command for that letter here.
         \newcommand\va{\vec{a}}
         \newcommand\vb{\vec{b}}
         \newcommand\vu{\vec{u}}
         \newcommand\vv{\vec{v}}
         \newcommand\vw{\vec{w}}
         \newcommand\vx{\vec{x}}
         \newcommand\vy{\vec{y}}
         \newcommand\vz{\vec{z}}
         \newcommand\vzero{\vec{0}}

         \newcommand\sP{\ensuremath{\mathscr P}}
         \newcommand\Span{\ensuremath{\operatorname{Span}}}

%% ---for augmented matrices---
\newenvironment{amatrix}[1]{%
  \left[\begin{array}{@{}*{#1}{r}|rr@{}}
}{%
  \end{array}\right]
}
\newenvironment{amatrix2}[1]{%
  \left[\begin{array}{@{}*{#1}{r}|rr@{}}
}{%
  \end{array}\right]
}

\begin{document}

         \metadata{Lab 3}{25 March 2016}
         \author{NAME:}
%%       Example:
%%         \metadata{Name: William DeMeo}{Lab 1 (due: 2016/01/29 by end of lab)}
%%         \author{NAME: William DeMeo}
%%

\maketitle


%%%%%%%%%%%%%%%%%%%%%%%%%%%%%%%%%%%%%%%%%%%%%%%%%%%%%%%%%

\noindent {\bf Instructions.}
\begin{itemize}
\item This assignment must be completed in the Carver 449 computer lab by 2pm on
  \begin{quote}
    {\bf Friday March 25.}  
  \end{quote}
  Late submissions will not be accepted.
\item When you have finished working or it is 2pm (whichever comes first): 
  \begin{enumerate}

  \item {\bf Ask the professor to check your work.}
  \item Save your worksheet and name it Lab03.sws. 
  \item {\bf Submit your Lab03.sws file on Blackboard}.
  \item Stop your Sage worksheet (Action $\rightarrow$ Save and quit worksheet; or use Stop button).
  \item Sign out of your Sage account and log off the computer.
  \end{enumerate}
\end{itemize}

%--  Part 1  ---------------------------------------------------
\part*{Part 1: Sage recap/review}
The purpose of the first part of this lab is simply to recapitulate
some of the things we learned about Sage in previous computer lab assignments,
in case you are a bit rusty.
Although this part will not be graded, you are strongly
encouraged to try out all of the examples yourself (by entering them in a Sage
worksheet). You need to be comfortable with the commands covered in
Sections~\ref{sec:matrices-sage} and~\ref{sec:vectors-sage} in order to complete the second part of the lab.



\section{Number systems}
\label{sec:number-systems}
Sage knows a wide variety of sets of numbers. These are known as ``rings'' or
``fields,'' but we may call them ``number fields'' or ``scalar fields.''  (We
need not worry about the differences for now.)
Examples include the integers and the rational numbers---these sets are denoted
by $\Z$ and $\Q$, respectively---as well as the real and complex number fields,
denoted $\R$ and $\C$, respectively.  Sage has special notation for each of
these sets of numbers, and these are listed in the table below:

\begin{center}
\begin{tabular}{|l|l|l|l|}
  \hline
  &&{\bf Standard} & {\bf Sage}\\
  {\bf Set} & {\bf Description} & {\bf notation} & {\bf notation} \\[4pt]
  \hline
  Integers & $\{\dots, -1, 0, 1, 2, \dots\}$& $\Z$ & {\tt ZZ} \\[4pt]
  \hline
  Rational numbers & $\{\frac{p}{q} : p, q \in \Z, q \neq 0\}$ & $\Q$ & {\tt QQ} \\[4pt] 
  \hline
  Real numbers &  $(-\infty, \infty)$ & $\R$ & {\tt RR} \\[4pt] 
  \hline
  Complex numbers & $\{a + ib : a, b \in \R\}$ &  $\C$ & {\tt CC}\\[4pt]
  \hline
\end{tabular}
\end{center}

Thus, {\tt RR} denotes (Sage's representation of) the set of real numbers (up to
``reasonable'' precision), and {\tt CC} is (Sage's representation of) the complex
numbers (up to ``reasonable'' precision).  In any computer
system, there are complications surrounding the inability of digital
arithmetic to accurately represent all real (or complex) numbers. In contrast,
Sage can represent rational numbers exactly as the quotient of two (perhaps very
large) integers.  So, for now at least, we will use {\tt QQ} as our main number system
so we can concentrate on understanding the key concepts and ignore the fact that
computers have trouble faithfully representing real and complex numbers.

\section{Matrices in Sage}
\label{sec:matrices-sage}
Sage is largely ``object-oriented,'' which means many functions are associated
with a specific ``object'' and are invoked using the ``dot'' notation. For
example a matrix {\tt A} is represented in Sage as a ``matrix object,'' and
the command {\tt A.nrows()} returns the number of rows of the matrix {\tt A}.
\begin{sageblock}
    sage: A = matrix(QQ, 2, 3, [[1,2,3],[4,5,6]])
    sage: print A
      [1 2 3]
      [4 5 6]

    sage: A.nrows(), A.ncols()
      (2, 3)

    sage: A.base_ring()
      Rational Field

    sage: A.parent()
      Full MatrixSpace of 2 by 3 dense matrices over Rational Field
\end{sageblock}

\section{Vectors in Sage}
\label{sec:vectors-sage}
Vectors in Sage are built, manipulated and interrogated in much the same way
as matrices. However as simple lists, they are simpler to construct
and manipulate. Sage will print a vector across the screen, even if we wish to
interpret it as a column vector. It will be delimited by parentheses ((,)) which allows
us to distinguish a vector from a matrix with just one row, if we look carefully. The
number of slots in a vector is not referred to in Sage as rows or columns, but rather
by ``degree.'' Here are some examples:

\begin{sageblock}
    sage: v = vector(QQ, 4, [1, 1/2, 1/3, 1/4])
    sage: print v
      (1, 1/2, 1/3, 1/4)

    sage: v.degree()
      4
  
    sage: v.parent()
      Vector space of dimension 4 over Rational Field

    sage: v[2]
      1/3

    sage: w = vector([1, 2, 3, 4, 5, 6])
    sage: print w
      (1, 2, 3, 4, 5, 6)

    sage: w.degree()
      6

    sage: w.parent()
      Ambient free module of rank 6 over the principal ideal domain Integer Ring

    sage: w[3]
      4
\end{sageblock}
Notice that if you use commands like {\tt .parent()} you will sometimes see references
to ``free modules.'' This is a technical generalization of the notion of a vector, which
is beyond the scope of this course, so just mentally convert to vectors when you see
this term.

The zero vector is super easy to build, but be sure to specify what number system
your zero is from.
\begin{sageblock}
    sage: z = zero_vector(QQ, 5)
    sage: print z
      (0, 0, 0, 0, 0)

\end{sageblock}

\begin{exercise}
  \label{ex:vectors}
  Login to your Sage account, open a new worksheet and name it Lab03.
  Then try out all of the examples above. (Enter what appears after the
  {\tt sage:}~prompt in a worksheet cell and click \keystroke{evaluate}; alternatively,
  you can evaluate a worksheet cell with the keyboard shortcut
  \keystroke{shift}+ \keystroke{enter}.)
  Also note, you can put multiple lines in a single worksheet cell and evaluate
  them all at once.
\end{exercise}

%--  Part 2  ---------------------------------------------------
\part*{Part 2: Spanning Sets \& the Four Fundamental Subspaces}
\section{Spanning Sets}
One of Sage's strengths is the ability to create infinite sets, such as the span of a set of
vectors, from finite descriptions. In other words, we can take a finite set with just a
handful of vectors and Sage will create the set that is the span of these vectors, which
is an infinite set. Here we learn how to do this for the example vector
space $\Q^4 = \{(x_1, x_2, x_3, x_4) : x_i \in \Q\}$, which Sage denotes
by {\tt QQ\^{}4}. 
The key command is the vector space method {\tt .span()}.

\begin{exercise}
  Continuing with the worksheet (named Lab03) that you created in
  Exercise~\ref{ex:vectors} above, try out each example below. 
\end{exercise}
\begin{sageblock}
    sage: V = QQ^4
    sage: v1 = vector(QQ, [1,1,2,-1])
    sage: v2 = vector(QQ, [2,3,5,-4])
    sage: W = V.span([v1, v2])
    sage: print W
      Vector space of degree 4 and dimension 2 over Rational Field
      Basis matrix:
      [ 1 0 1 1]
      [ 0 1 1 -2]

    sage: x = 2*v1 + (-3)*v2
    sage: print x
      (-4, -7, -11, 10)

    sage: x in W
      True

    sage: y = vector(QQ, [3, -1, 2, 2])
    sage: y in W
      False

    sage: u = vector(QQ, [3, -1, 2, 5])
    sage: u in W
      True

    sage: W <= V
      True
\end{sageblock}
Most of the above should be fairly self-explanatory, but a few comments are in
order. The span, {\tt W}, is constructed with the {\tt .span()} method,
which accepts a list of vectors and is called using the ``dot'' syntax (since it
is a method of a vector space object).
You can see the number system (Rational Field) and the number
of entries in each vector (degree 4). 

Notice that the span (and the {\tt .span()} command) is a
function that takes as input a finite set of vectors and generates as output the
span of these vectors. As we know, the result in an infinite collection of vectors.
Of course, Sage does not literally construct all of the vectors in the span at once,
since it's impossible to fit all of them in the computers finite physical memory.  However,
Sage provides us with a means of accessing any (finite) number of vectors in the
span, and, importantly, Sage can check whether any given vector belongs to the span.\footnote{If
  you have studied computer 
  science or programming, then you have probably heard of recursion, streams, and the like,
  and you should not be surprised that a computer can \emph{represent} and
  \emph{compute with} infinite sets, despite the fact that
  a computer is a finite entity with finite amount of physical
  memory. For example, a ``stream'' is basically just an infinite list
  that gives us access to its elements as needed... as many as we need, the
  stream will generate for us... but we are not allowed to ask for the entire list at once!}

In the example above, we know the vector {\tt x} will be in the span {\tt W} since we built
it as a linear combination of {\tt v1} and {\tt v2}. The vectors {\tt y} and
{\tt u} are a bit more mysterious, but Sage
can answer the membership question easily for both of these vectors as well. 
In fact, from your experience in class and previous computer labs, you already know
how to use Sage to prove that {\tt y} and {\tt u} belong to {\tt W}.

\begin{exercise}~\\
  If possible, use Sage to write {\tt y} as a linear combination of the vectors {\tt v1} and 
  {\tt v2}. Do the same for {\tt u}. That is, find the coefficients $c_1, c_2, d_1, d_2$,
  so that $\vy = c_1\vv_1 +c_2 \vv_2$ and $\vu = d_1\vv_1 +d_2 \vv_2$. Write these coefficients
  in the blanks spaces provided below.  If it's not possible, explain why not.
\end{exercise}

%\vskip5mm
\[
  \vy = \begin{bmatrix*}[r] 3 \\ -1 \\ 2 \\ 2\end{bmatrix*}= \text{\underline{\phantom{XXX}}}\vv_1 + \text{\underline{\phantom{XXX}}} \vv_2,
\qquad
\qquad
  \vu = \begin{bmatrix*}[r] 3 \\ -1 \\ 2 \\ 5\end{bmatrix*}= \text{\underline{\phantom{XXX}}}\vv_1 + \text{\underline{\phantom{XXX}}} \vv_2.
\]
\vskip5mm



\section{The Four Fundamental Subspaces}
\subsection{The Null Space $\operatorname{N}(A)$}
Sage can compute the null space of a matrix. Again, this may seem rather
remarkable, since the null space is very often an infinite set!
The null space command is quite powerful since, as we have learned,
there is a lot of theory associated with the null space; e.g., if we know
the null space of a given matrix, then we know a lot about (among other things)
systems of linear equations associated with that matrix.

One way to get Sage to build the null space of a matrix is with the
{\tt .right\_kernel()} command.
We experiment with this command below and, to ensure that the presentation of results
is consistent with our previous work and the textbook, we will always use the option {\tt basis="pivot"}.  (Also, for reasons that are not important
right now, we must continue to define matrices over the rational numbers.)
\begin{sageblock}
    sage: A = matrix(QQ, [[ 1, 4, 0, -1, 0, 7, -9],
                          [ 2, 8, -1, 3, 9, -13, 7],
                          [ 0, 0, 2, -3, -4, 12, -8],
                          [-1, -4, 2, 4, 8, -31, 37]])
    
    sage: NA = A.right_kernel(basis="pivot")  # (name this whatever you want...
                                              # ...but NA seems reasonable)
    sage: print NA
      Vector space of degree 7 and dimension 4 over Rational Field
      User basis matrix:
      [-4 1 0 0 0 0 0]
      [-2 0 -1 -2 1 0 0]
      [-1 0 3 6 0 1 0]
      [ 3 0 -5 -6 0 0 1]
\end{sageblock}
(Stuff that appears after a hashtag symbol {\tt \#} is called a comment; Sage
ignores all comments.)\\
\\
We can test membership in {\tt NA}, 
\begin{sageblock}
    sage: x = vector(QQ, [3, 0, -5, -6, 0, 0, 1])
    sage: x in NA
      True

    sage: vector(QQ, [-4, 1, -3, -2, 1, 1, 1]) in NA  
      True     # (we needn't name a vector to test its membership in NA)

    sage: vector(QQ, [1, 0, 0, 0, 0, 0, 2]) in NA
      False
\end{sageblock}
We can also verify that {\tt NA} is an
infinite set,
\begin{sageblock}
    sage: NA.is_finite()
      False
\end{sageblock}
\vskip3mm
\noindent As we also know, {\it sometimes} the null space is finite.  When does this
occur?
\vskip3mm
\begin{exercise}
  \label{ex:null-space-oper}
  Build your own matrix (call it {\tt F}) such that the following command returns {\tt True}.
  \begin{sageblock}
    sage: F.right_kernel(basis="pivot").is_finite()
  \end{sageblock}
\end{exercise}
\vskip3mm
\noindent If you're having trouble with Exercise~\ref{ex:null-space-oper}, here's a hint:
Since the null space of {\tt F} is finite, we can list
its vectors. For example, if our matrix happens to have two columns, then we will observe
\begin{sageblock}
    sage: F.right_kernel(basis="pivot").list()
      [(0, 0)]
\end{sageblock}
\vskip5mm


\subsection{The Column Space $\operatorname{C}(A)$}
Using span, we can expand our techniques for checking the consistency
of a linear system $A\vx = \vb$. Recall that the system is consistent if and only if 
$\vb$ is a linear combination of the columns of $A$.
So consistency of a system is equivalent to the membership of $\vb$
in the span of the columns of $A$.

We make use of the matrix method {\tt columns()} to get all of the columns into a list at once.
\begin{sageblock}
    sage: A = matrix(QQ, [[33, -16, 10, -2],
                          [99, -47, 27, -7],
                          [78, -36, 17, -6],
                          [-9,   2,  3,  4]])
    sage: column_list = A.columns()
\end{sageblock}
    %% sage: print column_list
    %%     [(33, 99, 78, -9), (-16, -47, -36, 2), (10, 27, 17, 3), (-2, -7, -6, 4)]

    %% sage: CA = (QQ^4).span(column_list)
    %% sage: b = vector(QQ, [-27, -77, -52, 5])
    %% sage: b in CA
    %%     True
%% \end{sageblock}
\vskip4mm
\begin{exercise}
Let {\tt b = vector(QQ, [-27, -77, -52, 5])}.
Use the Sage commands {\tt columns} and {\tt (QQ\^{}4).span} to verify that
the system {\tt Ax = b} is consistent.
%% $A\vx = \vb$ is consistent.\\[4pt]
\vskip4mm
\noindent [{\it Hint:} write a line in Sage that tests for membership of {\tt b} in the column space of {\tt A}
and returns {\tt True}.]
\end{exercise}
\newpage
\section*{Alternatives}
Although the above works perfectly well, Sage has its own column space command,
called {\tt .column\_space()}.  It's very convenient, so let's try it out.
\begin{sageblock}
    sage: A = matrix(QQ, [[ 1,-1, 0],
                          [ 2, 3, 9],
                          [ 0, -3, -4],
                          [-1, 4, 8]])
    sage: CA = A.column_space()
    sage: CA == (QQ^4).span(A.columns())  # check we get the same answer either way
      True
\end{sageblock}
That the equality test in the last line above returns {\tt True} provides evidence 
that our means of computing the column space of {\tt A} is equivalent to
Sage's build-in {\tt column\_space()} command.

Now if, say, {\tt b = vector(QQ, [-16, 43, -10, 126])}, then we can check that
the system {\tt Ax = b} is consistent as follows:
\begin{sageblock}
    sage: b = vector(QQ, [-16, 43, -10, 126])
    sage: b in A.column_space()
      True
\end{sageblock}
Knowing that {\tt b} is in the column space of {\tt A}, we can now
ask Sage to solve for {\tt x} in the system {\tt Ax = b}, without fear of
producing an error.
\begin{sageblock}
    sage: A.solve_right(b)
      (-46, -30, 25)
\end{sageblock}

\section*{Where to go from here?}
\noindent To gain more familiarity with Sage, continue exploring on your own.  Here are
some suggestions: construct a right-hand side vector for which the system above is
inconsistent, then look up what Sage commands are available for least squares
solutions and/or projections, and apply these to the example you just constructed to find the
vector in the column space of $A$ that is closest to your right-hand side vector.

\end{document}





Earlier we discussed solutions to systems of equations and the membership
of a vector $\vb$ in the span of the columns of the matrix $A$. 
We can now be slightly more efficient, while still using the same ideas. 
For example,
\begin{sageblock}
    sage: A = matrix(QQ, [[ 2, 1, 7, -7],
                          [-3, 4, -5, -6],
                          [ 1, 1, 4, -5]])
    sage: b = vector(QQ, [8, -12, 4])
    sage: c = vector(QQ, [2, 3, 2])
    sage: CA = A.column_space()
    sage: A.solve_right(b)
      (4, 0, 0, 0)
\end{sageblock}
\subsection{The Row Space and Left Null Space}
For this section, we'll want to know how to computer inner products.
\begin{sageblock}
    S = [x1, x2, x3, x4]
    ips = [S[i].hermitian_inner_product(S[j])
    for i in range(3) for j in range(i+1,3)]
    all([ip == 0 for ip in ips])
\end{sageblock}
Notice how the list comprehension computes each pair just once, and never checks
the inner product of a vector with itself.


