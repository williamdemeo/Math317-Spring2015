\documentclass[fleqn,11pt]{paper}

%% This is the Homework LaTeX template.  Use this file to fill in your solutions. 
%%
%% Notes: 
%%    1. If possibly, try to write your answers inside a \begin{solution}...\end{solution}
%%    environment.
%%
%%    2. If you enter your answers into this Homework*.tex source file, you
%%    will have to compile it into a pdf document.  There are a number of ways to
%%    do that. Probably the easiest is to use the website called ShareLaTeX.com.
%%    Alternatively,
%%
%%       Mac OS X users: you might try MacTeX. 
%%       Linux users: most come with TeX; otherwise do a full install of TeXLive.
%%       Windows users: ...proTeXt maybe? (or switch to a better operating system) 
%%
%%       There is a Makefile in this directory, so (on Linux) you could just 
%%       enter `make` to compile all the Homework*.tex files at once (assuming
%%       you have everything set up properly).
%%
%%    3. Please don't hesitate to inform the professor if you have trouble, or open
%%       a ``New issue'' on GitHub or post to Piazza or ask in lecture.
%%
%%    4. Update the title and date if necessary.
         \title{Math 317: Computer Lab 4}
         \date{Due: 29 April 2016, 2pm}


%%%%%%%%%%%%%%%%%%%%%%%%%%%%%%%%%%%%%%%%
% Basic packages
%%%%%%%%%%%%%%%%%%%%%%%%%%%%%%%%%%%%%%%%
\usepackage[letterpaper,top=3cm,bottom=3cm,left=2.5cm,right=2.5cm]{geometry}
\usepackage{tikz-cd}
\usepackage{verbatim}
\usepackage{scalefnt}
\usepackage{amsmath,amsthm,amssymb}
\usepackage{mathtools}
\usepackage{etoolbox}
\usepackage{fancyhdr}
\usepackage{xcolor}
\usepackage[colorlinks=true,urlcolor=black,linkcolor=black,citecolor=black]{hyperref}
\usepackage{xspace}
\usepackage{comment}
\usepackage{sagetex}
\usepackage{url} % for url in bib entries
\usepackage{mathrsfs}
\usepackage[T1]{fontenc}
\usepackage[utf8]{inputenc}
\usepackage{tikz}
\usetikzlibrary{shadows}
\usepackage[mathcal]{euscript}

\newcommand*\keystroke[1]{%
  \tikz[baseline=(key.base)]
    \node[%
      draw,
      fill=white,
      drop shadow={shadow xshift=0.25ex,shadow yshift=-0.25ex,fill=black,opacity=0.75},
      rectangle,
      rounded corners=2pt,
      inner sep=1pt,
      line width=0.5pt,
      font=\scriptsize\sffamily
    ](key) {#1\strut}
  ;
}


\usepackage{fcla-math}
%switches}
\theoremstyle{remark}
\newtheorem{theorem}{Theorem}
\newtheorem{exercise}{Exercise}
\newtheorem*{prop}{Proposition}
\newtheorem{problem}{Problem}
\newtheorem*{prob}{Problem}
\newtheorem*{solution}{{\bf Solution}}
\newtheorem*{hint}{{\it Hint}}
\newtheorem*{ex}{Exercise}


%%%%%%%%%%%%%%%%%%%%%%%%%%%%%%%%%%%%%%%%%%%%%%%%%%
%% Surround the problem and solution with 
%% \begin{ProbBox}  and   \end{ProbBox}
%% to prevent pagebreaks.
\newenvironment{ProbBox}{\noindent\begin{minipage}{\linewidth}}{\end{minipage}}

%%%%%%%%%%%%%%%%%%%%%%%%
% Fancy page style     %
%%%%%%%%%%%%%%%%%%%%%%%%
\pagestyle{fancy}
\newcommand{\metadata}[2]{
  \lhead{}
  \chead{}
  \rhead{\bfseries Math 317: Linear Algebra}
  \lfoot{#1}
  \cfoot{#2}
  \rfoot{\thepage}
}
\renewcommand{\headrulewidth}{0.4pt}
\renewcommand{\footrulewidth}{0.4pt}


%%%%%%%%%%%%%%%%%%%%%%%%%%%%%%%%%%
% Customize list enviroonments   %
%%%%%%%%%%%%%%%%%%%%%%%%%%%%%%%%%%
% package to customize three basic list environments: enumerate, itemize and description.
%% \usepackage{enumitem}
%% \setitemize{noitemsep, topsep=0pt, leftmargin=*}
%% \setenumerate{noitemsep, topsep=0pt, leftmargin=*}
%% \setdescription{noitemsep, topsep=0pt, leftmargin=*}

\usepackage{enumerate}

%%%%%%%%%%%%%%%%%%%%%%%%%%%%
%% Space between problems  %
%%%%%%%%%%%%%%%%%%%%%%%%%%%%
\newrobustcmd*{\probskip}{\vskip1cm}

%%    Try to use standard notation, as used in class and in the textbook.
%%    For the most basic symbols, you may wish to use LaTeX macros to keep 
%%    the conventions you use consistent and easy to remember.

%%%%%%%%%%%%%%%%%%%%%%%%%%
%%    Math shortcuts     %
%%%%%%%%%%%%%%%%%%%%%%%%%%
\newcommand\join{\ensuremath{\vee}}
\newcommand\meet{\ensuremath{\wedge}}
\newcommand\R{\fld{R}}
%\newcommand\Z{\fld{Z}}
\newcommand\C{\fld{C}}
\newcommand\Q{\fld{Q}}
\newcommand\proj{\ensuremath{\operatorname{proj}}}
\newcommand\End{\ensuremath{\operatorname{End}}}
\newcommand\Aut{\ensuremath{\operatorname{Aut}}}
\newcommand\Hom{\ensuremath{\operatorname{Hom}}}
\newcommand{\Aff}{\ensuremath{\operatorname{Aff}}}
\newcommand{\ann}[1]{\ensuremath{\operatorname{ann}(#1)}}
\newcommand{\id}{\ensuremath{\operatorname{id}}}
\newcommand{\nulity}[1]{\ensuremath{\operatorname{null}(#1)}}
\renewcommand{\ker}[1]{\ensuremath{\operatorname{ker}(#1)}}
\renewcommand{\dim}[1]{\ensuremath{\operatorname{dim}(#1)}}
\newcommand\im[1]{\ensuremath{\operatorname{im}(#1)}}
%\newcommand{\rank}[1]{\ensuremath{\operatorname{rank}(#1)}}
%% \newcommand{\trace}[1]{\ensuremath{\operatorname{trace}(#1)}}
\renewcommand{\phi}{\ensuremath{\varphi}}

\renewcommand{\vec}[1]{\mathbf{#1}}
         \newcommand\alg[1]{\ensuremath{\mathbf{#1}}}
         \newcommand{\<}{\ensuremath{\langle}}
         \renewcommand{\>}{\ensuremath{\rangle}}
         \newcommand\fld[1]{\ensuremath{\mathbb{#1}}}

%%       To make a boldface vector, use backslash v in front of the 
%        letter and add a new command for that letter here.
         \newcommand\sB{\ensuremath{\mathcal B}}
         \newcommand\sE{\ensuremath{\mathcal E}}
         \newcommand\va{\vec{a}}
         \newcommand\vb{\vec{b}}
         \newcommand\vu{\vec{u}}
         \newcommand\vv{\vec{v}}
         \newcommand\vw{\vec{w}}
         \newcommand\vx{\vec{x}}
         \newcommand\vy{\vec{y}}
         \newcommand\vz{\vec{z}}
         \newcommand\vzero{\vec{0}}

         \newcommand\sP{\ensuremath{\mathscr P}}
         \newcommand\sV{\ensuremath{\mathcal V}}
         \newcommand\sW{\ensuremath{\mathcal W}}
         \newcommand\Span{\ensuremath{\operatorname{Span}}}

%% ---for augmented matrices---
\newenvironment{amatrix}[1]{%
  \left[\begin{array}{@{}*{#1}{r}|rr@{}}
}{%
  \end{array}\right]
}
\newenvironment{amatrix2}[1]{%
  \left[\begin{array}{@{}*{#1}{r}|rr@{}}
}{%
  \end{array}\right]
}

\begin{document}

         \metadata{Lab 4}{29 April 2016}

\maketitle


%%%%%%%%%%%%%%%%%%%%%%%%%%%%%%%%%%%%%%%%%%%%%%%%%%%%%%%%%

\section*{Instructions}
\begin{itemize}
\item This assignment must be completed and submitted on Blackboard by 2pm on
  \begin{quote}
    {\bf Friday, April 29.}  
  \end{quote}
\item Save your worksheet and name it Lab4.sws. 
\item {\bf Submit your Lab4.sws worksheet file on Blackboard}.
\end{itemize}

%--  Part 1  ---------------------------------------------------
%% \part*{Part 1: Linear Transformations in Sage}
\section{Linear Transformations in Sage}
%% \section{Computing symbolically}
In this lab we will learn how to create a linear transformation in Sage, and try 
out some of the many operations we can perform on such objects.  As in the previous lab
assignment, rather than work over the real or complex numbers as our base number field, 
we will work (at least initially) over $\Q$, the field of 
rational numbers, as this gives us better insight into the theory we have been 
studying.\footnote{Working over $\Q$ makes it possible for Sage to perform 
exact arithmetic, which avoids the round-off errors associated with finite precision 
arithmetic over the real or complex numbers.  This makes our demonstration of the theory much
cleaner and clearer.  In  applications, we don't always have the 
luxury of working over $\Q$. Typically the results we observe when applying linear algebra 
``in the real-world'' are not as elegant as the results presented in a first course 
in linear algebra.  Nonetheless, the theoretical principles that we highlight (using 
exact arithmetic) in this lab assignment apply equally well in practical situations, 
even those that require real and complex numerical computation.}

Let $V = \Q^3$ and 
$W = \Q^4$, and consider the linear transformation $T: V \to W$ defined as follows:
%
\begin{equation}
\label{eq:1}
T(\vx) = T \colvector{x_1\\x_2\\x_3} =\colvector{-x_1 + 2x_3\\x_1 + 3x_2 + 7x_3\\x_1 + x_2 + x_3\\2x_1 + 3x_2 + 5x_3}.
\end{equation}
%
To create this linear transformation in Sage, we will create a ``symbolic'' function.
But this requires some symbolic variables which we will call \verb!x1!, \verb!x2! and \verb!x3! 
in this case.  This is done as follows:
%
\begin{sageblock}
x1, x2, x3 = var('x1, x2, x3')                                            (eq1)
\end{sageblock}
%
Now we give a name to the transformation mentioned in (\ref{eq:1}), and define it (symbolically) as follows:
%
\begin{sageblock}
Tsymb(x1, x2, x3) = [-x1+2*x3, x1+3*x2+7*x3, x1+x2+x3, 2*x1+3*x2+5*x3]    (eq2)
\end{sageblock}
%
\newpage
\begin{exercise} Define and experiment with \verb!Tsymb!, using the following steps to guide you:
  \begin{enumerate}[(a)]
  \item Create a new Sage worksheet if you haven't already, and name it something like
    \verb!Lab4!. 
    Enter the above expressions \verb!(eq1)! and \verb!(eq2)! in a worksheet cell and then
    evaluate the cell. 
  \item Experiment with the \verb!Tsymb! object.  For example, you could 
    evaluate the function at triples of rational numbers---say, \verb!Tsymb(3, -1, 2/3)!. 
    Alternatively, you could try something a little more
    exotic, like calculating the partial derivative of \verb!Tsymb! with respect to \verb!x3!. 
    To see what methods are available to the \verb!Tsymb! object,
    try the following: 
    In a new cell of your worksheet, type \verb!Tsymb.!, making sure to include the trailing dot.  
    Then, with the cursor immediately after the dot, hit the \keystroke{Tab} key.
    You should see a drop-down list of all the functions that can be applied to the 
    object \verb!Tsymb!. Some of the entries in the list begin with the string 
    \verb!Tsymb.is_! followed by the name of some property.  Such commands allow us to
    interrogate \verb!Tsymb! to find out whether it has the given property.
  \item Find out whether \verb!Tsymb! is \emph{immutable}.
    %% with the command \verb!Tsymb.is_immutable()!. 
    If you don't know what ``immutable'' means, enter 
    the expression \verb!Tsymb.is_immutable?! and evaluate it.  By appending a question mark to a command,
    we are asking Sage to display the documentation page for that command.  \emph{The question mark is a 
      very useful feature of Sage that you should keep in mind for future reference.}

  \item Another command that should have appeared in the drop-down list when you typed \verb!Tsymb.!\keystroke{Tab} 
    is called \verb!derivative!.  You probably have an idea of what this function does,
    but just to be sure, evaluate \verb!Tsymb.derivative?! in a worksheet cell, then
    compute the partial derivative of \verb!Tsymb! with respect to \verb!x3!.
  \end{enumerate}
\end{exercise}

\subsection{Constructing a linear transformation}
At this point, our \verb!Tsymb! object is a vector of mathematical expressions 
that involve symbolic variables.
We want to tell Sage to convert it into a linear transformation, as this will 
prompt Sage to make available a whole new set of functions that we can apply to \verb!Tsymb!.
We will use the \verb?linear_transformation()? constructor, which requires 
us to carefully specify the domain and codomain, as vector spaces over the rational number field $\Q$.
%
\begin{sageblock}
V= QQ^3
W = QQ^4
T = linear_transformation(V, W, Tsymb)
\end{sageblock}
%
Now, we can try out our newly constructed linear transformation \verb?T? by evaluating it on
some input vector, say, $\vx = (3,-1,2)$,
%
\begin{sageblock}
x = vector(QQ, [3, -1, 2])
T(x)
\end{sageblock}
%
Enter these commands in a Sage worksheet cell and make sure you get the answer \verb?(1, 14, 4, 13)?.

%% Let's experiment with \verb?T? using Sage's handy tab-completion feature.  
%% Even some seemingly simple operations, such as printing \verb?T? will require some explanation.


\newpage
%--  Part 2  ---------------------------------------------------
%% \part*{Part 2: Representations of Linear Transformations}
%% \section{Matrix representations}
\section{Representations of Linear Transformations}
Recall the linear transformation from Homework Exercise 4.3.7,
\begin{equation}
\label{eq:2}
T(\vx) = T\colvector{x_1\\x_2\\x_3} =\colvector{-x_1 + 2x_2 + x_3\\x_2 + 3x_3\\x_1 - x_2 + x_3}.
\end{equation}
%
\begin{exercise} Use Sage to find the matrix representation of $T$ by following these steps.
  \begin{enumerate}[(a)]
  \item In your Sage worksheet, construct the linear transformation described by (\ref{eq:2}). 
    (Use the what you learned in Part 1.)
%
\begin{sageblock}
V = QQ^3
Tsymb = ...          (fill in the rest)
T = linear_transformation ...   
                     (be careful to specify the right domain and codomain)
\end{sageblock}
%
\item  By now you have a lot of experience with linear transformations, and with just a 
  glance at Equation (\ref{eq:2}) you can surely write down a matrix that represents 
  $T$.  (Do it!)  Of course, Sage has a command that will do this for you.  Try 
  evaluating \verb!T.matrix()! in your Sage worksheet. Is the output of this command what you expected?
  If not, maybe there is an option that makes the 
  \verb!matrix()! function produce a more satisfying output.  Try to find this option and use it.
   (Remember, you can get help with the \verb!matrix! function by entering \verb!T.matrix?!.)

\item In this part, we will use Sage to find the matrix representation $[T]_\sB$ of $T$ with respect to the basis
  $\sB = \left\{
  \begin{bmatrix*}[r]1\\0\\-1 \end{bmatrix*},
  \begin{bmatrix*}[r]0\\2\\3 \end{bmatrix*},
  \begin{bmatrix*}[r]1\\1\\1 \end{bmatrix*}\right\}$. We will do this in two different ways 
  and then check that the results we get are the same.
  
  {\bf Method 1:} 
  Define the three basis vectors in $\sB$ using the \verb!vector! constructor:  \\
  \verb!b1 = vector(QQ, [1,0,-1])!, etc.
  Construct the change-of-basis matrix $P$, and then 
  %% using the vectors \verb!b1!, \verb!b2!, and \verb!b3! \emph{as columns of} $P$. 
  compute $P^{-1} T P$. One way to construct $P$ is with the command
  \verb!matrix(QQ, [b1,b2,b3]).transpose()!.
  %% , and the inverse of a matrix \verb!M!
  %% is computed with \verb!M.inverse()!.)
  Don't forget to apply the \verb!matrix! method to $T$ before using it in matrix computations, as in\\
  \verb!TB1 = P.inverse() * T.matrix(side="right") * P!
  \\[5pt]
  {\bf Method 2:}
  Let \verb!V = QQ^3! and
  define \verb!VB = V.subspace_with_basis([b1, b2, b3])!.
  %Use the \verb!Tsymb! you already defined as an argument to the
  Use the \verb!linear_transformation! constructor, as you did above, but this time
  replace each occurrence of \verb!V!
  with \verb!VB!, since \verb!VB! is the vector space \verb!QQ^3! endowed with the new basis.
  That is, compute 
  \begin{sageblock}
   TB2 = linear_transformation(VB, VB, Tsymb)
  \end{sageblock}
    %% Tsymb(x1, x2, x3) = [-x1+2*x2+x3, x2+3*x3, x1-x2+x3]
    %% T = linear_transformation(V, V, Tsymb)
    %% TB = linear_transformation(VB, VB, Tsymb)
  Finally, check that the two methods give the same result:
  \begin{sageblock}
    TB1 == TB.matrix(side="right")      # (this should return True)
  \end{sageblock}

  
  \end{enumerate}
\end{exercise}

\section{Diagonalization: finding the best basis possible}
Given a linear transformation $T: V \to W$, we may want to find a basis that is 
in some sense optimal for representing $T$. Specifically, we often  seek a matrix
representation of $T$ that is as close to diagonal as possible.

Let $A$ be the standard basis representation of $T$, that is, $A = [T]_\sE$.   
We learned that a necessary and sufficient condition for diagonalizability
of $A$ is that the algebraic multiplicity of each eigenvalue of $A$ must be 
the same as the geometric multiplicity.
We can interpret this in a few equivalent ways: 
\begin{itemize}
\item $A$ is similar to a diagonal matrix, $D = P^{-1} A P$.
\item There is a change-of-basis that diagonalizes $A$.
\item There is a basis $\sB$ with respect to which
the $\sB$-basis representation $[T]_\sB$ is a diagonal matrix.
\end{itemize}
These statements all say the same thing.  Importantly, when these 
conditions hold, the basis $\sB$ consists of eigenvectors of 
$A$, and $D$ has the eigenvalues of $A$ along the main diagonal.

\begin{exercise}
Example 5 on page 273 of our textbook involves two matrices,
\[
A =\begin{bmatrix*}[r]-1&4&2\\-1&3 &1\\-1&2 &2\end{bmatrix*} \qquad \text{ and } \qquad
B =\begin{bmatrix*}[r]0&3&1\\-1&3 &1\\0&1 &1\end{bmatrix*}.
\]
Use Sage to compute the characteristic polynomial and the eigenvalues/vectors of 
$A$. Check that $A$ is diagonalizable and 
perform the similarity transformation that gives the diagonalization. 
Use the steps below as a guide. Later, come back and do the same $B$. What goes wrong?

\begin{enumerate}[(a)]
\item Construct the matrix $A$ with the command \verb|A = matrix(QQ, [[-1,4,2], [-1,3,1], [-1,2,2]])|, 
then use compute its characteristic polynomial.
%% $p(t) = -(t - 1)^2 (t - 2)$. 
\begin{sageblock}
  p = A.characteristic_polynomial()
  p.factor()
\end{sageblock}
\item Compute the roots of the characteristic polynomial
  with \verb|p.roots()|.  
  The output should be a list of two ordered pairs.  
  Can you make sense of this output?
\item You computed the eigenvalues of $A$ indirectly in the last step. Now compute them directly
with the Sage command \verb|A.eigenvalues()|. Compare the output to the result from part (b).
\item You found eigenvalues in the last two steps; name them \verb|ev1| and \verb|ev2|.
Compute the eigenvectors of $A$ using the \verb|right_kernel| command that we learned in
Lab 3. Recall that, 
for each eigenvalue $\lambda$, the eigenspace is the null space of $A - \lambda I$.
Here's the first one: %(assuming you gave the name \verb|ev1| to an eigenvalue in the last step):
\begin{sageblock}
I = identity_matrix(3)
E1 = (A-ev1*I).right_kernel(basis=’pivot’)
\end{sageblock}
Finally, compute the change-of-basis matrix $P$ whose columns are the eigenvectors of $A$ and
check that $P^{-1}A P$ gives a diagonal matrix with the eigenvalues of $A$ along the main diagonal.
Here's one way to do it:
\begin{sageblock}
b12 = E1.matrix().transpose()    # we want the eigenvectors as columns,
b3 = E2.matrix().transpose()     # not rows, so we apply transpose() 
P = b12.augment(b3); print P     
\end{sageblock}
After checking that the columns of \verb|P| are exactly the eigenvectors you found above,
you can compute the diagonalization of $A$ as usual: \verb|P.inverse()*A*P|.
%% That is, the eigenvalue 1
%% has algebraic multiplicity 2 and the eigenvalue 2 has algebraic multiplicity 1. To decide
%% whether the matrices are diagonalizable, we need to know the geometric multiplicity of the274
%% Chapter 6 Eigenvalues and Eigenvectors
%% eigenvalue 1. Well,
%% has rank 1 and so dim E A (1) = 2. We infer from Theorem 2.4 that A is diagonalizable.
\end{enumerate}

\end{exercise}

  \section{Notes for further study.}
  \begin{itemize}
  \item (Exercise 2)
  To experiment more with change-of-basis, see if you can come up with an example of the
  more general version of the change-of-basis theorem---Theorem 4.2, page 231 of our textbook---and
  try it out in Sage. 
  That is, given a linear transformation $T: V \to W$, and two bases $\sV$, $\sV'$ for $V$,
  and two bases $\sW$, $\sW'$ for $W$, use the formula
  $[T]_{\sV',\sW'} = Q^{-1}[T]_{\sV,\sW}P$ to compute the $\sV',\sW'$-basis representation of $T$. 
  You will have to generalize one of the two methods described in Exercise 2 (c) above, 
  but by now it should be clear how to do this. If it isn't, then 
  check out the examples on page 127 (Section MR) and page 138 (Section MRCB) 
  of the document 
  \href{http://linear.ups.edu/download/fcla-2.22-sage-4.7.1-preview.pdf}{fcla-2.22-sage-4.7.1-preview.pdf},
  which is available for download from the Resources tab of our Piazza site.
\item (Exercise 3) 
  Of course, Sage has its own built-in commands for computing eigenvectors and eigenspaces.  
  Here are some commands you could try in the context of Exercise 3 part (d): 
  \verb|A.eigenvectors_right()|, \verb|A.eigenspaces_right()|. Compare
  the results of these commands with the results produced by 
  the \verb|right_kernel| that we used above.

  See also Section EE (p. 95) of the document 
  \href{http://linear.ups.edu/download/fcla-2.22-sage-4.7.1-preview.pdf}{fcla-2.22-sage-4.7.1-preview.pdf}
  for more details about computing eigenvalues in Sage.  In particular, we want to work over a field
  that contains the roots of the characteristic polynomial.  (In Exercise 3,
  the eigenvalues were real numbers so this was not an issue.)
  \end{itemize}

\end{document}
