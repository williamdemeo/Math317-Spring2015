\documentclass[fleqn,11pt]{paper}

%% This is the Homework LaTeX template.  Use this file to fill in your solutions. 
%%
%% Notes: 
%%    1. If possibly, try to write your answers inside a \begin{solution}...\end{solution}
%%    environment.
%%
%%    2. If you enter your answers into this Homework*.tex source file, you
%%    will have to compile it into a pdf document.  There are a number of ways to
%%    do that. Probably the easiest is to use the website called ShareLaTeX.com.
%%    Alternatively,
%%
%%       Mac OS X users: you might try MacTeX. 
%%       Linux users: most come with TeX; otherwise do a full install of TeXLive.
%%       Windows users: ...proTeXt maybe? (or switch to a better operating system) 
%%
%%       There is a Makefile in this directory, so (on Linux) you could just 
%%       enter `make` to compile all the Homework*.tex files at once (assuming
%%       you have everything set up properly).
%%
%%    3. Please don't hesitate to inform the professor if you have trouble, or open
%%       a ``New issue'' on GitHub or post to Piazza or ask in lecture.
%%
%%    4. Update the title and date if necessary.
         \title{Math 317: Computer Lab 2}
         \date{Due: 12 February 2016, 2pm}


%%%%%%%%%%%%%%%%%%%%%%%%%%%%%%%%%%%%%%%%
% Basic packages
%%%%%%%%%%%%%%%%%%%%%%%%%%%%%%%%%%%%%%%%
\usepackage[letterpaper,top=3cm,bottom=3cm,left=2.5cm,right=2.5cm]{geometry}
\usepackage{tikz-cd}
\usepackage{scalefnt}
\usepackage{amsmath,amsthm,amssymb}
\usepackage{mathtools}
\usepackage{etoolbox}
\usepackage{fancyhdr}
\usepackage{xcolor}
\usepackage[colorlinks=true,urlcolor=blue,linkcolor=blue,citecolor=blue]{hyperref}
\usepackage{xspace}
\usepackage{comment}
\usepackage{sagetex}
\usepackage{url} % for url in bib entries
\usepackage{mathrsfs}

\theoremstyle{remark}
\newtheorem{theorem}{Theorem}
\newtheorem{exercise}{Exercise}
\newtheorem*{prop}{Proposition}
\newtheorem{problem}{Problem}
\newtheorem*{prob}{Problem}
\newtheorem*{solution}{{\bf Solution}}
\newtheorem*{hint}{{\it Hint}}
\newtheorem*{ex}{Exercise}


%%%%%%%%%%%%%%%%%%%%%%%%%%%%%%%%%%%%%%%%%%%%%%%%%%
%% Surround the problem and solution with 
%% \begin{ProbBox}  and   \end{ProbBox}
%% to prevent pagebreaks.
\newenvironment{ProbBox}{\noindent\begin{minipage}{\linewidth}}{\end{minipage}}

%%%%%%%%%%%%%%%%%%%%%%%%
% Fancy page style     %
%%%%%%%%%%%%%%%%%%%%%%%%
\pagestyle{fancy}
\newcommand{\metadata}[2]{
  \lhead{}
  \chead{}
  \rhead{\bfseries Math 317: Linear Algebra}
  \lfoot{#1}
  \cfoot{#2}
  \rfoot{\thepage}
}
\renewcommand{\headrulewidth}{0.4pt}
\renewcommand{\footrulewidth}{0.4pt}


%%%%%%%%%%%%%%%%%%%%%%%%%%%%%%%%%%
% Customize list enviroonments   %
%%%%%%%%%%%%%%%%%%%%%%%%%%%%%%%%%%
% package to customize three basic list environments: enumerate, itemize and description.
%% \usepackage{enumitem}
%% \setitemize{noitemsep, topsep=0pt, leftmargin=*}
%% \setenumerate{noitemsep, topsep=0pt, leftmargin=*}
%% \setdescription{noitemsep, topsep=0pt, leftmargin=*}

\usepackage{enumerate}

%%%%%%%%%%%%%%%%%%%%%%%%%%%%
%% Space between problems  %
%%%%%%%%%%%%%%%%%%%%%%%%%%%%
\newrobustcmd*{\probskip}{\vskip1cm}

%%    Try to use standard notation, as used in class and in the textbook.
%%    For the most basic symbols, you may wish to use LaTeX macros to keep 
%%    the conventions you use consistent and easy to remember.

%%%%%%%%%%%%%%%%%%%%%%%%%%
%%    Math shortcuts     %
%%%%%%%%%%%%%%%%%%%%%%%%%%
\newcommand\join{\ensuremath{\vee}}
\newcommand\meet{\ensuremath{\wedge}}
\newcommand\R{\fld{R}}
\newcommand\proj{\ensuremath{\operatorname{proj}}}
\newcommand\End{\ensuremath{\operatorname{End}}}
\newcommand\Aut{\ensuremath{\operatorname{Aut}}}
\newcommand\Hom{\ensuremath{\operatorname{Hom}}}
\newcommand{\Aff}{\ensuremath{\operatorname{Aff}}}
\newcommand{\ann}[1]{\ensuremath{\operatorname{ann}(#1)}}
\newcommand{\id}{\ensuremath{\operatorname{id}}}
\newcommand{\nulity}[1]{\ensuremath{\operatorname{null}(#1)}}
\renewcommand{\ker}[1]{\ensuremath{\operatorname{ker}(#1)}}
\renewcommand{\dim}[1]{\ensuremath{\operatorname{dim}(#1)}}
\newcommand\im[1]{\ensuremath{\operatorname{im}(#1)}}
\newcommand{\rank}[1]{\ensuremath{\operatorname{rank}(#1)}}
\newcommand{\trace}[1]{\ensuremath{\operatorname{trace}(#1)}}
\renewcommand{\phi}{\ensuremath{\varphi}}

\renewcommand{\vec}[1]{\mathbf{#1}}
         \newcommand\alg[1]{\ensuremath{\mathbf{#1}}}
         \newcommand{\<}{\ensuremath{\langle}}
         \renewcommand{\>}{\ensuremath{\rangle}}
         \newcommand\fld[1]{\ensuremath{\mathbb{#1}}}

%%       To make a boldface vector, use backslash v in front of the 
%        letter and add a new command for that letter here.
         \newcommand\va{\vec{a}}
         \newcommand\vb{\vec{b}}
         \newcommand\vu{\vec{u}}
         \newcommand\vv{\vec{v}}
         \newcommand\vw{\vec{w}}
         \newcommand\vx{\vec{x}}
         \newcommand\vy{\vec{y}}
         \newcommand\vz{\vec{z}}
         \newcommand\vzero{\vec{0}}

         \newcommand\sP{\ensuremath{\mathscr P}}
         \newcommand\Span{\ensuremath{\operatorname{Span}}}

%% ---for augmented matrices---
\newenvironment{amatrix}[1]{%
  \left[\begin{array}{@{}*{#1}{r}|rr@{}}
}{%
  \end{array}\right]
}
\newenvironment{amatrix2}[1]{%
  \left[\begin{array}{@{}*{#1}{r}|rr@{}}
}{%
  \end{array}\right]
}

\begin{document}

         \metadata{Name:}{Lab 2 (due 2/12)}
         \author{NAME:}
%%       Example:
%%         \metadata{Name: William DeMeo}{Lab 1 (due: 2016/01/29 by end of lab)}
%%         \author{NAME: William DeMeo}
%%

\maketitle


%%%%%%%%%%%%%%%%%%%%%%%%%%%%%%%%%%%%%%%%%%%%%%%%%%%%%%%%%

\noindent {\bf Instructions.}
\begin{itemize}
\item This assignment must be completed in the Carver 449 computer lab by 2pm on
  \begin{quote}
    {\bf Friday February 12.}  
  \end{quote}
  Late submissions will not be accepted.
Here is a summary of what it takes to complete each part of the lab.
More detailed instructions appear on the following pages.
\item {\bf Part 1.} To complete this part, you must login to your account on the 
  Math Department Sage server and demonstrate that you know how to input a matrix, augment a matrix with a vector,
  and compute the echelon form of an (augmented) matrix.  

  Completing Part 1 is worth 1 point. If you are not interested in completing Part 2 of the
  assignment, you may turn in your paper and leave the lab after completing Part 1. \\
  {\bf Make sure your name is on your paper and it was signed by the instructor.}
\item {\bf Part 2.} To complete Part 2, you must first download the Sage worksheet file 
  Math317-Lab2.sws from either Blackboard or GitHub, load this worksheet into your Sage session
  and follow the instructions provided.  You must then save your completed Sage worksheet 
  to a file and upload the resulting sws file to Blackboard.

\item When you have finished working or it is 2pm (whichever comes first): 
  \begin{enumerate}
  \item {\bf stop your Sage worksheet} (Action $\rightarrow$ Save and quit worksheet; or use Stop button),
  \item {\bf sign out of Sage}, 
  \item {\bf logout of your computer}, 
  \item {\bf write your name on your paper},
  \item {\bf hand it in to the instructor}.
  \end{enumerate}
\end{itemize}


%--  Part 1  ---------------------------------------------------
\section*{Part 1}
In Homework 3, we proved that the vector 
    $\vb = (1, -1, 1, -1)$ is as a linear combination 
    of the vectors 
    $\vv_1 = (1 , 0 , 1 , -2)$, 
    $\vv_2 = (0 , -1 , 0 , 1)$, and 
    $\vv_3 = (1, -2 , 1 , 0)$.  Do you remember how we did this?
    That's right! We recognized that if
    the matrix $A$ has first column $\vv_1$, second column $\vv_2$, and third
    column $\vv_3$, then writing $\vb$ as a linear combination of 
    the given vectors is the same as solving the system $A\vx = \vb$.
    That is, if we find $\vx = (x_1, x_2, x_3)$ that solves 
    this system, then $\vb = x_1\vv_1+x_2\vv_2+x_3\vv_3$ is the
    desired linear combination. 

    In this first part of this lab assignment, we will simply re-solve the above problem, but 
    this time we will make Sage do the tedious work.  Carry out the following steps:

  \begin{enumerate}
  \item Login to any machine in Carver 449, open up a browser (preferably Chrome) and navigate to

    \begin{quote}
      \url{https://sage.math.iastate.edu/}
    \end{quote}

    Login to your account on the Math Department Sage server using the login information you
    were given for Lab 1. Once you are logged in, create a new worksheet and name it Lab2.  

    \item In the Lab2 worksheet, click the first cell and type the following:
    \begin{sageblock}
      b = vector([1, -1, 1, -1]); print b
    \end{sageblock}
    Then type Shift+Enter or click `evaluate`.
    Did Sage print what you expect? If so, move on.  If not, try again and/or ask the instructor for help.
  
  \item Click somewhere inside the next cell and enter the following  expression:
    \begin{sageblock}
      A = matrix(4, 3, [1, 0, 1, 0, -1, -2, 1, 0, 1, -2, 1, 0]) 
      print A
    \end{sageblock}
    What did Sage print?  Is it a matrix with three columns, 
    $\vv_1$, $\vv_2$, $\vv_3$? If not, try again and/or your neighbor for help.
    Can you decypher the syntax we used to input this matrix $A \in \R^{4\times 3}$?
    If so, move on.  If not, think a litter harder and/or discuss it with your neighbor.
\item Next, have Sage augment the matrix $A$ with the vector $b$ by entering the following into the
  next worksheet cell:
    \begin{sageblock}
      Ab = A.augment(b, subdivide=True)
      print Ab
    \end{sageblock}
\item Ask Sage to put your augmented matrix in reduced-row echelon form:
    \begin{sageblock}
      Ab.rref()
    \end{sageblock}
    You should now see two pivots (both equal to 1).  Let the free variable $x_3 = s$ and write 
    the vector $\vb$ as a linear combination of the vectors $\vv_1$, $\vv_2$, $\vv_3$
    involving $s$.
  \end{enumerate}
\vskip5mm
\[
  \vb = \begin{bmatrix*}[r] 1 \\ -1 \\ 1 \\ -1\end{bmatrix*}= \text{\underline{\phantom{XXX}}}\vv_1 + \text{\underline{\phantom{XXX}}} \vv_2+
\text{\underline{\phantom{XXX}}} \vv_3.
\]
\vskip5mm
\noindent  If you want to stop here, ask the instructor to check your work and sign below
then save and quit. Otherwise, move on to Part 2 by loading the \href{https://github.com/williamdemeo/Math317-Spring2016/raw/master/computerlab/Math317-Lab2.sws}{Math317-Lab2.sws} file into your 
Sage session.
    \\\\
    Instructor signature: \underline{\phantom{XXXXXXXXXXXXXXXXXXXXXXXXXXXXXXXX}} (1 point)


\end{document}












%%%%%%%%%%%%%%%%%%% OLD STUFF BELOW (not used) %%%%%%%%%%%%%%%%%%%%%%%%%%%%%%%%%%%%%%%%%%%




%--  Part 2  ---------------------------------------------------
\section*{Part 2}
In this part we will use Sage to solve the last exercise of Homework 4.
Recall, in that problem, we had
  $A = \begin{bmatrix*}[r] 1&1&1\\0&1&1\\1&2&1\end{bmatrix*}$ 
    and 
    $\vb = \begin{bmatrix*}[r] 3\\0\\1\end{bmatrix*}$.
      \begin{sageblock}
        A = matrix([[1,0,1],[1,1,2],[1,1,1]]) 
      \end{sageblock}
      Use Sage to compute $A^{-1}$.
      Try changing the 2 in the matrix $A$ to 1 and then take the inverse.  What happens?
      Using your answer to (i) to solve $A\vx = \vb$.
      \item Use your answer to (ii) to express $\vb$ as a linear combination of the columns of $A$.
      \end{enumerate}

\end{document}
      
    \end{sageblock}

    If a 4 appears in the worksheet, congratulations on completing Part 1. Ask the
    instructor to check your work and sign below, then do Part 2 (or Save \& quit if you want to stop here):\\
    \\\\
    Instructor signature: \underline{\phantom{XXXXXXXXXXXXXXXXXXXXXXXXXXXXXXXX}} (1 point)

  \end{enumerate}

\newpage
