\documentclass[fleqn,11pt]{paper}

%% This is the Homework LaTeX template.  Use this file to fill in your solutions. 
%%
%% Notes: 
%%    1. If possibly, try to write your answers inside a \begin{solution}...\end{solution}
%%    environment.
%%
%%    2. If you will use acronyms, please define them in the macros.tex file.
%%
%%    3. If you enter your answers into this Homework*.tex source file, you
%%    will have to compile it into a pdf document.  There are a number of ways to
%%    do that. Probably the easiest is to use the website called ShareLaTeX.com.
%%    Alternatively,
%%
%%       Mac OS X users: you might try MacTeX. 
%%       Linux users: most come with TeX; otherwise do a full install of TeXLive.
%%       Windows users: ...proTeXt maybe? (or switch to a better operating system) 
%%
%%       There is a Makefile in this directory, so (on Linux) you could just 
%%       enter `make` to compile all the Homework*.tex files at once (assuming
%%       you have everything set up properly).
%%
%%    4. Please don't hesitate to inform the professor if you have trouble, or open
%%       a ``New issue'' on GitHub or post to Piazza or ask in lecture.
%%
%%    5. Update the title and date if necessary.
         \title{Math 317: Homework 1}
         \date{Due: 22 January 2016}


%%%%%%%%%%%%%%%%%%%%%%%%%%%%%%%%%%%%%%%%
% Basic packages
%%%%%%%%%%%%%%%%%%%%%%%%%%%%%%%%%%%%%%%%
\usepackage[letterpaper,top=3cm,bottom=3cm,left=3cm,right=3cm]{geometry}
\usepackage{tikz-cd}
\usepackage{scalefnt}
\usepackage{amsmath,amsthm,amssymb}
\usepackage{mathtools}
\usepackage{etoolbox}
\usepackage{fancyhdr}
\usepackage{xcolor}
\usepackage[colorlinks=true,urlcolor=blue,linkcolor=blue,citecolor=blue]{hyperref}
\usepackage{xspace}
\usepackage{comment}
\usepackage{url} % for url in bib entries
\usepackage{mathrsfs}

\theoremstyle{remark}
\newtheorem{theorem}{Theorem}
\newtheorem*{prop}{Proposition}
\newtheorem{problem}{Problem}
\newtheorem*{prob}{Problem}
\newtheorem*{solution}{{\bf Solution}}
\newtheorem*{hint}{{\it Hint}}
\newtheorem*{ex}{Exercise}


%%%%%%%%%%%%%%%%%%%%%%%%%%%%%%%%%%%%%%%%%%%%%%%%%%
%% Surround the problem and solution with 
%% \begin{ProbBox}  and   \end{ProbBox}
%% to prevent pagebreaks.
\newenvironment{ProbBox}{\noindent\begin{minipage}{\linewidth}}{\end{minipage}}

%%%%%%%%%%%%%%%%%%%%%%%%
% Fancy page style     %
%%%%%%%%%%%%%%%%%%%%%%%%
\pagestyle{fancy}
\newcommand{\metadata}[2]{
  \lhead{}
  \chead{}
  \rhead{\bfseries Math 317: Linear Algebra}
  \lfoot{#1}
  \cfoot{#2}
  \rfoot{\thepage}
}
\renewcommand{\headrulewidth}{0.4pt}
\renewcommand{\footrulewidth}{0.4pt}


%%%%%%%%%%%%%%%%%%%%%%%%%%%%%%%%%%
% Customize list enviroonments   %
%%%%%%%%%%%%%%%%%%%%%%%%%%%%%%%%%%
% package to customize three basic list environments: enumerate, itemize and description.
%% \usepackage{enumitem}
%% \setitemize{noitemsep, topsep=0pt, leftmargin=*}
%% \setenumerate{noitemsep, topsep=0pt, leftmargin=*}
%% \setdescription{noitemsep, topsep=0pt, leftmargin=*}

\usepackage{enumerate}

%%%%%%%%%%%%%%%%%%%%%%%%%%%%
%% Space between problems  %
%%%%%%%%%%%%%%%%%%%%%%%%%%%%
\newrobustcmd*{\probskip}{\vskip1cm}

%%    Try to use standard notation, as used in class and in the textbook.
%%    For the most basic symbols, you may wish to use LaTeX macros to keep 
%%    the conventions you use consistent and easy to remember.

%%%%%%%%%%%%%%%%%%%%%%%%%%
%%    Math shortcuts     %
%%%%%%%%%%%%%%%%%%%%%%%%%%
\newcommand\join{\ensuremath{\vee}}
\newcommand\meet{\ensuremath{\wedge}}
\newcommand\R{\fld{R}}
\newcommand\proj{\ensuremath{\operatorname{proj}}}
\newcommand\End{\ensuremath{\operatorname{End}}}
\newcommand\Aut{\ensuremath{\operatorname{Aut}}}
\newcommand\Hom{\ensuremath{\operatorname{Hom}}}
\newcommand{\Aff}{\ensuremath{\operatorname{Aff}}}
\newcommand{\ann}[1]{\ensuremath{\operatorname{ann}(#1)}}
\newcommand{\id}{\ensuremath{\operatorname{id}}}
\newcommand{\nulity}[1]{\ensuremath{\operatorname{null}(#1)}}
\renewcommand{\ker}[1]{\ensuremath{\operatorname{ker}(#1)}}
\renewcommand{\dim}[1]{\ensuremath{\operatorname{dim}(#1)}}
\newcommand\im[1]{\ensuremath{\operatorname{im}(#1)}}
\newcommand{\rank}[1]{\ensuremath{\operatorname{rank}(#1)}}
\newcommand{\trace}[1]{\ensuremath{\operatorname{trace}(#1)}}
\renewcommand{\phi}{\ensuremath{\varphi}}

\renewcommand{\vec}[1]{\mathbf{#1}}
         \newcommand\alg[1]{\ensuremath{\mathbf{#1}}}
         \newcommand{\<}{\ensuremath{\langle}}
         \renewcommand{\>}{\ensuremath{\rangle}}
         \newcommand\fld[1]{\ensuremath{\mathbb{#1}}}

%%       To make a boldface vector, use backslash v in front of the 
%        letter and add a new command for that letter here.
         \newcommand\va{\vec{a}}
         \newcommand\vb{\vec{b}}
         \newcommand\vp{\vec{p}}
         \newcommand\vu{\vec{u}}
         \newcommand\vv{\vec{v}}
         \newcommand\vw{\vec{w}}
         \newcommand\vx{\vec{x}}
         \newcommand\vy{\vec{y}}
         \newcommand\vz{\vec{z}}
         \newcommand\vzero{\vec{0}}

         \newcommand\sP{\ensuremath{\mathscr P}}
         \newcommand\Span{\ensuremath{\operatorname{Span}}}

\begin{document}

%%    INSERT YOUR NAME HERE!!!
         \metadata{Name:}{HW 2 (due 2016/01/29)}
         \author{NAME:}
%%       Example:
%%         \metadata{Name: William DeMeo}{HW 1 (due 2015/09/04)}
%%         \author{NAME: William DeMeo}
%%

\maketitle

%%  PAGE 1  %%%%%%%%%%%%%%%%%%%%%%%%%%%%%%%%%%%%%%%%%%%%%%%%%%%%%%%

%--  PROBLEM 1  ---------------------------------------------------
\begin{problem}[SA 1.4.3af]
For each of the following matrices $A$, determine its reduced echelon form and
give a general solution to $A\vx = \vzero$ in standard form.
\[
\mathrm{a.} \quad A = 
\begin{bmatrix*}[r]
  1 & 0 & -1 \\
  -2 & 3 & -1 \\
  3 & -3 & 0
\end{bmatrix*},
%% \qquad
%% \mathrm{c.} \quad A = 
%% \begin{bmatrix*}[r]
%%   1 & 2 & -1 \\
%%   1 & 3 & 1 \\
%%   2 & 4 & 3\\
%%  -1 & 1 & 6
%% \end{bmatrix*},
\qquad
\mathrm{f.} \quad A = 
\begin{bmatrix*}[r]
  1 & 2 & 0 & -1 & -1 \\
  -1 & -3 & 1 & 2 & 3\\
  1 & -1 & 3 & 1 & 1\\
  2 & -3 & 7 & 3 & 4
\end{bmatrix*}.
\]

\end{problem}



\newpage
%%  PAGE 2  %%%%%%%%%%%%%%%%%%%%%%%%%%%%%%%%%%%%%%%%%%%%%%%%%%%%%%%

%--  PROBLEM 2  ---------------------------------------------------
\begin{problem}[SA 1.4.4de]
  For the matrix $A$ and vector $\vb$ given, find the general solution of the
  equation $A\vx = \vb$ in standard form. 
\begin{enumerate}[a.]
%% \item[b.]
%% \[
%% A = 
%% \begin{bmatrix*}[r]
%%   2 & -1\\
%%   2 & 1\\
%%   -1 & 1
%% \end{bmatrix*},
%% \quad
%% \vb = \begin{bmatrix*}[r] -4\\ 0 \\ 3 \end{bmatrix*}
%% \]

\item[d.]
\[
A = 
\begin{bmatrix*}[r]
  2 & -1 & 1\\
  2 & 1 & 3\\
  1 & 1 & 2
\end{bmatrix*},
\quad
\vb = \begin{bmatrix*}[r] 1\\ -1 \\ -1 \end{bmatrix*}
\]

\item[e.]
\[
A = 
\begin{bmatrix*}[r]
  1 & 1 & 1 & 1\\
  3 & 3 & 2 & 0
\end{bmatrix*},
\quad
\vb = \begin{bmatrix*}[r] 6\\ 17 \end{bmatrix*}
\]

\end{enumerate}
\end{problem}

\newpage

%%  PAGE 3 %%%%%%%%%%%%%%%%%%%%%%%%%%%%%%%%%%%%%%%%%%%%%%%%%%%%%%%

%--  PROBLEM 3  ---------------------------------------------------
\begin{problem}[SA 1.4.7a]
  One might need to find solutions of $A\vx = \vb$ for several different
  $\vb$s, say $\vb_1, \dots, \vb_k$.
  In this situation, one can augment the matrix $A$ with all the $\vb$s
  simultaneously, forming the ``multi-augmented'' matrix
  $[ A | \vb_1\; \vb_2 \; \cdots \; \vb_k ]$.
  One can then read off the various solutions from the reduced echelon form of
  the multi-augmented matrix. Use this method to solve $A\vx = \vb_j$
  for the given matrix $A$ and vectors $\vb_j$.
\begin{enumerate}[a.]
\item
\[
A = 
\begin{bmatrix*}[r]
  1 & 0 & -1 \\
  2 & 1 & -1 \\
  -1 & 2 & 2
\end{bmatrix*},
\quad
\vb_1 = \begin{bmatrix*}[r] -1\\ 1\\ 5 \end{bmatrix*},
\quad
\vb_2 = \begin{bmatrix*}[r] 1\\ 3\\ 2 \end{bmatrix*}
\]
\end{enumerate}
\end{problem}

\newpage

%--  PROBLEM 4  ---------------------------------------------------
\begin{problem}[SA 1.4.14]
Let $A$ be an $m \times n$ matrix, and let $\vb \in \R^m$.
\begin{enumerate}[a.]
\item Show that if the vectors $\vu$ and $\vv$ in $\R^n$ are both
  solutions to $A\vx = \vb$, then $\vu - \vv$ is a solution to $A\vx = \vzero$.
\item Suppose $\vu$ is a solution to $A\vx = \vzero$ and $\vp$ is a solution to
  $A\vx = \vb$. Show that $\vu + \vp$ is a solution to $A\vx = \vb$.
  ({\it Hint:} Use Exercise 1.4.13.)
\end{enumerate}
\end{problem}

\newpage

%--  PROBLEM 5  ---------------------------------------------------
\begin{problem}[SA 1.4.15a]
Prove or give a counterexample: If $A$ is an $m \times n$
  matrix and $\vx \in \R^n$ is a vector satisfying $A\vx = \vzero$,
  then either every entry of $A$ is $0$ or $\vx = \vzero$.
\end{problem}

\end{document}
