\documentclass[fleqn,11pt]{paper}

%% This is the Homework LaTeX template.  Use this file to fill in your solutions. 
%%
%% Notes: 
%%    1. If possibly, try to write your answers inside a \begin{solution}...\end{solution}
%%    environment.
%%
%%    2. If you will use acronyms, please define them in the macros.tex file.
%%
%%    3. If you enter your answers into this Homework*.tex source file, you
%%    will have to compile it into a pdf document.  There are a number of ways to
%%    do that. Probably the easiest is to use the website called ShareLaTeX.com.
%%    Alternatively,
%%
%%       Mac OS X users: you might try MacTeX. 
%%       Linux users: most come with TeX; otherwise do a full install of TeXLive.
%%       Windows users: ...proTeXt maybe? (or switch to a better operating system) 
%%
%%       There is a Makefile in this directory, so (on Linux) you could just 
%%       enter `make` to compile all the Homework*.tex files at once (assuming
%%       you have everything set up properly).
%%
%%    4. Please don't hesitate to inform the professor if you have trouble, or open
%%       a ``New issue'' on GitHub or post to Piazza or ask in lecture.
%%
%%    5. Update the title and date if necessary.
         \title{Math 317: Homework 1}
         \date{Due: 22 January 2016}


%%%%%%%%%%%%%%%%%%%%%%%%%%%%%%%%%%%%%%%%
% Basic packages
%%%%%%%%%%%%%%%%%%%%%%%%%%%%%%%%%%%%%%%%
\usepackage[letterpaper,top=3cm,bottom=3cm,left=3cm,right=3cm]{geometry}
\usepackage{tikz-cd}
\usepackage{scalefnt}
\usepackage{amsmath,amsthm,amssymb}
\usepackage{mathtools}
\usepackage{etoolbox}
\usepackage{fancyhdr}
\usepackage{xcolor}
\usepackage[colorlinks=true,urlcolor=blue,linkcolor=blue,citecolor=blue]{hyperref}
\usepackage{xspace}
\usepackage{comment}
\usepackage{url} % for url in bib entries
\usepackage{mathrsfs}

\theoremstyle{remark}
\newtheorem{theorem}{Theorem}
\newtheorem*{prop}{Proposition}
\newtheorem{problem}{Problem}
\newtheorem*{prob}{Problem}
\newtheorem*{solution}{{\bf Solution}}
\newtheorem*{hint}{{\it Hint}}
\newtheorem*{ex}{Exercise}


%%%%%%%%%%%%%%%%%%%%%%%%%%%%%%%%%%%%%%%%%%%%%%%%%%
%% Surround the problem and solution with 
%% \begin{ProbBox}  and   \end{ProbBox}
%% to prevent pagebreaks.
\newenvironment{ProbBox}{\noindent\begin{minipage}{\linewidth}}{\end{minipage}}

%%%%%%%%%%%%%%%%%%%%%%%%
% Fancy page style     %
%%%%%%%%%%%%%%%%%%%%%%%%
\pagestyle{fancy}
\newcommand{\metadata}[2]{
  \lhead{}
  \chead{}
  \rhead{\bfseries Math 317: Linear Algebra}
  \lfoot{#1}
  \cfoot{#2}
  \rfoot{\thepage}
}
\renewcommand{\headrulewidth}{0.4pt}
\renewcommand{\footrulewidth}{0.4pt}


%%%%%%%%%%%%%%%%%%%%%%%%%%%%%%%%%%
% Customize list enviroonments   %
%%%%%%%%%%%%%%%%%%%%%%%%%%%%%%%%%%
% package to customize three basic list environments: enumerate, itemize and description.
%% \usepackage{enumitem}
%% \setitemize{noitemsep, topsep=0pt, leftmargin=*}
%% \setenumerate{noitemsep, topsep=0pt, leftmargin=*}
%% \setdescription{noitemsep, topsep=0pt, leftmargin=*}

\usepackage{enumerate}

%%%%%%%%%%%%%%%%%%%%%%%%%%%%
%% Space between problems  %
%%%%%%%%%%%%%%%%%%%%%%%%%%%%
\newrobustcmd*{\probskip}{\vskip1cm}

%%    Try to use standard notation, as used in class and in the textbook.
%%    For the most basic symbols, you may wish to use LaTeX macros to keep 
%%    the conventions you use consistent and easy to remember.

%%%%%%%%%%%%%%%%%%%%%%%%%%
%%    Math shortcuts     %
%%%%%%%%%%%%%%%%%%%%%%%%%%
\newcommand\join{\ensuremath{\vee}}
\newcommand\meet{\ensuremath{\wedge}}
\newcommand\R{\fld{R}}
\newcommand\proj{\ensuremath{\operatorname{proj}}}
\newcommand\End{\ensuremath{\operatorname{End}}}
\newcommand\Aut{\ensuremath{\operatorname{Aut}}}
\newcommand\Hom{\ensuremath{\operatorname{Hom}}}
\newcommand{\Aff}{\ensuremath{\operatorname{Aff}}}
\newcommand{\ann}[1]{\ensuremath{\operatorname{ann}(#1)}}
\newcommand{\id}{\ensuremath{\operatorname{id}}}
\newcommand{\nulity}[1]{\ensuremath{\operatorname{null}(#1)}}
\renewcommand{\ker}[1]{\ensuremath{\operatorname{ker}(#1)}}
\renewcommand{\dim}[1]{\ensuremath{\operatorname{dim}(#1)}}
\newcommand\im[1]{\ensuremath{\operatorname{im}(#1)}}
\newcommand{\rank}[1]{\ensuremath{\operatorname{rank}(#1)}}
\newcommand{\trace}[1]{\ensuremath{\operatorname{trace}(#1)}}
\renewcommand{\phi}{\ensuremath{\varphi}}

\renewcommand{\vec}[1]{\mathbf{#1}}
         \newcommand\alg[1]{\ensuremath{\mathbf{#1}}}
         \newcommand{\<}{\ensuremath{\langle}}
         \renewcommand{\>}{\ensuremath{\rangle}}
         \newcommand\fld[1]{\ensuremath{\mathbb{#1}}}

%%       To make a boldface vector, use backslash v in front of the 
%        letter and add a new command for that letter here.
         \newcommand\va{\vec{a}}
         \newcommand\vu{\vec{u}}
         \newcommand\vv{\vec{v}}
         \newcommand\vw{\vec{w}}
         \newcommand\vx{\vec{x}}
         \newcommand\vy{\vec{y}}
         \newcommand\vz{\vec{z}}
         \newcommand\vzero{\vec{0}}

         \newcommand\sP{\ensuremath{\mathscr P}}
         \newcommand\Span{\ensuremath{\operatorname{Span}}}

\begin{document}

%%    INSERT YOUR NAME HERE!!!
         \metadata{Name:}{HW 1 (due 2016/01/22)}
         \author{NAME:}
%%       Example:
%%         \metadata{Name: William DeMeo}{HW 1 (due 2015/09/04)}
%%         \author{NAME: William DeMeo}
%%

\maketitle

%%  PAGE 1  %%%%%%%%%%%%%%%%%%%%%%%%%%%%%%%%%%%%%%%%%%%%%%%%%%%%%%%
\noindent {\bf General Hints:}
When you are asked to ``show'' or ``prove'' something, you should make it a
point to write down clearly the information you are given and what it is you are
to show. One word of warning regarding the second part of Problem  1.1.22: 
To say that $\vec{v}$ is a linear combination of $\vec{v}_1, \dots, \vec{v}_k$
is to say that 
$\vec{v} = c_1\vec{v}_1 + c_2\vec{v}_2 +\cdots + c_k\vec{v}_k$ for some scalars 
$c_1,\dots, c_k$. These scalars
will surely be different when you express a different vector $\vec{w}$ 
as a linear combination of $\vec{v}_1, \dots, \vec{v}_k$, so be sure you give
the scalars for $\vec{w}$ different names. \\
\\
(``SA 1.1.21'' means exercise 21 in Section 1.1 of textbook by 
{\bf S}hifrin and {\bf A}dams.)\\[4pt]

%--  PROBLEM 0  ---------------------------------------------------
\noindent {\it Problem} 0 (SA 1.1.28). Carefully prove the following properties
of vector arithmetic. Justify all steps. 
Optional: Give the geometric interpretation of each property in case 
$n=2$ or $n=3$.  \\
You are required to turn proofs for e. f. g. (but you are encouraged to try 
them all).
\begin{enumerate}[a.]
\item 
For all $\vx, \vy \in \R^n$ , $\vx + \vy = \vy + \vx$.
\item 
For all $\vx, \vy, \vz \in \R^n$, $(\vx + \vy) + \vz = \vx + (\vy + \vz)$.
\item 
For all $\vx\in \R^n$, $\vzero + \vx = \vx$ for all $\vx \in \R^n$.
\item 
For each $\vx\in \R^n$, there is a vector $-\vx$ so that $\vx + (-\vx) = \vzero$.
\item 
For all $c, d \in \R^n$ and $\vx \in \R^n$, $c(d\vx) = (cd)\vx$.
\item 
For all $c \in \R$ and $\vx, \vy \in \R^n$ , $c(\vx + \vy) = c\vx + c\vy$.
\item 
For all $c, d \in \R$ and $\vx \in \R^n$, $(c + d)\vx = c\vx + d\vx$.
\item 
For all $\vx \in \R^n$, $1\vx = \vx$.
\end{enumerate}
(You must either type your homework using \LaTeX, or write them by hand on this prinout.
If you will use this printout to complete your homework, write your solutions to 
Problem~0 on the next page.) 



\newpage
%%  PAGE 2  %%%%%%%%%%%%%%%%%%%%%%%%%%%%%%%%%%%%%%%%%%%%%%%%%%%%%%%

\begin{enumerate}
\item[e.] {\bf Claim:}
For all $c, d \in \R^n$ and $\vx \in \R^n$, $c(d\vx) = (cd)\vx$.
\begin{proof}
  ~
\vskip5cm
\end{proof}
\item[f.] 
For all $c \in \R$ and $\vx, \vy \in \R^n$ , $c(\vx + \vy) = c\vx + c\vy$.
\begin{proof}
  ~
\vskip5cm
\end{proof}
\item[g.] 
For all $c, d \in \R$ and $\vx \in \R^n$, $(c + d)\vx = c\vx + d\vx$.
\begin{proof}
  ~
\vskip5cm
\end{proof}

\end{enumerate}



\newpage
%%  PAGE 3  %%%%%%%%%%%%%%%%%%%%%%%%%%%%%%%%%%%%%%%%%%%%%%%%%%%%%%%

%--  PROBLEM 1  ---------------------------------------------------
\begin{problem}[SA 1.1.21]
Suppose $\vv, \vw \in \R^n$ and $c$ is a scalar. Prove that 
$\Span (\vv + c\vw,\vw) = \Span (\vv,\vw)$. (See the blue box on p.~12 of the textbook.)
\end{problem}
\begin{proof}
%%%%%
%%%%% ENTER YOUR ANSWER BELOW:
%%%%% (Remove ~\vfill; replace it with your solution.)
%%%%%
~\vfill
\end{proof}

\newpage
%%  PAGE 4  %%%%%%%%%%%%%%%%%%%%%%%%%%%%%%%%%%%%%%%%%%%%%%%%%%%%%%%

%--  PROBLEM 2  ---------------------------------------------------
\begin{problem}[SA 1.1.22]
Suppose the vectors $\vec{v}$ and $\vec{w}$ are both linear combinations of 
$\vec{v}_1, \dots, \vec{v}_k$.
\begin{enumerate}[a.]
\item 
Prove for any scalar $c$ that $c\vec{v}$ is a linear combination of 
$\vec{v}_1,\dots, \vec{v}_k$.
\item
Prove that $\vec{v} + \vec{w}$ is a linear combination of 
$\vec{v}_1,\dots,\vec{v}_k$.
\end{enumerate}
\end{problem}
\begin{proof}
%%%%%
%%%%% ENTER YOUR ANSWER BELOW:
%%%%% (Remove ~\vfill; replace it with your solution.)
%%%%%
~\vfill
\end{proof}

\newpage
%%  PAGE 5  %%%%%%%%%%%%%%%%%%%%%%%%%%%%%%%%%%%%%%%%%%%%%%%%%%%%%%%

%--  PROBLEM 3  ---------------------------------------------------
\begin{problem}[SA 1.1.25]
Suppose $\vx, \vy \in \R^n$ are nonparallel vectors. (Recall def, p.3 of text.)
\begin{enumerate}[a.]
\item 
Prove that if $s\vx + t\vy = 0$, then $s = t = 0$. 
(Hint: Show neither $s \neq 0$ nor $t \neq0$ is possible.)
\item Prove that if $a\vx + b\vy = c\vx + d\vy$, then $a = c$ and $b = d$.
(Hint: Use part a.)
\end{enumerate}
\end{problem}
\begin{proof}
%%%%%
%%%%% ENTER YOUR ANSWER BELOW:
%%%%% (Remove ~\vfill; replace it with your solution.)
%%%%%
~\vfill
\end{proof}


\newpage
%%  PAGE 6  %%%%%%%%%%%%%%%%%%%%%%%%%%%%%%%%%%%%%%%%%%%%%%%%%%%%%%%

%--  PROBLEM 4  ---------------------------------------------------
\begin{problem}[SA 1.2.11]
Suppose $\vx, \vv_1, \dots, \vv_k \in \R^n$ and $\vx$ is orthogonal to each of
the vectors $\vv_1,\dots, \vv_k$. Show that $\vx$ is orthogonal to every linear
combination $c_1 \vv_1 + c_2 \vv_2 + \cdots + c_k \vv_k$.
\end{problem}
\begin{proof}
%%%%%
%%%%% ENTER YOUR ANSWER BELOW:
%%%%% (Remove ~\vfill; replace it with your solution.)
%%%%%
~\vfill
\end{proof}

\newpage
%%  PAGE 7  %%%%%%%%%%%%%%%%%%%%%%%%%%%%%%%%%%%%%%%%%%%%%%%%%%%%%%%

%--  PROBLEM 5  ---------------------------------------------------
\begin{problem}[SA 1.2.18]
Prove the triangle inequality: For all 
$\vx, \vy \in \R^n$, $\|\vx + \vy\| \leq \|\vx\| + \|\vy\|$. \\
(Hint: Use the dot product to calculate $\|\vx + \vy\|^2$.)
\end{problem}
\begin{proof}
%%%%%
%%%%% ENTER YOUR ANSWER BELOW:
%%%%% (Remove ~\vfill; replace it with your solution.)
%%%%%
~\vfill
\end{proof}

\newpage
%%  PAGE 8  %%%%%%%%%%%%%%%%%%%%%%%%%%%%%%%%%%%%%%%%%%%%%%%%%%%%%%%

%--  PROBLEM 6  ---------------------------------------------------
\begin{problem}[SA 1.3.12]
Suppose $\va \neq \vzero$ and $\sP \subset \R^3$ is the plane through the origin
with normal vector $\va$. 
Suppose $\sP$ is spanned by $\vu$ and $\vv$.
\begin{enumerate}[a.]
\item 
Suppose $\vu \cdot \vv = 0$. Show that for every $\vx \in \sP$, we have
$\vx = \proj_\vu \vx + \proj_\vv \vx$.
\item
Suppose $\vu \cdot \vv = 0$. Show that for every $\vx \in \R^3$, we have
$\vx = \proj_\va \vx + \proj_\vu \vx + \proj_\vv \vx$.\\
(Hint: Apply part a. to the vector $\vx - \proj_\va \vx$.)
\item
Give an example to show the result of part a is false when $\vu$ and $\vv$ are
not orthogonal. 
\end{enumerate}
\end{problem}
\begin{proof}
%%%%%
%%%%% ENTER YOUR ANSWER BELOW:
%%%%% (Remove ~\vfill; replace it with your solution.)
%%%%%
~\vfill
\end{proof}

%% \bibliographystyle{plain}
%% \bibliography{refs}

\end{document}
