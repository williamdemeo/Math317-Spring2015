%%This is the LaTeX file for the Math 317 Homework.  

%% You may use this file to fill in your solutions if you know how 
%% to compile a LaTeX file to turn it into a pdf document.  If you 
%% don't know how to do this but want to learn, there are plenty of
%% resources for learning about LaTeX on the net.  You may also ask
%% your professor for help.
%% 
%% For more information about using this file to complete the homework,
%% please see the NOTES section below of the comments below.

%% \documentclass[fleqn,11pt]{paper}
\documentclass[11pt]{paper}


\usepackage[
letterpaper,
top    = 3cm,
bottom = 3cm,
left   = 3.00cm,
right  = 3.00cm]{geometry}

\usepackage{tikz-cd}
\usepackage{amsthm}
\usepackage{scalefnt}

%%%%%%%%%%%%%%%%%%%%%%%%%%%%%%%%%%%%%%%%
% Basic packages
%%%%%%%%%%%%%%%%%%%%%%%%%%%%%%%%%%%%%%%%
\usepackage{amsmath,amsthm,amssymb}
\usepackage{mathtools}
\usepackage{etoolbox}
\usepackage[mathcal]{euscript}
\usepackage{fancyhdr}
 \usepackage{xcolor}
\usepackage[colorlinks=true,urlcolor=blue,linkcolor=blue,citecolor=blue]{hyperref}
\usepackage{xspace}
\usepackage{comment}
\usepackage{url} % for url in bib entries
\usepackage{mathrsfs}

\theoremstyle{remark}
\newtheorem{theorem}{Theorem}
\newtheorem*{prop}{Proposition}
\newtheorem{problem}{Problem}
\newtheorem*{prob}{Problem}
\newtheorem*{definition}{Definition}
\newtheorem*{exercise}{Exercise}
\newtheorem*{solution}{{\bf Solution}}
\newtheorem*{hint}{{\it Hint}}
\newtheorem*{ex}{Exercise}


%%%%%%%%%%%%%%%%%%%%%%%%%%%%%%%%%%%%%%%%%%%%%%%%%%%%%%%%%%%%%%%%%%%%%%%%%%%%%%%
%% If you want to prevent pagebreaks, surround the problem and/or solution with 
%% \begin{ProbBox}  and   \end{ProbBox}
\newenvironment{ProbBox}{\noindent\begin{minipage}{\linewidth}}{\end{minipage}}

%%%%%%%%%%%%%%%%%%%%%%%%
% Fancy page style     %
%%%%%%%%%%%%%%%%%%%%%%%%
\pagestyle{fancy}
\newcommand{\metadata}[2]{
  \lhead{}
  \chead{}
  \rhead{\bfseries Math 317: Linear Algebra}
  \lfoot{#1}
  \cfoot{#2}
  \rfoot{\thepage}
}
\renewcommand{\headrulewidth}{0.4pt}
\renewcommand{\footrulewidth}{0.4pt}
\newrobustcmd*{\vocab}[1]{\emph{#1}}
\newrobustcmd*{\latin}[1]{\textit{#1}}

\usepackage{enumerate}

%%%%%%%%%%%%%%%%%%%%%%%%%%%%
%% Space between problems  %
%%%%%%%%%%%%%%%%%%%%%%%%%%%%
\newrobustcmd*{\probskip}{\vskip1cm}


%%%%%%%%%%%%%%%%%%%%%%%%%%
%%    Math shortcuts     %
%%%%%%%%%%%%%%%%%%%%%%%%%%
\newcommand\join{\ensuremath{\vee}}
\newcommand\meet{\ensuremath{\wedge}}
\newcommand\R{\fld{R}}
\newcommand\proj{\ensuremath{\operatorname{proj}}}
\newcommand\End{\ensuremath{\operatorname{End}}}
\newcommand\Aut{\ensuremath{\operatorname{Aut}}}
\newcommand\Hom{\ensuremath{\operatorname{Hom}}}
\newcommand{\Aff}{\ensuremath{\operatorname{Aff}}}
\newcommand{\ann}[1]{\ensuremath{\operatorname{ann}(#1)}}
\newcommand{\id}{\ensuremath{\operatorname{id}}}
\newcommand{\nulity}{\ensuremath{\operatorname{null}}}
\renewcommand{\ker}{\ensuremath{\operatorname{ker}}}
\renewcommand{\dim}{\ensuremath{\operatorname{dim}}}
\newcommand\im{\ensuremath{\operatorname{im}}}
\newcommand{\rank}{\ensuremath{\operatorname{rank}}}
\newcommand{\trace}{\ensuremath{\operatorname{trace}}}
\newcommand{\vR}{\ensuremath{\operatorname{R}}}
\newcommand{\vN}{\ensuremath{\operatorname{N}}}
\newcommand{\vC}{\ensuremath{\operatorname{C}}}
\renewcommand{\phi}{\ensuremath{\varphi}}
\renewcommand{\vec}[1]{\mathbf{#1}}

%%%% NOTES %%%%%%%%%%%%%%%%%%%%%%%%%%%%%%%%%%%%%%%%%%%%%%%%%%%%%%%%% 
         \newcommand\alg[1]{\ensuremath{\mathbf{#1}}}
         \newcommand{\<}{\ensuremath{\langle}}
         \renewcommand{\>}{\ensuremath{\rangle}}
         \newcommand\fld[1]{\ensuremath{\mathbb{#1}}}
%%       To make a boldface vector, use backslash v in front of the 
%%       letter and add a new command for that letter here.
         \newcommand\va{\vec{a}}
         \newcommand\vb{\vec{b}}
         \newcommand\vn{\vec{n}}
         \newcommand\vp{\vec{p}}
         \newcommand\vu{\vec{u}}
         \newcommand\vv{\vec{v}}
         \newcommand\vw{\vec{w}}
         \newcommand\vx{\vec{x}}
         \newcommand\vy{\vec{y}}
         \newcommand\vz{\vec{z}}
         \newcommand\vzero{\vec{0}}
         \newcommand\sV{\ensuremath{\mathcal V}}
         \newcommand\sW{\ensuremath{\mathcal W}}
         \newcommand\sP{\ensuremath{\mathcal P}}
         \newcommand\Span{\ensuremath{\operatorname{Span}}}

         \author         {NAME:                     }
         %% Put your name here ^^^^^^^^^^^^^^^^^^^^^, like this:
         %%     \author{NAME: William DeMeo}
         %%
         %% Update the title and date if necessary.
         \title{Math 317: Homework 9}
         \date{Due: 31 March 2016}
         \metadata{HW 9}{Due March 31}
%%%%%%%%%%%%%%%%%%%%%%%%%%%%%%%%%%%%%%%%%%%%%%%%%%%%%%%%%%%%%%%%%%% 
%%%%%%%%%%%%%%%%%%%%%%%%%%%%%%%%%%%%%%%%%%%%%%%%%%%%%%%%%%%%%%%%%%% 


\begin{document}

\maketitle

\section*{Section 4.1}
\noindent Recall the following
\begin{definition}
If $V \leq \R^m$ is a subspace, and $\vb \in \R^m$, then we define the 
\emph{projection of $\vb$ onto $V$} to be the unique vector $\vp \in V$ 
with the property that $\vb - \vp \in V^\bot$, and we write $\vp = \proj_V\vb$
in this case.
\end{definition}

\medskip
\noindent The following exercise is recommended; it will not be graded.

%--  Exercise  ---------------------------------------------------
\begin{exercise}[SA 4.1.13]
Use the definition of projection given above to show that for any subspace 
$V \leq \R^m$, the function $\proj_V : \R^m \to \R^m$ is a linear
transformation.  %% That is, show that
%% a. proj V (x + y) = proj V x + proj V y for all vectors x and y;
%% b. proj V (cx) = c proj V x for all vectors x and scalars c.
\end{exercise}
%% \medskip
%% \begin{solution}
%% \end{solution}

\newpage
%--  PROBLEM 4  ---------------------------------------------------
\begin{problem}[SA 4.1.14]
Prove \emph{directly from the definition above} that if 
we let $P$ denote the matrix projection onto $V$---that is, 
$P\vb = \proj_V\vb$---then $P = P^2$ and
$P = P^\top$. [{\it Hints:} For the latter, show that 
$P\vx\cdot \vy = \vx \cdot P\vy$ 
for all $\vx, \vy$. It may be helpful
to write $\vx$ and $\vy$ as the sum of vectors in 
$V$ and $V^\bot$. Then use Exercise 2.5.24.]
\end{problem}
%% \medskip
%% \begin{solution}
%% \end{solution}


\newpage

\noindent The next exercise is recommended but will not be graded.
\medskip

\begin{exercise}[SA 4.1.15]
Prove the converse of the fact in the last exercise. That is,
if $A$ is a matrix and $A^2 = A$ and $A^\bot = A$, then $A$ is a projection 
matrix. [{\it Hints:} 
First decide onto which subspace $V$ it should be projecting. 
Then Show that for any $\vb$, the vector 
$\vp = A\vb$ satisfies the definition above of the projection of $\vb$ on the 
subspace $V$.]
\end{exercise}
%% \medskip
%% \begin{solution}
%% \end{solution}

\newpage

\noindent As we have seen in lecture, if $V\leq \R^m$ is a subspace 
and $\vb\in \R^m$, then 
\[
\vb = \proj_V \vb + \proj_{V^\bot} \vb.
\]  
Therefore, if we know $\vb$ and $\proj_{V^\bot} \vb$, then we easily 
compute $\proj_V \vb$ as follows: $\proj_V \vb =\vb - \proj_{V^\bot} \vb$.  
It's sometimes the case that $\proj_{V^\bot}\vb$ is very easy to compute,
as in this first exercise below, where the subspace $V$ is a plane and
$V^{\bot}$ is equal to the span of a ``normal'' vector (i.e., a vector 
orthogonal to the plane $V$).  For example, in Part (b), a 
normal vector to $V$ is $\vn = (1,1,1,0)$, so the projection of $\vb$ 
onto $V^{\bot}$ is $\frac{\vb\cdot \vn}{\|\vn\|^2} \vn = (1,1,1,0)$.
Therefore, $\proj_V \vb  = (-1,0,1,3)$.
%% \[\proj_V \vb =\vb - \proj_{V^\bot} \vb =   (0, 1, 2, 3) - (1,1,1,0) = (-1,0,1,3).\]

%--  Problem 2  ---------------------------------------------------
\begin{problem}[SA 4.1.1]
Find the projection of the given
vector $\vb \in \R^m$ onto the given hyperplane $V \leq \R^m$
by first finding the projection onto $V^\bot$, as suggested above.
\begin{enumerate}[(a)]
\item $V = \{x_1 + x_2 + x_3 = 0\} \leq \R^3$, $\vb = (2, 1, 1)$.
\item $V = \{x_1 + x_2 + x_3 = 0\} \leq \R^4$, $\vb = (0, 1, 2, 3)$.
\item $V = \{x_1 - x_2 + x_3 + 2x_4 = 0\} \leq \R^4$, $\vb = (1, 1, 1, 1)$.
\end{enumerate}
\end{problem}

\newpage

%--  PROBLEM 3  ---------------------------------------------------
\begin{problem}[SA 4.1.2]
Use the formula $P_V = A(A^\top A)^{-1} A^\top$ 
for the projection matrix to check that 
$P_V = P_V^\top$ 
and 
$P_V^2 = P_V$.  Show that $I - P_V$ has the same properties, and explain why.
\end{problem}
%% \medskip
%% \begin{solution}
%% \end{solution}

\newpage
%\probskip

%--  PROBLEM 4  ---------------------------------------------------
\begin{problem}[SA 4.1.4]
Let $V = \Span \{(1, 0, 1), (0, 1, -2)\} \leq R^3$. 
Construct the matrix $P_V$ representing $\proj_V$ in two ways:
\begin{enumerate}[(a)]
\item 
by finding $P_{V^\bot}$;
\item 
by using the formula $P_V = A(A^\top A)^{-1} A^\top$.
\end{enumerate}
\end{problem}
%% \medskip
%% \begin{solution}
%% \end{solution}

\newpage
%\probskip


%--  PROBLEM 5  ---------------------------------------------------
\begin{problem}[SA 4.1.7]
Find the least squares solution of
\begin{align*}
x_1 +x_2 &= 1\\
x_1 - 3x_2 &= 4\\
2x_1 +x_2 &= 3.
\end{align*}
Use your answer to find the point on the plane spanned by $(1, 1, 2)$ and 
$(1, -3, 1)$ that is closest to $(1, 4, 3)$. 
\end{problem}
%% \medskip
%% \begin{solution}
%% \end{solution}


\newpage
%--  PROBLEM 6  ---------------------------------------------------
\begin{problem}[SA 4.2.3]
Let $V = \Span \{(2, 1, 0, -2), (3, 3, 1, 0)\}\leq \R^4$.
\begin{enumerate}[(a)]
\item 
Find an orthogonal basis for $V$.
\item Use your answer to part (a) to find the projection of 
  $\vb = (0, 4, -4, -7)$ onto $V$.
\item Use your answer to part (a) to find the projection matrix $P_V$.
\end{enumerate}
\end{problem}
%% \medskip
%% \begin{solution}
%% \end{solution}

\newpage

%--  PROBLEM 7  ---------------------------------------------------
\begin{problem}
Let $\mathcal{C}^0([a, b])$ denote the vector space of continuous real-valued
functions defined on $[a,b]$. Recall that an inner product on 
$\mathcal{C}^0([a, b])$ can be defined as follows: 
for $f, g \in \mathcal{C}^0([a, b])$,
\[
\<f, g\> =
\int^b_a f(t)g(t)\, dt.
\]
Using this inner product, find an orthogonal basis for the 
subspace $\mathcal{P}_1 \leq C^0([0, 1])$,
and use your answer to find the projection
of $f(t) = t^2 + t - 1$ onto $\mathcal{P}_1$.
\end{problem}
%% \medskip
%% \begin{solution}
%% \end{solution}


\end{document}
