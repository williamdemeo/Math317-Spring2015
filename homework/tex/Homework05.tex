%%This is the LaTeX file for the Math 317 Homework.  

%% You may use this file to fill in your solutions if you know how 
%% to compile a LaTeX file to turn it into a pdf document.  If you 
%% don't know how to do this but want to learn, there are plenty of
%% resources for learning about LaTeX on the net.  You may also ask
%% your professor for help.
%% 
%% For more information about using this file to complete the homework,
%% please see the NOTES section below of the comments below.

\documentclass[fleqn,11pt]{paper}


\usepackage[
letterpaper,
top    = 3cm,
bottom = 3cm,
left   = 3.00cm,
right  = 3.00cm]{geometry}

\usepackage{tikz-cd}
\usepackage{amsthm}
\usepackage{scalefnt}
\usepackage{multicol}

%%%%%%%%%%%%%%%%%%%%%%%%%%%%%%%%%%%%%%%%
% Basic packages
%%%%%%%%%%%%%%%%%%%%%%%%%%%%%%%%%%%%%%%%
\usepackage{amsmath,amsthm,amssymb}
\usepackage{mathtools}
\usepackage{etoolbox}
\usepackage{fancyhdr}
 \usepackage{xcolor}
\usepackage[colorlinks=true,urlcolor=blue,linkcolor=blue,citecolor=blue]{hyperref}
\usepackage{xspace}
\usepackage{comment}
\usepackage{url} % for url in bib entries
\usepackage{mathrsfs}

\theoremstyle{remark}
\newtheorem{theorem}{Theorem}
\newtheorem*{prop}{Proposition}
\newtheorem{problem}{Problem}
\newtheorem*{prob}{Problem}
\newtheorem*{solution}{{\bf Solution}}
\newtheorem*{hint}{{\it Hint}}
\newtheorem*{ex}{Exercise}


%%%%%%%%%%%%%%%%%%%%%%%%%%%%%%%%%%%%%%%%%%%%%%%%%%
%% Surround the problem and solution with 
%% \begin{ProbBox}  and   \end{ProbBox}
%% to prevent pagebreaks.
\newenvironment{ProbBox}{\noindent\begin{minipage}{\linewidth}}{\end{minipage}}

%%%%%%%%%%%%%%%%
% Acronyms     %
%%%%%%%%%%%%%%%%
\usepackage[acronym, shortcuts]{glossaries}

%% HERE IS HOW YOU DEFINE ACRONYMS:
\newacronym{FTA}{FTA}{Fundamental Theorem of Algebra}
\newacronym{CRT}{CRT}{Chinese Remainder Theorem}

% Make \ac robust.
\robustify{\ac}

%%%%%%%%%%%%%%%%%%%%%%%%
% Fancy page style     %
%%%%%%%%%%%%%%%%%%%%%%%%
\pagestyle{fancy}
\newcommand{\metadata}[2]{
  \lhead{}
  \chead{}
  \rhead{\bfseries Math 317: Linear Algebra}
  \lfoot{#1}
  \cfoot{#2}
  \rfoot{\thepage}
}
\renewcommand{\headrulewidth}{0.4pt}
\renewcommand{\footrulewidth}{0.4pt}


\newrobustcmd*{\vocab}[1]{\emph{#1}}
\newrobustcmd*{\latin}[1]{\textit{#1}}

%%%%%%%%%%%%%%%%%%%%%%%%%%%%%%%%%%
% Customize list enviroonments   %
%%%%%%%%%%%%%%%%%%%%%%%%%%%%%%%%%%
% package to customize three basic list environments: enumerate, itemize and description.
%% \usepackage{enumitem}
%% \setitemize{noitemsep, topsep=0pt, leftmargin=*}
%% \setenumerate{noitemsep, topsep=0pt, leftmargin=*}
%% \setdescription{noitemsep, topsep=0pt, leftmargin=*}

\usepackage{enumerate}

%%%%%%%%%%%%%%%%%%%%%%%%%%%%
%% Space between problems  %
%%%%%%%%%%%%%%%%%%%%%%%%%%%%
\newrobustcmd*{\probskip}{\vskip1cm}


%%%%%%%%%%%%%%%%%%%%%%%%%%
%%    Math shortcuts     %
%%%%%%%%%%%%%%%%%%%%%%%%%%
%\newcommand\iff{\ensuremath{\Longleftrightarrow}}
\newcommand\join{\ensuremath{\vee}}
\newcommand\meet{\ensuremath{\wedge}}
\newcommand\R{\fld{R}}
\newcommand\proj{\ensuremath{\operatorname{proj}}}
\newcommand\End{\ensuremath{\operatorname{End}}}
\newcommand\Aut{\ensuremath{\operatorname{Aut}}}
\newcommand\Hom{\ensuremath{\operatorname{Hom}}}
\newcommand{\Aff}{\ensuremath{\operatorname{Aff}}}
\newcommand{\ann}[1]{\ensuremath{\operatorname{ann}(#1)}}
\newcommand{\id}{\ensuremath{\operatorname{id}}}
\newcommand{\nulity}[1]{\ensuremath{\operatorname{null}(#1)}}
\renewcommand{\ker}[1]{\ensuremath{\operatorname{ker}(#1)}}
\renewcommand{\dim}[1]{\ensuremath{\operatorname{dim}(#1)}}
\newcommand\im[1]{\ensuremath{\operatorname{im}(#1)}}
\newcommand{\rank}[1]{\ensuremath{\operatorname{rank}(#1)}}
\newcommand{\trace}[1]{\ensuremath{\operatorname{trace}(#1)}}
\renewcommand{\phi}{\ensuremath{\varphi}}

\renewcommand{\vec}[1]{\mathbf{#1}}



%%%% NOTES %%%%%%%%%%%%%%%%%%%%%%%%%%%%%%%%%%%%%%%%%%%%%%%%%%%%%%%%% 

%%
%%    1. If possible, try to write your answers inside a 
%%
%%               \begin{solution}...\end{solution}
%%
%%       environment.
%%
%%    2. If you will use acronyms, please define them in the macros.tex file.
%%
%%    3. If you enter your answers into this Homework*.tex source file, you
%%       will have to compile it into a pdf document.  There are a number of ways to
%%       do that. Probably the easiest is to use the website called ShareLaTeX.com.
%%       Alternatively,
%%
%%           Mac OS X users: you might try MacTeX. 
%%           Linux users: most come with TeX; otherwise do a full install of TeXLive.
%%           Windows users: try proTeXt maybe? (or switch to a real operating system!) 
%%
%%       There is a Makefile in this directory, so on Linux you could just 
%%       enter `make` to compile all the Homework*.tex files at once.
%%
%%    4. Please don't hesitate to inform the professor if you have trouble, or open
%%       a ``New issue'' on GitHub or post to Piazza or ask in lecture.  Otherwise,
%%       send an email to williamdemeo at gmail.
%%
%%    5. Try to use standard notation, as used in class and in the textbook.
%%       For the most basic symbols, you may wish to use LaTeX macros to keep 
%%       the conventions you use consistent and easy to remember.
%%       For example, to denote an algebra,
         \newcommand\alg[1]{\ensuremath{\mathbf{#1}}}
         \newcommand{\<}{\ensuremath{\langle}}
         \renewcommand{\>}{\ensuremath{\rangle}}
%%       So, an algebra in LaTeX is typed as $\alg{A} = \<A, F\>$.
%%       Similarly, for a field, let's use:
         \newcommand\fld[1]{\ensuremath{\mathbb{#1}}}
%%       So, a field in LaTeX is typed as $\fld{F}$.
%%
%%       To make a boldface vector, use backslash v in front of the 
%%       letter and add a new command for that letter here or in 
%%       the macros.tex file:
         \newcommand\va{\vec{a}}
         \newcommand\vb{\vec{b}}
         \newcommand\vu{\vec{u}}
         \newcommand\vv{\vec{v}}
         \newcommand\vw{\vec{w}}
         \newcommand\vx{\vec{x}}
         \newcommand\vy{\vec{y}}
         \newcommand\vz{\vec{z}}
         \newcommand\vzero{\vec{0}}
         \newcommand\sP{\ensuremath{\mathscr P}}
         \newcommand\Span{\ensuremath{\operatorname{Span}}}
%%
%%    6. Insert your name here!!!
%%

         %% \metadata       {Name:                     }{HW 5 (due Oct 5)}
         %% Put your name here ^^^^^^^^^^^^^^^^^^^^^, like this:
         %% \metadata  {Name: William DeMeo}{HW 5 (due Oct 5)}  
         %%
         
         %% \author         {NAME:                     }
         %% Put your name here ^^^^^^^^^^^^^^^^^^^^^, like this:
         %%     \author{NAME: William DeMeo}
         %%
%%
%%
%%    7. Update the title and date if necessary.
         \title{Math 317: Homework 5}
         \date{14 Feb 2016}
%%
%%%%%%%%%%%%%%%%%%%%%%%%%%%%%%%%%%%%%%%%%%%%%%%%%%%%%%%%%%%%%%%%%%% 


\begin{document}

\maketitle

%--  PROBLEM 1  ---------------------------------------------------
\begin{problem}[SA 2.4.1a]
Consider the matrix 
$A = \begin{bmatrix*}[r] 1 & 0&-1 \\ -2&3 &-1 \\3&-3 &0 \end{bmatrix*}$
from Exercise 1.4.3a.\\[6pt]
Find a product of elementary matrices $E = E_k \cdots E_2 E_1$
so that $EA$ is in echelon form. Use the matrix $E$ you've found to give
constraint equations for $A\vx = \vb$ to be consistent.
\end{problem}

\probskip

%--  PROBLEM 2  ---------------------------------------------------
\begin{problem}[SA 2.4.3a]
Find the $LU$ decomposition of the matrix $A$ of the previous exercise.
\end{problem}

\probskip

%% %--  PROBLEM 2  ---------------------------------------------------
%% \begin{problem}[SA 2.4.5b]
%% Find a left inverse of the matrix
%% $A = \begin{bmatrix*}[r] 1 & 2 \\ 1 &3 \\ 1&-1 \end{bmatrix*}$.
%% \end{problem}

%% \probskip

%--  PROBLEM 3  ---------------------------------------------------
\begin{problem}[SA 2.4.7]
Show that the inverse of every elementary matrix is again an elementary matrix
by giving a simple prescription for determining the inverse of each type of
elementary matrix. (See the proof of Theorem 4.1 of Chapter 1.)
\end{problem}

\probskip

%--  PROBLEM 4  ---------------------------------------------------
\begin{problem}[SA 2.4.8]
Prove or give a counterexample: Every invertible matrix can be written as a
product of elementary matrices.
\end{problem}

\probskip


%--  PROBLEM 5  ---------------------------------------------------
\begin{problem}[SA 2.5.1beg]
Let $A =\begin{bmatrix*}[r]1&2\\3&4\end{bmatrix*}$,
$B =\begin{bmatrix*}[r]2&1\\4&3\end{bmatrix*}$,
$C =\begin{bmatrix*}[r]1&2&1\\0&1&2\end{bmatrix*}$, and
$C =\begin{bmatrix*}[r]0&1\\1&0\\2&3\end{bmatrix*}$.
Calculate each of the following expressions or explain why it is not defined.
\begin{multicols}{3}
\begin{enumerate}
\item[b.] $2A - B^\top$
\item[e.] $A^\top C$
\item[g.] $C^\top A^\top$
\end{enumerate}
\end{multicols}
\end{problem}


\probskip

%--  PROBLEM 6  ---------------------------------------------------
\begin{problem}[SA 2.5.2abf]
Let $\va =\begin{bmatrix*}[r]1\\2\\1\end{bmatrix*}$
and $\vb =\begin{bmatrix*}[r]0\\3\\-1\end{bmatrix*}$.
Calculate the following matrices.
\begin{multicols}{3}
\begin{enumerate}
\item[a.] $\va\va^\top$
\item[b.] $\va^\top \va$
\item[f.] $\va^\top \vb$
\end{enumerate}
\end{multicols}
\end{problem}


\probskip

%--  PROBLEM 7  ---------------------------------------------------
\begin{problem}[SA 2.5.3b]
Let $\va =(4, 3)$. Find the standard matrix for the projection $\proj_{\va}$.
[{\it Hint:} see Example 3 in the textbook.]
%\begin{bmatrix*}[r]4\\3\end{bmatrix*}$.
\end{problem}

\probskip

%--  PROBLEM 8  ---------------------------------------------------
\begin{problem}[SA 2.5.5]
  Suppose $A$ and $B$ are symmetric.  Show that $AB$ is symmetric if and only if
  $AB = BA$.
\end{problem}

\probskip

%--  PROBLEM 9  ---------------------------------------------------
\begin{problem}[SA 2.5.8]
  Suppose $A$ is invertible.  Check that $(A^{-1})^{\top}A^\top = I$ and
  $A^\top (A^{-1})^\top = I$ and deduce that $A^\top$ is likewise invertible with
  inverse $(A^{-1})^\top$.
\end{problem}

\probskip
%--  PROBLEM 10  ---------------------------------------------------
\begin{problem}[SA 2.5.15]
Suppose $A$ is an $m \times n$ matrix and $\vx \in \R^n$ satisfies $(A^\top A)\vx$ = 0. 
Prove that $A\vx = \vzero$. (Hint: What is $\|A\vx\|$?)
\end{problem}
%% \bibliographystyle{plain}
%% \bibliography{refs}
\end{document}
