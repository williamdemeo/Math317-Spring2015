%%This is the LaTeX file for the Math 317 Homework.  

%% You may use this file to fill in your solutions if you know how 
%% to compile a LaTeX file to turn it into a pdf document.  If you 
%% don't know how to do this but want to learn, there are plenty of
%% resources for learning about LaTeX on the net.  You may also ask
%% your professor for help.
%% 
%% For more information about using this file to complete the homework,
%% please see the NOTES section below of the comments below.

\documentclass[fleqn,11pt]{paper}


\usepackage[
letterpaper,
top    = 3cm,
bottom = 3cm,
left   = 3.00cm,
right  = 3.00cm]{geometry}

\usepackage{tikz-cd}
\usepackage{amsthm}
\usepackage{scalefnt}

%%%%%%%%%%%%%%%%%%%%%%%%%%%%%%%%%%%%%%%%
% Basic packages
%%%%%%%%%%%%%%%%%%%%%%%%%%%%%%%%%%%%%%%%
\usepackage{amsmath,amsthm,amssymb}
\usepackage[mathcal]{euscript}
\usepackage{mathtools}
\usepackage{etoolbox}
\usepackage{fancyhdr}
 \usepackage{xcolor}
\usepackage[colorlinks=true,urlcolor=blue,linkcolor=blue,citecolor=blue]{hyperref}
\usepackage{xspace}
\usepackage{comment}
\usepackage{url} % for url in bib entries
\usepackage{mathrsfs}

\theoremstyle{remark}
\newtheorem{theorem}{Theorem}
\newtheorem*{prop}{Proposition}
\newtheorem{problem}{Problem}
\newtheorem*{prob}{Problem}
\newtheorem*{solution}{{\bf Solution}}
\newtheorem*{hint}{{\it Hint}}
\newtheorem*{ex}{Exercise}


%%%%%%%%%%%%%%%%%%%%%%%%%%%%%%%%%%%%%%%%%%%%%%%%%%
%% Surround the problem and solution with 
%% \begin{ProbBox}  and   \end{ProbBox}
%% to prevent pagebreaks.
\newenvironment{ProbBox}{\noindent\begin{minipage}{\linewidth}}{\end{minipage}}

%%%%%%%%%%%%%%%%
% Acronyms     %
%%%%%%%%%%%%%%%%
\usepackage[acronym, shortcuts]{glossaries}

%% HERE IS HOW YOU DEFINE ACRONYMS:
\newacronym{FTA}{FTA}{Fundamental Theorem of Algebra}
\newacronym{CRT}{CRT}{Chinese Remainder Theorem}

% Make \ac robust.
\robustify{\ac}

%%%%%%%%%%%%%%%%%%%%%%%%
% Fancy page style     %
%%%%%%%%%%%%%%%%%%%%%%%%
\pagestyle{fancy}
\newcommand{\metadata}[2]{
  \lhead{}
  \chead{}
  \rhead{\bfseries Math 317: Linear Algebra}
  \lfoot{#1}
  \cfoot{#2}
  \rfoot{\thepage}
}
\renewcommand{\headrulewidth}{0.4pt}
\renewcommand{\footrulewidth}{0.4pt}


\newrobustcmd*{\vocab}[1]{\emph{#1}}
\newrobustcmd*{\latin}[1]{\textit{#1}}

%%%%%%%%%%%%%%%%%%%%%%%%%%%%%%%%%%
% Customize list enviroonments   %
%%%%%%%%%%%%%%%%%%%%%%%%%%%%%%%%%%
% package to customize three basic list environments: enumerate, itemize and description.
%% \usepackage{enumitem}
%% \setitemize{noitemsep, topsep=0pt, leftmargin=*}
%% \setenumerate{noitemsep, topsep=0pt, leftmargin=*}
%% \setdescription{noitemsep, topsep=0pt, leftmargin=*}

\usepackage{enumerate}

%%%%%%%%%%%%%%%%%%%%%%%%%%%%
%% Space between problems  %
%%%%%%%%%%%%%%%%%%%%%%%%%%%%
\newrobustcmd*{\probskip}{\vskip1cm}


%%%%%%%%%%%%%%%%%%%%%%%%%%
%%    Math shortcuts     %
%%%%%%%%%%%%%%%%%%%%%%%%%%
%\newcommand\iff{\ensuremath{\Longleftrightarrow}}
\newcommand\join{\ensuremath{\vee}}
\newcommand\meet{\ensuremath{\wedge}}
\newcommand\R{\fld{R}}
\newcommand\proj{\ensuremath{\operatorname{proj}}}
\newcommand\End{\ensuremath{\operatorname{End}}}
\newcommand\Aut{\ensuremath{\operatorname{Aut}}}
\newcommand\Hom{\ensuremath{\operatorname{Hom}}}
\newcommand{\Aff}{\ensuremath{\operatorname{Aff}}}
\newcommand{\ann}[1]{\ensuremath{\operatorname{ann}(#1)}}
\newcommand{\id}{\ensuremath{\operatorname{id}}}
\newcommand{\nulity}[1]{\ensuremath{\operatorname{null}(#1)}}
%\renewcommand{\ker}[1]{\ensuremath{\operatorname{ker}(#1)}}
\renewcommand{\ker}{\ensuremath{\operatorname{ker}}}
%% \renewcommand{\dim}[1]{\ensuremath{\operatorname{dim}(#1)}}
\renewcommand{\dim}{\ensuremath{\operatorname{dim}}}
%\newcommand\im[1]{\ensuremath{\operatorname{im}(#1)}}
\newcommand\im{\ensuremath{\operatorname{im}}}
%% \newcommand{\rank}[1]{\ensuremath{\operatorname{rank}(#1)}}
\newcommand{\rank}{\ensuremath{\operatorname{rank}}}
\newcommand{\trace}[1]{\ensuremath{\operatorname{trace}(#1)}}
\renewcommand{\phi}{\ensuremath{\varphi}}

\renewcommand{\vec}[1]{\mathbf{#1}}



%%%% NOTES %%%%%%%%%%%%%%%%%%%%%%%%%%%%%%%%%%%%%%%%%%%%%%%%%%%%%%%%% 
%%
%%       Try to use standard notation, as used in class and in the textbook.
%%       For the most basic symbols, you may wish to use LaTeX macros to keep 
%%       the conventions you use consistent and easy to remember.
%%       For example, to denote an algebra,
         \newcommand\alg[1]{\ensuremath{\mathbf{#1}}}
         \newcommand{\<}{\ensuremath{\langle}}
         \renewcommand{\>}{\ensuremath{\rangle}}
%%       So, an algebra in LaTeX is typed as $\alg{A} = \<A, F\>$.
%%       Similarly, for a field, let's use:
         \newcommand\fld[1]{\ensuremath{\mathbb{#1}}}
%%       So, a field in LaTeX is typed as $\fld{F}$.
%%
%%       To make a boldface vector, use backslash v in front of the 
%%       letter and add a new command for that letter here or in 
%%       the macros.tex file:
         \newcommand\va{\vec{a}}
         \newcommand\vf{\vec{f}}
         \newcommand\vu{\vec{u}}
         \newcommand\vv{\vec{v}}
         \newcommand\vw{\vec{w}}
         \newcommand\vx{\vec{x}}
         \newcommand\vy{\vec{y}}
         \newcommand\vz{\vec{z}}
         \newcommand\vzero{\vec{0}}
         \newcommand\sC{\ensuremath{\mathcal C}}
         \newcommand\sB{\ensuremath{\mathcal B}}
         \newcommand\bE{\ensuremath{\mathbf E}}
         \newcommand\sV{\ensuremath{\mathcal V}}
         \newcommand\sW{\ensuremath{\mathcal W}}
         \newcommand\sP{\ensuremath{\mathcal P}}
         \newcommand\Span{\ensuremath{\operatorname{Span}}}
         \metadata       {HW 12}{due 2016/04/26}
%%
%%       Insert your name here!!!
%%
         \author         {NAME:                     }
         %% Put your name here ^^^^^^^^^^^^^^^^^^^^^, like this:
         %%     \author{NAME: William DeMeo}
         %%
%%
%%
%%    8. Update the title and date if necessary.
         \title{Math 317: Homework 12}
         \date{Due: 26 April 2015}
%%
%%%%%%%%%%%%%%%%%%%%%%%%%%%%%%%%%%%%%%%%%%%%%%%%%%%%%%%%%%%%%%%%%%% 

  %% \begin{bmatrix*}[r] 1 & 0 & -1 \\ -2 & 3 & -1 \\ 3 & -3 & 0\end{bmatrix*}.

\begin{document}

\maketitle


\section*{Section 6.2}
%--  PROBLEM 1  ---------------------------------------------------
\begin{problem}[cf.~SA 6.2.1fkp]
Recall, the matrices in Problem 6.1.1 of Homework 11 were,
%% \begin{enumerate}
%% \item[f.] $\begin{bmatrix*}[r]1&1\\-1&3\end{bmatrix*}$
%% \item[k.] $\begin{bmatrix*}[r]1&-2&2\\-1&0&-1\\0&2&-1\end{bmatrix*}$
%% \item[p.] $\begin{bmatrix*}[r]1&0&0&1\\0&1&1&1\\0&0&2&0\\0&0&0&2\end{bmatrix*}$
%% \end{enumerate}
\[ \text{f.} \begin{bmatrix*}[r]1&1\\-1&3\end{bmatrix*}, \qquad
\text{k.} \begin{bmatrix*}[r]1&-2&2\\-1&0&-1\\0&2&-1\end{bmatrix*}, \quad \text{ and }\quad
\text{p.} \begin{bmatrix*}[r]1&0&0&1\\0&1&1&1\\0&0&2&0\\0&0&0&2\end{bmatrix*}.
\]
Decide which of these matrices is diagonalizable.  Give your reasoning.
\end{problem}
%% \medskip
%% \begin{solution}
%% \end{solution}
\newpage

%--  PROBLEM 2  ---------------------------------------------------
\begin{problem}[cf.~SA 6.2.2]
Three of the six claims below are valid, three are not.
Prove the valid claims and find a counterexample refuting the invalid claims.
\begin{enumerate}[a.]
\item If $A$ is an $n \times n$ matrix with $n$ distinct (real) eigenvalues,
  then $A$ is diagonalizable.
\item If $A$ is diagonalizable and $AB = BA$, then $B$ is diagonalizable.
\item If there is an invertible matrix $P$ so that $A = P^{-1}BP$, then $A$ and
  $B$ have the same eigenvalues.
\item If $A$ and $B$ have the same eigenvalues, then there is an invertible matrix $P$ so that
$A = P^{-1}BP$.
\item There is no real $2 \times 2$ matrix $A$ satisfying $A^2 = -I_2$.
\item If $A$ and $B$ are diagonalizable and have the same eigenvalues (with the same algebraic
multiplicities), then there is an invertible matrix $P$ so that $A = P^{-1}BP$.
\end{enumerate}
{\it Hints:} You may find Lemma 1.4 and Corollary 2.2 helpful. 
\end{problem}
%% \medskip
%% \begin{solution}
%% \end{solution}
\newpage

%--  EXERCISE  ---------------------------------------------------
\noindent The following exercise is recommended but not required.
\begin{ex}[SA 6.2.11]
Suppose $A$ is an $n \times n$ matrix with the property that $A^2 = A$.
\begin{enumerate}[a.]
\item Show that if $\lambda$ is an eigenvalue of $A$, then $\lambda = 0$ or $\lambda = 1$.
\item Prove that $A$ is diagonalizable. ({\it Hint:} See Exercise 3.2.13.)
\end{enumerate}
\end{ex}
%% \medskip
%% \begin{solution}
%% \end{solution}
\newpage

%--  PROBLEM 3  ---------------------------------------------------
\begin{problem}[SA 6.2.20a]
Let $\lambda$ and $\mu$ be distinct eigenvalues of a linear
transformation. Suppose 
$\{\vv_1,\dots, \vv_k\} \subset \mathbf{E}(\lambda)$
is linearly independent and 
$\{\vw_1,\dots, \vw_\ell\} \subset \mathbf{E}(\mu)$
is linearly independent.
Prove that $\{\vv_1,\dots, \vv_k, \vw_1,\dots, \vw_\ell\}$ is linearly
independent. (This is essentially the result needed
to complete the proof of Theorem 2.4.)
\end{problem}
%% \medskip
%% \begin{solution}
%% \end{solution}
\newpage

\section*{Section 6.4}
%--  PROBLEM 5  ---------------------------------------------------
\begin{problem}[SA 6.4.9]\
  \begin{enumerate}[a.]
  \item   Suppose $A$ is a symmetric $n \times n$ matrix. Using the Spectral Theorem, 
    prove that if $A\vx \cdot \vx = 0$ for every vector $\vx \in \R^n$, then $A = O$.
  \item Give a counterexample to show that if we drop the symmetry hypothesis, then the statement
    in part a is false.
  \end{enumerate}
\end{problem}
%% \medskip
%% \begin{solution}
%% \end{solution}
\newpage
\noindent The next exercise is recommended but not required.
%--  PROBLEM 4  ---------------------------------------------------
\begin{ex}[cf.~SA 6.4.3]
Find a matrix $A$ with the following properties:
\begin{itemize}
\item $A$ is symmetric;
\item $A$ has eigenvalues 1 and 2;
\item $\bE(2) = \Span \bigl\{(1,1,1)\bigr\}$. %\begin{bmatrix*}[r]1\\1\\1\end{bmatrix*}\right\}$.
\end{itemize}
({\it Hint:} the Spectral Theorem implies $\bE(1) =\bE(2)^\top$.
Find a basis for $\bE(1)$, and then compute $A = P \Lambda P^{-1}$ for 
an appropriate choice of $P$ and $\Lambda$.)
\end{ex}
%% \medskip
%% \begin{solution}
%% \end{solution}
\end{document}
