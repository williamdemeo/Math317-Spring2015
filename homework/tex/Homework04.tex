%%This is the LaTeX file for the Math 317 Homework.  

%% You may use this file to fill in your solutions if you know how 
%% to compile a LaTeX file to turn it into a pdf document.  If you 
%% don't know how to do this but want to learn, there are plenty of
%% resources for learning about LaTeX on the net.  You may also ask
%% your professor for help.
%% 
%% For more information about using this file to complete the homework,
%% please see the NOTES section below of the comments below.

\documentclass[fleqn,11pt]{paper}


\usepackage[
  letterpaper,
  top    = 3cm,
  bottom = 3cm,
  left   = 3.00cm,
  right  = 3.00cm]{geometry}

\usepackage{tikz-cd}
\usepackage{amsthm}
\usepackage{scalefnt}

%%%%%%%%%%%%%%%%%%%%%%%%%%%%%%%%%%%%%%%%
% Basic packages
%%%%%%%%%%%%%%%%%%%%%%%%%%%%%%%%%%%%%%%%
\usepackage{amsmath,amsthm,amssymb}
\usepackage{mathtools}
\usepackage{etoolbox}
\usepackage{fancyhdr}
\usepackage{xcolor}
\usepackage[colorlinks=true,urlcolor=blue,linkcolor=blue,citecolor=blue]{hyperref}
\usepackage{xspace}
\usepackage{comment}
\usepackage{url} % for url in bib entries
\usepackage{mathrsfs}

\theoremstyle{remark}
\newtheorem{theorem}{Theorem}
\newtheorem*{prop}{Proposition}
\newtheorem{problem}{Problem}
\newtheorem*{prob}{Problem}
\newtheorem*{solution}{{\bf Solution}}
\newtheorem*{hint}{{\it Hint}}
\newtheorem*{ex}{Exercise}

\usepackage{background}
\backgroundsetup{angle=0,scale=1,color=black,position=current page.north west,hshift=100,vshift=-25,
  contents={\ifnum\value{page}=1 $<$--- {\bf staple required}\else \fi}}
\pagestyle{empty}

%%%%%%%%%%%%%%%%%%%%%%%%%%%%%%%%%%%%%%%%%%%%%%%%%%
%% Surround the problem and solution with 
%% \begin{ProbBox}  and   \end{ProbBox}
%% to prevent pagebreaks.
\newenvironment{ProbBox}{\noindent\begin{minipage}{\linewidth}}{\end{minipage}}

%%%%%%%%%%%%%%%%
% Acronyms     %
%%%%%%%%%%%%%%%%
\usepackage[acronym, shortcuts]{glossaries}

%% HERE IS HOW YOU DEFINE ACRONYMS:
\newacronym{FTA}{FTA}{Fundamental Theorem of Algebra}
\newacronym{CRT}{CRT}{Chinese Remainder Theorem}

% Make \ac robust.
\robustify{\ac}

%%%%%%%%%%%%%%%%%%%%%%%%
% Fancy page style     %
%%%%%%%%%%%%%%%%%%%%%%%%
\pagestyle{fancy}
\newcommand{\metadata}[2]{
  \lhead{}
  \chead{}
  \rhead{\bfseries Math 317: Linear Algebra}
  \lfoot{#1}
  \cfoot{#2}
  \rfoot{\thepage}
}
\renewcommand{\headrulewidth}{0.4pt}
\renewcommand{\footrulewidth}{0.4pt}


\newrobustcmd*{\vocab}[1]{\emph{#1}}
\newrobustcmd*{\latin}[1]{\textit{#1}}

%%%%%%%%%%%%%%%%%%%%%%%%%%%%%%%%%%
% Customize list enviroonments   %
%%%%%%%%%%%%%%%%%%%%%%%%%%%%%%%%%%
% package to customize three basic list environments: enumerate, itemize and description.
%% \usepackage{enumitem}
%% \setitemize{noitemsep, topsep=0pt, leftmargin=*}
%% \setenumerate{noitemsep, topsep=0pt, leftmargin=*}
%% \setdescription{noitemsep, topsep=0pt, leftmargin=*}

\usepackage{enumerate}

%%%%%%%%%%%%%%%%%%%%%%%%%%%%
%% Space between problems  %
%%%%%%%%%%%%%%%%%%%%%%%%%%%%
\newrobustcmd*{\probskip}{\vskip1cm}


%%%%%%%%%%%%%%%%%%%%%%%%%%
%%    Math shortcuts     %
%%%%%%%%%%%%%%%%%%%%%%%%%%
%\newcommand\iff{\ensuremath{\Longleftrightarrow}}
\newcommand\join{\ensuremath{\vee}}
\newcommand\meet{\ensuremath{\wedge}}
\newcommand\R{\fld{R}}
\newcommand\proj{\ensuremath{\operatorname{proj}}}
\newcommand\End{\ensuremath{\operatorname{End}}}
\newcommand\Aut{\ensuremath{\operatorname{Aut}}}
\newcommand\Hom{\ensuremath{\operatorname{Hom}}}
\newcommand{\Aff}{\ensuremath{\operatorname{Aff}}}
\newcommand{\ann}[1]{\ensuremath{\operatorname{ann}(#1)}}
\newcommand{\id}{\ensuremath{\operatorname{id}}}
\newcommand{\nulity}[1]{\ensuremath{\operatorname{null}(#1)}}
\renewcommand{\ker}[1]{\ensuremath{\operatorname{ker}(#1)}}
\renewcommand{\dim}[1]{\ensuremath{\operatorname{dim}(#1)}}
\newcommand\im[1]{\ensuremath{\operatorname{im}(#1)}}
\newcommand{\rank}[1]{\ensuremath{\operatorname{rank}(#1)}}
\newcommand{\trace}[1]{\ensuremath{\operatorname{trace}(#1)}}
\renewcommand{\phi}{\ensuremath{\varphi}}

\renewcommand{\vec}[1]{\mathbf{#1}}



%%%% NOTES %%%%%%%%%%%%%%%%%%%%%%%%%%%%%%%%%%%%%%%%%%%%%%%%%%%%%%%%% 

%%
%%    1. If possible, try to write your answers inside a 
%%
%%               \begin{solution}...\end{solution}
%%
%%       environment.
%%
%%    2. If you enter your answers into this Homework*.tex source file, you
%%       will have to compile it into a pdf document.  There are a number of ways to
%%       do that. Probably the easiest is to use the website called ShareLaTeX.com.
%%       Alternatively,
%%
%%           Linux: most come with TeX; otherwise do a full install of TeXLive.
%%           Mac OS X: you might try MacTeX. 
%%           Windows: try proTeXt maybe? (or switch to a better operating system!) 
%%
%%       There is a Makefile in this directory, so on Linux you could just 
%%       enter `make` to compile all the Homework*.tex files at once.
%%
%%    3. Please don't hesitate to inform the professor if you have trouble; 
%%       post to Piazza or ask in lecture.
%%
%%    4. Try to use standard notation, as used in class and in the textbook.
%%       For the most basic symbols, you may wish to use LaTeX macros to keep 
%%       the conventions you use consistent and easy to remember.
%%       For example, to denote an algebra,
\newcommand\alg[1]{\ensuremath{\mathbf{#1}}}
\newcommand{\<}{\ensuremath{\langle}}
\renewcommand{\>}{\ensuremath{\rangle}}
%%       So, an algebra in LaTeX is typed as $\alg{A} = \<A, F\>$.
%%       Similarly, for a field, let's use:
\newcommand\fld[1]{\ensuremath{\mathbb{#1}}}
%%       So, a field in LaTeX is typed as $\fld{F}$.
%%
%%       To make a boldface vector, use backslash v in front of the 
%%       letter and add a new command for that letter here:
\newcommand\va{\vec{a}}
\newcommand\vb{\vec{b}}
\newcommand\vu{\vec{u}}
\newcommand\vv{\vec{v}}
\newcommand\vw{\vec{w}}
\newcommand\vx{\vec{x}}
\newcommand\vy{\vec{y}}
\newcommand\vz{\vec{z}}
\newcommand\vzero{\vec{0}}
\newcommand\sP{\ensuremath{\mathscr P}}
\newcommand\Span{\ensuremath{\operatorname{Span}}}
%%
%%    5. Insert your name here!!!
%%

\metadata       {Name:                     }{HW 4 (due 2016/02/12)}
%% Put your name here ^^^^^^^^^^^^^^^^^^^^^, like this:
%% \metadata  {Name: William DeMeo}{HW 4 (due 2016/02/12)}  
%%

\author         {NAME:                     }
%% Put your name here ^^^^^^^^^^^^^^^^^^^^^, like this:
%%     \author{NAME: William DeMeo}
%%
%%
%%
%%    6. Update the title and date if necessary.
\title{Math 317: Homework 4}
\date{Due: 12 Feb 2016}
%%
%%%%%%%%%%%%%%%%%%%%%%%%%%%%%%%%%%%%%%%%%%%%%%%%%%%%%%%%%%%%%%%%%%% 


\begin{document}
\maketitle
\noindent {\bf Note:} Only numbered exercises have potential to be graded.  Unnumbered exercises 
(like the first problem in this assignment) are recommended but not required. Solutions to unnumbered problems
will not be graded.

\begin{prob}[Block multiplication] We can think of an $(m + n) \times (m + n)$ 
matrix as being decomposed into ``blocks,'' and thinking of these blocks as matrices themselves, we
can form products and sums appropriately. Suppose $A$ and $A'$ are $m \times m$ matrices, $B$ and $B'$
are $m \times n$ matrices, $C$ and $C'$ are $n \times m$ matrices, and $D$ and $D'$ are 
$n \times n$ matrices.
Verify the following formula for the product of ``block'' matrices:
\begin{equation}
  \label{eq:1}
\left[ 
  \begin{array}{c|c}
    A &B \\
    \hline 
    C &D 
  \end{array}
\right]
\left[ 
  \begin{array}{c|c}A' &B' \\ \hline C' &D' \end{array}
\right]
  =
\left[ 
  \begin{array}{c|c}
    AA' + BC' & AB' + BD' \\ 
    \hline 
    CA' + DC' & CB'+ DD' 
  \end{array}
\right].
\end{equation}
\end{prob}
%%%   (remove the %%% characters and enter your solution below)
%%% \begin{solution}
%%%   (type your solution here)
%%% \end{solution}
\newpage

%--  PROBLEM 1  ---------------------------------------------------
\begin{problem}[SA 2.1.10]
  Suppose $A$ and $B$ are nonsingular $n \times n$ matrices.
  Prove that $AB$ is nonsingular.
  \begin{quote}
    {\it Hint:} Although it is tempting to try to show that the reduced echelon
    form of $AB$ is the identity matrix, there is no direct way to do this. As is
    the case in most non-numerical problems regarding nonsingularity, you should
    remember that $AB$ is nonsingular precisely when the only solution of
    $(AB)\vx = \vzero$ is $\vx = \vzero$.
  \end{quote}
  
\end{problem}
%%%   (remove the %%% characters and enter your solution below)
%%% \begin{solution}
%%%   (type your solution here)
%%% \end{solution}
\newpage


%--  PROBLEM 2  ---------------------------------------------------
\begin{problem}[SA 2.2.1]
  Suppose that $T : \R^3 \to \R^2$ is a linear transformation and that
  \[
  T\begin{bmatrix*}[r] 1 \\ 2 \\ 1 \end{bmatrix*}= \begin{bmatrix*}[r] -1 \\ 2\end{bmatrix*}
    \; \text{ and } \;
    T\begin{bmatrix*}[r] 2 \\ -1 \\ 1 \end{bmatrix*}= \begin{bmatrix*}[r] 3 \\ 0\end{bmatrix*}.
\quad
      \text{ Compute }
      T\left(2\begin{bmatrix*}[r] 2 \\ -1 \\ 1 \end{bmatrix*}\right),
      T\begin{bmatrix*}[r] 3 \\ 6 \\ 3 \end{bmatrix*}, \text{ and  }
      T\begin{bmatrix*}[r] -1 \\ 3 \\ 0 \end{bmatrix*}.\]

      %% $T\left(2\begin{bmatrix*}[r] 2 \\ -1 \\ 1 \end{bmatrix*}\right)$,
      %% $T\begin{bmatrix*}[r] 3 \\ 6 \\ 3 \end{bmatrix*}$, and 
      %% $T\begin{bmatrix*}[r] -1 \\ 3 \\ 0 \end{bmatrix*}$.

      \noindent {\it Hint:} begin by writing each vector as a linear combination of 
      the vectors 
      $\begin{bmatrix*}[r] 1 \\ 2 \\ 1 \end{bmatrix*}$,
      $\begin{bmatrix*}[r] 2 \\ -1 \\ 1 \end{bmatrix*}$.
\end{problem}
%%%   (remove the %%% characters and enter your solution below)
%%% \begin{solution}
%%%   (type your solution here)
%%% \end{solution}
\newpage


%--  PROBLEM 3  ---------------------------------------------------
\begin{problem}[SA 2.2.3b]
  Let $T : \R^2 \to \R^2$ be a linear transformation with
  $T\begin{bmatrix*}[r] 2 \\ 1\end{bmatrix*} = \begin{bmatrix*}[r] 5 \\ 3\end{bmatrix*}$,
    and 
    $T\begin{bmatrix*}[r] 0 \\ 1\end{bmatrix*} = \begin{bmatrix*}[r] 1 \\ -3\end{bmatrix*}$.
      Find the standard matrix representation of $T$.
\end{problem}
%%%   (remove the %%% characters and enter your solution below)
%%% \begin{solution}
%%%   (type your solution here)
%%% \end{solution}
\newpage


%--  PROBLEM 4  ---------------------------------------------------
\begin{problem}[SA 2.2.4bef]
  Determine whether each of the following functions is a linear transformation. If so,
  provide a proof; if not, explain why.
  \begin{enumerate}
  \item[b.]  
    $T\begin{bmatrix*}[r] x_1 \\ x_2\end{bmatrix*} = \begin{bmatrix*}[c]  x_1 + 2x_2 \\ 0\end{bmatrix*}$.
  \item[e.]  
    $T\begin{bmatrix*}[r] x_1 \\ x_2\end{bmatrix*} = \begin{bmatrix*}[c]  x_1 + 2x_2 \\ x_2\\-x_1+3x_2\end{bmatrix*}$.
  \item[f.]  
    $T: \R^n \to \R$ given by $T\vx = \|\vx\|$.
  \end{enumerate}
\end{problem}
%%%   (remove the %%% characters and enter your solution below)
%%% \begin{solution}
%%%   (type your solution here)
%%% \end{solution}
\newpage


%--  PROBLEM 5  ---------------------------------------------------
\begin{problem}[SA 2.2.11]
Solving each part is recommended, but only part {\bf c} will be graded.
  \begin{enumerate}[a.]
  \item 
    Prove that if $T : \R^n \to \R^m$ is a linear transformation and $c$ is any
    scalar, then the function $cT : \R^n \to \R^m$ defined by 
    $(cT)(\vx) = c T(\vx)$ (i.e., the scalar $c$ times the vector $T(\vx)$) 
    is also a linear transformation.
  \item Prove that if $S : \R^n \to \R^m$ and $T : \R^n \to \R^m$ are linear
    transformations, then the function $S + T : \R^n \to \R^m$ defined by 
    $(S + T)(\vx) = S(\vx) + T (\vx)$ is also a linear transformation.
  \item[{\bf c.}] Prove that if 
    $S : \R^m \to \R^p$ and $T : \R^n \to \R^m$ are linear
    transformations, then the function $S \circ T : \R^n \to \R^p$ defined by
    $(S\circ T)(\vx) = S(T(\vx))$ is also a linear transformation.
  \end{enumerate}
\end{problem}
%%%   (remove the %%% characters and enter your solution below)
%%% \begin{solution}
%%%   (type your solution here)
%%% \end{solution}
\newpage


%--  PROBLEM 6  ---------------------------------------------------
\begin{problem}[SA 2.3.1d]
  Use Gaussian elimination to find $A^{-1}$ (if it exists) where 
  \[
  A = \begin{bmatrix*}[r] 1&2&3\\1&1&2\\0&1&2\end{bmatrix*}.
    \]
\end{problem}
%%%   (remove the %%% characters and enter your solution below)
%%% \begin{solution}
%%%   (type your solution here)
%%% \end{solution}
\newpage


%--  PROBLEM 7  ---------------------------------------------------
\begin{problem}[SA 2.3.2c]
  Let
  $A = \begin{bmatrix*}[r] 1&1&1\\0&1&1\\1&2&1\end{bmatrix*}$ 
    and 
    $\vb = \begin{bmatrix*}[r] 3\\0\\1\end{bmatrix*}$.
      \begin{enumerate}[(i)]
      \item Find $A^{-1}$.
      \item Use your answer to (i) to solve $A\vx = \vb$.
      \item Use your answer to (ii) to express $\vb$ as a linear combination of the columns of $A$.
      \end{enumerate}
\end{problem}
%%%   (remove the %%% characters and enter your solution below)
%%% \begin{solution}
%%%   (type your solution here)
%%% \end{solution}
\newpage


%--  PROBLEM 8  ---------------------------------------------------
\begin{problem}[SA 2.3.12]
  Suppose $A$ is an invertible $m \times m$ matrix and $B$ is an invertible 
  $n \times n$ matrix. Let $O$ denote a matrix of all zeros (with the appropriate
  dimensions).
 \begin{enumerate}[a.]
  \item 
    Show that the matrix
    $\left[ 
      \begin{array}{c|c}
        A &O \\ 
        \hline
        O &B
      \end{array}
      \right]$
    is invertible and give a formula for its inverse.

  \item Suppose $C$ is an arbitrary $m \times n$ matrix. Is the matrix
    $\left[ 
    \begin{array}{c|c}
      A &C \\ 
      \hline
      O &B
    \end{array}
    \right]$
    invertible? If so, find a formula for the inverse.  If not, explain why not.
  \end{enumerate}
  [{\it Hint:} to see how ``block matrix multiplication'' works,
    look at the first (optional) exercise of this assignment.]
 
\end{problem}
%%%   (remove the %%% characters and enter your solution below)
%%% \begin{solution}
%%%   (type your solution here)
%%% \end{solution}
\newpage

\noindent The next problem is recommended, but solutions will not be graded.
%%--  PROBLEM  ---------------------------------------------------
\begin{prob}[SA 2.3.3]
Suppose $A$ is an $n \times n$ matrix and $B$ is an $n \times n$ invertible matrix.
  Simplify the following.
  \begin{enumerate}[a.]
  \item $(BAB^{-1})^2$
  \item $(BAB^{-1})^n$ ($n$ a positive integer)
  \item $(BAB^{-1})^{-1}$ (What additional assumption is required here?)
  \end{enumerate}
\end{prob}
%%%   (remove the %%% characters and enter your solution below)
%%% \begin{solution}
%%%   (type your solution here)
%%% \end{solution}


% \bibliographystyle{plain}
%% \bibliography{refs}

\end{document}
