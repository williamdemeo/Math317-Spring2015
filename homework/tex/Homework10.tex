%%This is the LaTeX file for the Math 317 Homework.  

%% You may use this file to fill in your solutions if you know how 
%% to compile a LaTeX file to turn it into a pdf document.  If you 
%% don't know how to do this but want to learn, there are plenty of
%% resources for learning about LaTeX on the net.  You may also ask
%% your professor for help.
%% 
%% For more information about using this file to complete the homework,
%% please see the NOTES section below of the comments below.

\documentclass[fleqn,11pt]{paper}


\usepackage[
letterpaper,
top    = 3cm,
bottom = 3cm,
left   = 3.00cm,
right  = 3.00cm]{geometry}

\usepackage{tikz-cd}
\usepackage{amsthm}
\usepackage{scalefnt}

%%%%%%%%%%%%%%%%%%%%%%%%%%%%%%%%%%%%%%%%
% Basic packages
%%%%%%%%%%%%%%%%%%%%%%%%%%%%%%%%%%%%%%%%
\usepackage{amsmath,amsthm,amssymb}
\usepackage{mathtools}
\usepackage{etoolbox}
\usepackage[mathcal]{euscript}
\usepackage{fancyhdr}
 \usepackage{xcolor}
\usepackage[colorlinks=true,urlcolor=blue,linkcolor=blue,citecolor=blue]{hyperref}
\usepackage{xspace}
\usepackage{comment}
\usepackage{url} % for url in bib entries
\usepackage{mathrsfs}

\theoremstyle{remark}
\newtheorem{theorem}{Theorem}
\newtheorem*{prop}{Proposition}
\newtheorem{problem}{Problem}
\newtheorem*{prob}{Problem}
\newtheorem*{exercise}{Exercise}
\newtheorem*{solution}{{\bf Solution}}
\newtheorem*{hint}{{\it Hint}}
\newtheorem*{ex}{Exercise}


%%%%%%%%%%%%%%%%%%%%%%%%%%%%%%%%%%%%%%%%%%%%%%%%%%
%% Surround the problem and solution with 
%% \begin{ProbBox}  and   \end{ProbBox}
%% to prevent pagebreaks.
\newenvironment{ProbBox}{\noindent\begin{minipage}{\linewidth}}{\end{minipage}}

%%%%%%%%%%%%%%%%
% Acronyms     %
%%%%%%%%%%%%%%%%
\usepackage[acronym, shortcuts]{glossaries}

%% HERE IS HOW YOU DEFINE ACRONYMS:
\newacronym{FTA}{FTA}{Fundamental Theorem of Algebra}
\newacronym{CRT}{CRT}{Chinese Remainder Theorem}

% Make \ac robust.
\robustify{\ac}

%%%%%%%%%%%%%%%%%%%%%%%%
% Fancy page style     %
%%%%%%%%%%%%%%%%%%%%%%%%
\pagestyle{fancy}
\newcommand{\metadata}[2]{
  \lhead{}
  \chead{}
  \rhead{\bfseries Math 317: Linear Algebra}
  \lfoot{#1}
  \cfoot{#2}
  \rfoot{\thepage}
}
\renewcommand{\headrulewidth}{0.4pt}
\renewcommand{\footrulewidth}{0.4pt}


\newrobustcmd*{\vocab}[1]{\emph{#1}}
\newrobustcmd*{\latin}[1]{\textit{#1}}

%%%%%%%%%%%%%%%%%%%%%%%%%%%%%%%%%%
% Customize list enviroonments   %
%%%%%%%%%%%%%%%%%%%%%%%%%%%%%%%%%%
% package to customize three basic list environments: enumerate, itemize and description.
%% \usepackage{enumitem}
%% \setitemize{noitemsep, topsep=0pt, leftmargin=*}
%% \setenumerate{noitemsep, topsep=0pt, leftmargin=*}
%% \setdescription{noitemsep, topsep=0pt, leftmargin=*}

\usepackage{enumerate}

%%%%%%%%%%%%%%%%%%%%%%%%%%%%
%% Space between problems  %
%%%%%%%%%%%%%%%%%%%%%%%%%%%%
\newrobustcmd*{\probskip}{\vskip1cm}


%%%%%%%%%%%%%%%%%%%%%%%%%%
%%    Math shortcuts     %
%%%%%%%%%%%%%%%%%%%%%%%%%%
%\newcommand\iff{\ensuremath{\Longleftrightarrow}}
\newcommand\join{\ensuremath{\vee}}
\newcommand\meet{\ensuremath{\wedge}}
\newcommand\R{\fld{R}}
\newcommand\proj{\ensuremath{\operatorname{proj}}}
\newcommand\End{\ensuremath{\operatorname{End}}}
\newcommand\Aut{\ensuremath{\operatorname{Aut}}}
\newcommand\Hom{\ensuremath{\operatorname{Hom}}}
\newcommand{\Aff}{\ensuremath{\operatorname{Aff}}}
\newcommand{\ann}[1]{\ensuremath{\operatorname{ann}(#1)}}
\newcommand{\id}{\ensuremath{\operatorname{id}}}
%% \newcommand{\nulity}[1]{\ensuremath{\operatorname{null}(#1)}}
%% \renewcommand{\ker}[1]{\ensuremath{\operatorname{ker}(#1)}}
%% \renewcommand{\dim}[1]{\ensuremath{\operatorname{dim}(#1)}}
%% \newcommand\im[1]{\ensuremath{\operatorname{im}(#1)}}
%% \newcommand{\rank}[1]{\ensuremath{\operatorname{Rank}(#1)}}
%% \newcommand{\trace}[1]{\ensuremath{\operatorname{trace}(#1)}}
\newcommand{\nulity}{\ensuremath{\operatorname{null}}}
\renewcommand{\ker}{\ensuremath{\operatorname{ker}}}
\renewcommand{\dim}{\ensuremath{\operatorname{dim}}}
\newcommand\im{\ensuremath{\operatorname{im}}}
\newcommand{\rank}{\ensuremath{\operatorname{rank}}}
\newcommand{\trace}{\ensuremath{\operatorname{trace}}}
\newcommand{\vR}{\ensuremath{\operatorname{R}}}
\newcommand{\vN}{\ensuremath{\operatorname{N}}}
\newcommand{\vC}{\ensuremath{\operatorname{C}}}
\renewcommand{\phi}{\ensuremath{\varphi}}

\renewcommand{\vec}[1]{\mathbf{#1}}

%%       Try to use standard notation, as used in class and in the textbook.
%%       For the most basic symbols, you may wish to use LaTeX macros to keep 
%%       the conventions you use consistent and easy to remember.
%%       For example, to denote an algebra,
         \newcommand\alg[1]{\ensuremath{\mathbf{#1}}}
         \newcommand{\<}{\ensuremath{\langle}}
         \renewcommand{\>}{\ensuremath{\rangle}}
%%       So, an algebra in LaTeX is typed as $\alg{A} = \<A, F\>$.
%%       Similarly, for a field, let's use:
         \newcommand\fld[1]{\ensuremath{\mathbb{#1}}}
%%       So, a field in LaTeX is typed as $\fld{F}$.
%%
%%       To make a boldface vector, use backslash v in front of the 
%%       letter and add a new command for that letter here or in 
%%       the macros.tex file:
         \newcommand\va{\vec{a}}
         \newcommand\vc{\vec{c}}
         \newcommand\ve{\vec{e}}
         \newcommand\vu{\vec{u}}
         \newcommand\vv{\vec{v}}
         \newcommand\vw{\vec{w}}
         \newcommand\vx{\vec{x}}
         \newcommand\vy{\vec{y}}
         \newcommand\vz{\vec{z}}
         \newcommand\vzero{\vec{0}}
         \newcommand\sV{\ensuremath{\mathcal V}}
         \newcommand\sB{\ensuremath{\mathcal B}}
         \newcommand\sE{\ensuremath{\mathcal E}}
         \newcommand\sW{\ensuremath{\mathcal W}}
         \newcommand\sP{\ensuremath{\mathcal P}}
         \newcommand\Span{\ensuremath{\operatorname{Span}}}
%%
%%       Insert your name here!!!
%%

         \author         {NAME:                     }
         %% Put your name here ^^^^^^^^^^^^^^^^^^^^^, like this:
         %%     \author{NAME: William DeMeo}
%%
%%       Update the title and date if necessary.
         \title{Math 317: Homework 10}
         \date{31 March 2016}

         \metadata       {HW 10}{(not due)}
         
%%
%%%%%%%%%%%%%%%%%%%%%%%%%%%%%%%%%%%%%%%%%%%%%%%%%%%%%%%%%%%%%%%%%%% 


\begin{document}

\maketitle
%--  EXERCISE 1 ---------------------------------------------------
\begin{exercise}[SA 4.3.5a]
  Let $\vv_1 = \begin{bmatrix*}[r]2\\3\end{bmatrix*}$
  and $\vv_2 = \begin{bmatrix*}[r]1\\2\end{bmatrix*}$,
  and consider the basis $\sB = \{\vv_1, \vv_2\}$ for $\R^2$.
  Suppose $T : \R^2 \to \R^2$ is a linear transformation whose standard matrix is 
$[T]_\sE = \begin{bmatrix*}[r]1& 5\\2 &-2 \end{bmatrix*}$.
    Find the matrix $[T]_\sB$.
\end{exercise}
%% \medskip
%% \begin{solution}
%% \end{solution}

\vskip5cm

%--  EXERCISE 2 ---------------------------------------------------
\begin{exercise}[SA 4.3.7]
  Suppose the standard matrix for a linear transformation $T : \R^3 \to \R^3$ is
  $[T]_\sE = \begin{bmatrix*}[r]
-1 &2 &1\\
0 &1 &3\\
1 &-1& 1
  \end{bmatrix*}$.
Find the matrix representation $[T]_\sB$ of $T$ with respect to the basis
$\sB = \left\{
\begin{bmatrix*}[r]1\\0\\-1 \end{bmatrix*},
\begin{bmatrix*}[r]0\\2\\3 \end{bmatrix*},
\begin{bmatrix*}[r]1\\1\\1 \end{bmatrix*}\right\}$.
\end{exercise}
%% \medskip
%% \begin{solution}
%% \end{solution}

\vfill

\noindent {\it Hint:} For the first two exercises, apply the change-of-basis formula given
in Proposition 3.2 on page 215 of the textbook.
\newpage

%--  EXERCISE 3  ---------------------------------------------------
\begin{exercise}[SA 4.3.16]
Let $V = \Span((1, 0, 2, 1), (0, 1,-1, 1)) \leq \R^4$.
Use the change-of-basis formula to find the standard matrix for $\proj_V: \R^4 \to \R^4$.
({\it Hint:} this is similar to Example 9 on page 216.)
\end{exercise}
%% \medskip
%% \begin{solution}
%% \end{solution}
  
\newpage

%--  EXERCISE ---------------------------------------------------
\begin{exercise}[SA 4.3.19] 
Let $\ve_1$, $\ve_2$ denote the standard basis, as usual. Let $T : \R^2 \to \R^2$ be defined by
\[
T (\ve_1) = 8\ve_1 - 4\ve_2 \quad \text{ and } \quad 
T (\ve_2) = 9\ve_1 - 4\ve_2.
\]
\begin{enumerate}
\item 
Give the standard matrix for $T$.
\item Let $\vv_1 = 3\ve_1 - 2\ve_2$ and $\vv_2 = -\ve_1 + \ve_2$. Calculate the matrix for $T$ with respect to
the basis $\sB = \{\vv_1, \vv_2\}$.
\item Is $T$ diagonalizable? Give your reasoning. (Hint: See part d of Exercise
  18.)
\end{enumerate}
\end{exercise}
%% \medskip
%% \begin{solution}
%% \end{solution}


\end{document}

First, we need to extend the given basis for $V$ to a basis for all
of $\mathbb R^4$.  It's not too hard to find two more vectors so that the four
vectors are linearly independent.  For example, you could use 
\[
\mathcal{B} = \{(1,0,2,1), (0,1,-1,1), (1,1,0,0), (0,1,0,1)\}.
\]
Now, the projection matrix onto $V$ in this basis is just
\[
  [\proj_V]_{\mathcal B} = \begin{bmatrix*}[r]
    1&0&0&0\\
    0&1&0&0\\
    0&0&0&0\\
    0&0&0&0
  \end{bmatrix*}
\]
Do you see why?  Given a $\mathcal B$-basis representation
$[\mathbf{x}]_{\mathcal B}$ of a vector $\mathbf{x}$, in order to project onto
$V$ you just project onto the first two basis vectors.  If you are working
in the basis $\mathcal{B}$, this simply means pick out the first two coordinates
of $[\mathbf{x}]_{\mathcal B}$ and zero out the other two. That's exactly what the matrix above does.

Next let's form the matrix $P$ that has the basis vectors of $\mathcal B$ as
columns.  Compute $P^{-1}$, and then your projection matrix in the standard basis will be
\[[\proj_V]_{\mathcal E} = P[\proj_V]_{\mathcal B} P^{-1}\]
You could do this by hand, but let's have Sage do it for us.
We get the following projection matrix in the standard basis
(which, incidentally, is different from the answer in the back of the book):
\[
  [P_V]_{\mathcal E} = \frac{1}{2}
  \begin{bmatrix*}[r]
    1&-1&0&1\\
    2&-2&-2&2\\
    0&0&2&0\\
    3&-3&-2&3
  \end{bmatrix*}
\]
\end{solution}
I wouldn't go so far as to suggest that the book's answer is wrong, but we can be 
fairly certain the answer above is right, especially if we test it on some vectors.
For example, if we take $[\mathbf{x}]_{\mathcal B} = (1,1,1,1)$, the projection
of this in the basis $\mathcal B$ is simply $(1,1,0,0)$.  The standard basis
representation of this vector is $\mathbf{x} = (2,3,1,3)$.  If we take the
projection of this vector onto $V$ in the standard basis, using the
matrix $[\operatorname{proj}_V]_{\mathcal E}$ computed above, we
get $(1,1,1,2)$. If we then convert this back to the $\mathcal B$-basis, we
get $(1,1,0,0)$, as expected.

