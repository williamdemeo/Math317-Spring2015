%%This is the LaTeX file for the Math 317 Homework.  

%% You may use this file to fill in your solutions if you know how 
%% to compile a LaTeX file to turn it into a pdf document.  If you 
%% don't know how to do this but want to learn, there are plenty of
%% resources for learning about LaTeX on the net.  You may also ask
%% your professor for help.
%% 
%% For more information about using this file to complete the homework,
%% please see the NOTES section below of the comments below.

\documentclass[fleqn,11pt]{paper}


\usepackage[
letterpaper,
top    = 3cm,
bottom = 3cm,
left   = 3.00cm,
right  = 3.00cm]{geometry}

\usepackage{tikz-cd}
\usepackage{amsthm}
\usepackage{scalefnt}

%%%%%%%%%%%%%%%%%%%%%%%%%%%%%%%%%%%%%%%%
% Basic packages
%%%%%%%%%%%%%%%%%%%%%%%%%%%%%%%%%%%%%%%%
\usepackage{amsmath,amsthm,amssymb}
\usepackage[mathcal]{euscript}
\usepackage{mathtools}
\usepackage{etoolbox}
\usepackage{fancyhdr}
 \usepackage{xcolor}
\usepackage[colorlinks=true,urlcolor=blue,linkcolor=blue,citecolor=blue]{hyperref}
\usepackage{xspace}
\usepackage{comment}
\usepackage{url} % for url in bib entries
\usepackage{mathrsfs}

\theoremstyle{remark}
\newtheorem{theorem}{Theorem}
\newtheorem*{prop}{Proposition}
\newtheorem{problem}{Problem}
\newtheorem*{prob}{Problem}
\newtheorem*{solution}{{\bf Solution}}
\newtheorem*{hint}{{\it Hint}}
\newtheorem*{ex}{Exercise}


%%%%%%%%%%%%%%%%%%%%%%%%%%%%%%%%%%%%%%%%%%%%%%%%%%
%% Surround the problem and solution with 
%% \begin{ProbBox}  and   \end{ProbBox}
%% to prevent pagebreaks.
\newenvironment{ProbBox}{\noindent\begin{minipage}{\linewidth}}{\end{minipage}}

%%%%%%%%%%%%%%%%
% Acronyms     %
%%%%%%%%%%%%%%%%
\usepackage[acronym, shortcuts]{glossaries}

%% HERE IS HOW YOU DEFINE ACRONYMS:
\newacronym{FTA}{FTA}{Fundamental Theorem of Algebra}
\newacronym{CRT}{CRT}{Chinese Remainder Theorem}

% Make \ac robust.
\robustify{\ac}

%%%%%%%%%%%%%%%%%%%%%%%%
% Fancy page style     %
%%%%%%%%%%%%%%%%%%%%%%%%
\pagestyle{fancy}
\newcommand{\metadata}[2]{
  \lhead{}
  \chead{}
  \rhead{\bfseries Math 317: Linear Algebra}
  \lfoot{#1}
  \cfoot{#2}
  \rfoot{\thepage}
}
\renewcommand{\headrulewidth}{0.4pt}
\renewcommand{\footrulewidth}{0.4pt}


\newrobustcmd*{\vocab}[1]{\emph{#1}}
\newrobustcmd*{\latin}[1]{\textit{#1}}

%%%%%%%%%%%%%%%%%%%%%%%%%%%%%%%%%%
% Customize list enviroonments   %
%%%%%%%%%%%%%%%%%%%%%%%%%%%%%%%%%%
% package to customize three basic list environments: enumerate, itemize and description.
%% \usepackage{enumitem}
%% \setitemize{noitemsep, topsep=0pt, leftmargin=*}
%% \setenumerate{noitemsep, topsep=0pt, leftmargin=*}
%% \setdescription{noitemsep, topsep=0pt, leftmargin=*}

\usepackage{enumerate}

%%%%%%%%%%%%%%%%%%%%%%%%%%%%
%% Space between problems  %
%%%%%%%%%%%%%%%%%%%%%%%%%%%%
\newrobustcmd*{\probskip}{\vskip1cm}


%%%%%%%%%%%%%%%%%%%%%%%%%%
%%    Math shortcuts     %
%%%%%%%%%%%%%%%%%%%%%%%%%%
%\newcommand\iff{\ensuremath{\Longleftrightarrow}}
\newcommand\join{\ensuremath{\vee}}
\newcommand\meet{\ensuremath{\wedge}}
\newcommand\R{\fld{R}}
\newcommand\proj{\ensuremath{\operatorname{proj}}}
\newcommand\End{\ensuremath{\operatorname{End}}}
\newcommand\Aut{\ensuremath{\operatorname{Aut}}}
\newcommand\Hom{\ensuremath{\operatorname{Hom}}}
\newcommand{\Aff}{\ensuremath{\operatorname{Aff}}}
\newcommand{\ann}[1]{\ensuremath{\operatorname{ann}(#1)}}
\newcommand{\id}{\ensuremath{\operatorname{id}}}
\newcommand{\nulity}[1]{\ensuremath{\operatorname{null}(#1)}}
%\renewcommand{\ker}[1]{\ensuremath{\operatorname{ker}(#1)}}
\renewcommand{\ker}{\ensuremath{\operatorname{ker}}}
%% \renewcommand{\dim}[1]{\ensuremath{\operatorname{dim}(#1)}}
\renewcommand{\dim}{\ensuremath{\operatorname{dim}}}
%\newcommand\im[1]{\ensuremath{\operatorname{im}(#1)}}
\newcommand\im{\ensuremath{\operatorname{im}}}
%% \newcommand{\rank}[1]{\ensuremath{\operatorname{rank}(#1)}}
\newcommand{\rank}{\ensuremath{\operatorname{rank}}}
\newcommand{\trace}[1]{\ensuremath{\operatorname{trace}(#1)}}
\renewcommand{\phi}{\ensuremath{\varphi}}
\newcommand{\vR}{\ensuremath{\operatorname{R}}}
\newcommand{\vN}{\ensuremath{\operatorname{N}}}
\newcommand{\vC}{\ensuremath{\operatorname{C}}}

\renewcommand{\vec}[1]{\mathbf{#1}}



%%%% NOTES %%%%%%%%%%%%%%%%%%%%%%%%%%%%%%%%%%%%%%%%%%%%%%%%%%%%%%%%% 
%%       Try to use standard notation, as used in class and in the textbook.
%%       For the most basic symbols, you may wish to use LaTeX macros to keep 
%%       the conventions you use consistent and easy to remember.
%%       For example, to denote an algebra,
         \newcommand\alg[1]{\ensuremath{\mathbf{#1}}}
         \newcommand{\<}{\ensuremath{\langle}}
         \renewcommand{\>}{\ensuremath{\rangle}}
%%       So, an algebra in LaTeX is typed as $\alg{A} = \<A, F\>$.
%%       Similarly, for a field, let's use:
         \newcommand\fld[1]{\ensuremath{\mathbb{#1}}}
%%       So, a field in LaTeX is typed as $\fld{F}$.
%%
%%       To make a boldface vector, use backslash v in front of the 
%%       letter and add a new command for that letter here or in 
%%       the macros.tex file:
         \newcommand\vE{\vec{E}}
         \newcommand\va{\vec{a}}
         \newcommand\vf{\vec{f}}
         \newcommand\vu{\vec{u}}
         \newcommand\vv{\vec{v}}
         \newcommand\vw{\vec{w}}
         \newcommand\vx{\vec{x}}
         \newcommand\vy{\vec{y}}
         \newcommand\vz{\vec{z}}
         \newcommand\vzero{\vec{0}}
         \newcommand\sC{\ensuremath{\mathcal C}}
         \newcommand\sB{\ensuremath{\mathcal B}}
         \newcommand\sV{\ensuremath{\mathcal V}}
         \newcommand\sW{\ensuremath{\mathcal W}}
         \newcommand\sP{\ensuremath{\mathcal P}}
         \newcommand\Span{\ensuremath{\operatorname{Span}}}
         \metadata       {HW 11}{due 2016/04/18}
%%
%%       Insert your name here!!!
         \author {NAME:                     }
%%       Your name goes here ^^^^^^^^^^^^^^, like this:
         %%     \author{NAME: William DeMeo}
%%
%%       Update the title and date if necessary.
         \title{Math 317: Homework 11}
         \date{Due: 18 April 2016}
%%
%%%%%%%%%%%%%%%%%%%%%%%%%%%%%%%%%%%%%%%%%%%%%%%%%%%%%%%%%%%%%%%%%%% 

\begin{document}

\maketitle

\section*{Section 4.4}
\noindent The first exercise is recommended but not required.  It will not be graded.
%--  PROBLEM 0  ---------------------------------------------------
\begin{ex}[SA 4.4.2]
  Let $V \leq \sC^\infty(\R)$ be the given subspace.
Let $D : V \to V$ be the \emph{differentiation operator}
$D(f) = f'$. Give the matrix for $D$ with respect to the given basis.
\begin{enumerate}[{\bf a.}]
  \item $V = \Span \{1, e^x, e^{2x}, \dots, e^{nx}\}$.
  \item $V = \Span\{e^x, xe^x, x^2 e^x, \dots, x^n e^x\}$. 
\end{enumerate}
{\it Hints:} For part (a), if you have trouble, consult the solution in back of
  the textbook. For part (b), 
we have $D(e^x) = e^x$, $D(xe^x) = e^x + xe^x$, $D(x^2e^x) = 2xe^x + x^2 e^x$, 
and, in general, $D(x^k e^x) = kx^{k-1} e^x + x^k e^x$.  Now write down the
matrix for $D$ with respect to the given basis, i.e., the basis
$\sB = \{e^x, xe^x, x^2 e^x, \dots, x^n e^x\}$. 
\end{ex}
%% \medskip
%% \begin{solution}
%% \end{solution}

\newpage


%--  PROBLEM 1  ---------------------------------------------------
\begin{problem}[SA 4.4.4]
Recall from  Example 5 (p.~229) (and from lecture), 
the matrix $[D]_{\sV, \sW}$ representing the differentiation operator 
$D : \mathcal{P}_3 \to \mathcal{P}_2$ with respect to the bases
$\sV = \{1, t, t^2 , t^3 \}$ and $\sW = \{1, t, t^2\}$ is given by
\[
[D]_{\sV, \sW} = \begin{bmatrix*}[r] 
0 & 1 & 0 & 0 \\ 
0 & 0 & 2 & 0 \\
0 & 0 & 0 & 3
\end{bmatrix*}.
\]
Use the \emph{Change-of-Basis Formula} (Theorem 4.2) to find, 
in each case, the matrix $[D]_{\sV', \sW'}$ that represents the differential operator
with respect to the given pair of bases.
\begin{enumerate}[{\bf a.}]
\item 
$\sV' = \{1, t - 1, (t - 1)^2 , (t - 1)^3 \}$ and  $\sW' = \sW$;
\item 
$\sV' = V$ and $\sW' = \{1, t - 1, (t - 1)^2 \}$;
\item
$\sV' = \{1, t - 1, (t - 1)^2 , (t - 1)^3 \}$ and $\sW' = \{1, t - 1, (t - 1)^2 \}$.
\end{enumerate}
\end{problem}
%% \medskip
%% \begin{solution}
%% \begin{enumerate}[{\bf a.}]
%% \item 
%% \end{enumerate}
%% \end{solution}

\newpage

%--  PROBLEM 2  ---------------------------------------------------
\begin{problem}[SA 4.4.16]
  \label{prob2}
Let $V$ be a finite-dimensional vector space, let $W$ be a vector space, and let 
$T : V \to W$ be a linear transformation. 
Prove the matrix-free analog of the ``rank+nullity theorem'' for linear transformations.
That is, give a (matrix-free) proof that $\dim(\ker T) + \dim(\im T) = \dim V$
by following these steps.
\begin{enumerate}[{\bf a.}]
\item 
Let $\{\vv_1, \vv_2, \dots, \vv_k\}$ 
be a basis for $\ker(T)$, and (following Exercise 3.4.17) 
extend to obtain a basis $\{\vv_1, \dots, \vv_k, \vv_{k+1},\dots, \vv_n\}$ 
for $V$. Show that $\{T(\vv_{k+1}), \dots, T(\vv_n)\}$
gives a basis for $\im(T)$.
\item Conclude the desired result. Explain why this is a restatement of
  Corollary 4.7 of Chapter 3 when $W$ is finite-dimensional.
\end{enumerate}
\end{problem}
\medskip
%% \begin{solution}\
%% \begin{enumerate}[{\bf a.}]
%% \item 
%% \end{enumerate}
%% \end{solution}

\newpage

%--  PROBLEM 3  ---------------------------------------------------
\begin{problem}[SA 4.4.18]
Prove the following claims.
\begin{enumerate}[{\bf a.}]
\item {\bf Claim 1:}
  If $T : V \to W$ is a linear transformation and  
  if $\{\vv_1, \vv_2, \dots, \vv_k \}$ is a linearly dependent set of vectors in
  $V$, then $\{T(\vv_1), T(\vv_2), \dots , T(\vv_k)\}$ is a 
  linearly dependent set in $W$. 
\item {\bf Claim 2:}
  Suppose $T : V \to V$ is a linear transformation and $V$ is
  finite-dimensional.
  If $\im(T) = V$ and if $\{\vv_1, \vv_2, \dots, \vv_k \}$
  is linearly independent, then $\{T(\vv_1), T(\vv_2), \dots , T(\vv_k)\}$ is 
  linearly independent.
  ({\it Hint:} Use Problem~\ref{prob2} (SA 4.4.16).)
\end{enumerate}
\end{problem}
%% \medskip
%% \begin{solution}\
%% \begin{enumerate}[{\bf a.}]
%% \item 
%% \end{enumerate}
%% \end{solution}
\newpage

%--  PROBLEM 4  ---------------------------------------------------
\begin{problem}[SA 4.4.21]
Let $V$ be a vector space.
\begin{enumerate}[{\bf a.}]
\item 
Let $V^\ast$ denote the set of all linear transformations from $V$ to $\R$. 
Show that $V^\ast$ is a vector space.\footnote{The vectors in 
$V^\ast$ are sometimes called the \emph{linear functionals} of $V$.}

\item Suppose $\{\vv_1, \vv_2, \dots, \vv_n\}$ is a basis for $V$. 
For $i = 1, 2, \dots, n$, define $\vf_i \in V^\ast$ by
\[
\vf_i (a_1 \vv_1 + a_2 \vv_2 + \cdots + a_n \vv_n) = a_i.
\]
Prove that $\{\vf_1, \vf_2, \dots, \vf_n\}$ gives a basis for $V^\ast$.
\item Deduce that whenever $V$ is finite-dimensional, $\dim V^\ast = \dim V$.
\end{enumerate}
\end{problem}
%% \medskip
%% \begin{solution}\
%% \begin{enumerate}[{\bf a.}]
%% \item
%% \end{enumerate}
%% \end{solution}
\newpage

\section*{Section 6.1}
%--  PROBLEM 5  ---------------------------------------------------
\noindent In the next exercise, parts f and k are required. Part p is recommended.
\begin{problem}[SA 6.1.1fkp]
Find eigenvalues and bases for the eigenspaces of the following matrices.
\begin{enumerate}
\item[{\bf f.}] $\begin{bmatrix*}[r]1&1\\-1&3\end{bmatrix*}$
\item[{\bf k.}] $\begin{bmatrix*}[r]1&-2&2\\-1&0&-1\\0&2&-1\end{bmatrix*}$
\item[{\bf p.}] $\begin{bmatrix*}[r]1&0&0&1\\0&1&1&1\\0&0&2&0\\0&0&0&2\end{bmatrix*}$
\end{enumerate}
\end{problem}
%% \medskip
%% \begin{solution}\
%% \begin{enumerate}
%% \item[{\bf f.}]
%% \item[{\bf k.}]
%% \item[{\bf p.}]
%% \end{enumerate}
%% \end{solution}
\newpage

%--  PROBLEM 6  ---------------------------------------------------
\begin{problem}[cf.~SA 6.1.4]
Let $V\leq \R^n$ be a subspace and let $P_V$ denote the projection onto $V$.  
What are the eigenvalues and eigenvectors of $P_V$? 
({\it Hint:} Think about what $P_V$ does to vectors $\vv\in V$ and to vectors
$\vw\in V^\bot$. Are these the only vectors you need to consider?)
\end{problem}
%% \medskip
%% \begin{solution}
%% \end{solution}
\newpage


%--  PROBLEM 8  ---------------------------------------------------
\noindent In the next exercise, part a is required.  Part b and c are recommended.
\begin{problem}[SA 6.1.13abc]
In each of the following cases, find the eigenvalues and eigenvectors of the linear
transformation $T : \mathcal{P}_3 \to \mathcal{P}_3$.
\begin{enumerate}[{\bf a.}]
\item $T(p)(t) = p'(t)$
\item $T(p)(t) = tp(t)$
\item $T(p)(t) = \int_0^t p'(u)\, du$
\end{enumerate}
\end{problem}
%% \medskip
%% \begin{solution}
%% Set $p(t) = a_0 + a_1 t + a_2 t^2 + a_3 t^3$.
%% \begin{enumerate}[{\bf a.}]
%% \item
%% \end{enumerate}
%% \end{solution}
\end{document}

