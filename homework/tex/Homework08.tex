%%This is the LaTeX file for the Math 317 Homework.  

%% You may use this file to fill in your solutions if you know how 
%% to compile a LaTeX file to turn it into a pdf document.  If you 
%% don't know how to do this but want to learn, there are plenty of
%% resources for learning about LaTeX on the net.  You may also ask
%% your professor for help.
%% 
%% For more information about using this file to complete the homework,
%% please see the NOTES section below of the comments below.

\documentclass[fleqn,11pt]{paper}


\usepackage[
letterpaper,
top    = 3cm,
bottom = 3cm,
left   = 3.00cm,
right  = 3.00cm]{geometry}

\usepackage{tikz-cd}
\usepackage{amsthm}
\usepackage{scalefnt}

%%%%%%%%%%%%%%%%%%%%%%%%%%%%%%%%%%%%%%%%
% Basic packages
%%%%%%%%%%%%%%%%%%%%%%%%%%%%%%%%%%%%%%%%
\usepackage{amsmath,amsthm,amssymb}
\usepackage{mathtools}
\usepackage{etoolbox}
\usepackage[mathcal]{euscript}
\usepackage{fancyhdr}
 \usepackage{xcolor}
\usepackage[colorlinks=true,urlcolor=blue,linkcolor=blue,citecolor=blue]{hyperref}
\usepackage{xspace}
\usepackage{comment}
\usepackage{url} % for url in bib entries
\usepackage{mathrsfs}

\theoremstyle{remark}
\newtheorem{theorem}{Theorem}
\newtheorem*{prop}{Proposition}
\newtheorem{problem}{Problem}
\newtheorem*{prob}{Problem}
\newtheorem*{exercise}{Exercise}
\newtheorem*{solution}{{\bf Solution}}
\newtheorem*{hint}{{\it Hint}}
\newtheorem*{ex}{Exercise}


%%%%%%%%%%%%%%%%%%%%%%%%%%%%%%%%%%%%%%%%%%%%%%%%%%%%%%%%%%%%%%%%%%%%%%%%%%%%%%%
%% If you want to prevent pagebreaks, surround the problem and/or solution with 
%% \begin{ProbBox}  and   \end{ProbBox}
\newenvironment{ProbBox}{\noindent\begin{minipage}{\linewidth}}{\end{minipage}}

%%%%%%%%%%%%%%%%%%%%%%%%
% Fancy page style     %
%%%%%%%%%%%%%%%%%%%%%%%%
\pagestyle{fancy}
\newcommand{\metadata}[2]{
  \lhead{}
  \chead{}
  \rhead{\bfseries Math 317: Linear Algebra}
  \lfoot{#1}
  \cfoot{#2}
  \rfoot{\thepage}
}
\renewcommand{\headrulewidth}{0.4pt}
\renewcommand{\footrulewidth}{0.4pt}
\newrobustcmd*{\vocab}[1]{\emph{#1}}
\newrobustcmd*{\latin}[1]{\textit{#1}}

\usepackage{enumerate}

%%%%%%%%%%%%%%%%%%%%%%%%%%%%
%% Space between problems  %
%%%%%%%%%%%%%%%%%%%%%%%%%%%%
\newrobustcmd*{\probskip}{\vskip1cm}


%%%%%%%%%%%%%%%%%%%%%%%%%%
%%    Math shortcuts     %
%%%%%%%%%%%%%%%%%%%%%%%%%%
\newcommand\join{\ensuremath{\vee}}
\newcommand\meet{\ensuremath{\wedge}}
\newcommand\R{\fld{R}}
\newcommand\proj{\ensuremath{\operatorname{proj}}}
\newcommand\End{\ensuremath{\operatorname{End}}}
\newcommand\Aut{\ensuremath{\operatorname{Aut}}}
\newcommand\Hom{\ensuremath{\operatorname{Hom}}}
\newcommand{\Aff}{\ensuremath{\operatorname{Aff}}}
\newcommand{\ann}[1]{\ensuremath{\operatorname{ann}(#1)}}
\newcommand{\id}{\ensuremath{\operatorname{id}}}
\newcommand{\nulity}{\ensuremath{\operatorname{null}}}
\renewcommand{\ker}{\ensuremath{\operatorname{ker}}}
\renewcommand{\dim}{\ensuremath{\operatorname{dim}}}
\newcommand\im{\ensuremath{\operatorname{im}}}
\newcommand{\rank}{\ensuremath{\operatorname{rank}}}
\newcommand{\trace}{\ensuremath{\operatorname{trace}}}
\newcommand{\vR}{\ensuremath{\operatorname{R}}}
\newcommand{\vN}{\ensuremath{\operatorname{N}}}
\newcommand{\vC}{\ensuremath{\operatorname{C}}}
\renewcommand{\phi}{\ensuremath{\varphi}}
\renewcommand{\vec}[1]{\mathbf{#1}}

%%%% NOTES %%%%%%%%%%%%%%%%%%%%%%%%%%%%%%%%%%%%%%%%%%%%%%%%%%%%%%%%% 
         \newcommand\alg[1]{\ensuremath{\mathbf{#1}}}
         \newcommand{\<}{\ensuremath{\langle}}
         \renewcommand{\>}{\ensuremath{\rangle}}
         \newcommand\fld[1]{\ensuremath{\mathbb{#1}}}
%%       To make a boldface vector, use backslash v in front of the 
%%       letter and add a new command for that letter here.
         \newcommand\va{\vec{a}}
         \newcommand\vu{\vec{u}}
         \newcommand\vv{\vec{v}}
         \newcommand\vw{\vec{w}}
         \newcommand\vx{\vec{x}}
         \newcommand\vy{\vec{y}}
         \newcommand\vz{\vec{z}}
         \newcommand\vzero{\vec{0}}
         \newcommand\sV{\ensuremath{\mathcal V}}
         \newcommand\sW{\ensuremath{\mathcal W}}
         \newcommand\sP{\ensuremath{\mathcal P}}
         \newcommand\Span{\ensuremath{\operatorname{Span}}}

         \author         {NAME:                     }
         %% Put your name here ^^^^^^^^^^^^^^^^^^^^^, like this:
         %%     \author{NAME: William DeMeo}
         %%
         %% Update the title and date if necessary.
         \title{Math 317: Homework 8}
         \date{Due: 22 March 2016}
         \metadata{HW 8}{Due March 22}
%%%%%%%%%%%%%%%%%%%%%%%%%%%%%%%%%%%%%%%%%%%%%%%%%%%%%%%%%%%%%%%%%%% 


\begin{document}

\maketitle
\section*{Section 3.4}
%--  PROBLEM 1  ---------------------------------------------------
\begin{problem}
%% Let $\vv_1 = (1,0,0)$, $\vv_2 = (1,1,0)$, $\vw_1 = (0,1,1)$, $\vw_2 = (0,0,1)$,
%% and let $\sV =\{\vv_1, \vv_2\}$ and $\sW= \{\vw_1, \vw_2\}$.
Let $\sV$ and $\sW$ be subsets of $\R^4$ given by
\[
\sV :=\{\vv_1, \vv_2\} = \{(1,0,0,0), (1,1,0,0)\} \; \text{ and } \;
\sW := \{\vw_1, \vw_2\} =\{(0,1,1,0), (0,0,1,0)\}.
\]
It is not hard to see that each of these sets is linearly independent.  What about their union?
Decide whether the following claim and its proof are valid.  If not, where is the error?
\\[5pt]
\underline{Claim}: The set $\sV \cup \sW = \{\vv_1, \vv_2, \vw_1, \vw_2\}$ is linearly independent. \\[4pt]
{\it Proof Attempt:} Consider, for each $i\in \{1, 2\}$, the set $\{\vw_1, \vw_2, \vv_i\}$.
Both of the matrices
\[
\begin{bmatrix*}
    | & | & |\\
    \vw_1& \vw_2 & \vv_1\\
    | & | & |
  \end{bmatrix*}=
  \begin{bmatrix*}
    0  & 0 & 1\\
    1  & 0 & 0\\
    1  & 1 & 0\\
    0  & 0 & 0
  \end{bmatrix*}
\; \text{ and } \;
\begin{bmatrix*}
    | & | & |\\
    \vw_1& \vw_2 & \vv_2\\
    | & | & |
  \end{bmatrix*}=
  \begin{bmatrix*}
    0  & 0 & 1\\
    1  & 0 & 1\\
    1  & 1 & 0\\
    0  & 0 & 0
  \end{bmatrix*}\]
  have reduced echelon form
  $\begin{bmatrix*}
    1  & 0 & 0\\
    0  & 1 & 0\\
    0  & 0 & 1\\
    0  & 0 & 0
  \end{bmatrix*}$.
We deduce the following facts:
\begin{itemize}
\item the set $\{\vw_1, \vw_2, \vv_1\}$ is linearly independent;
\item the set $\{\vw_1, \vw_2, \vv_2\}$ is linearly independent;
\item the set $\sV = \{\vv_1, \vv_2\}$ is linearly independent.
\end{itemize}
It follows that the set $\sV \cup \sW = \{\vv_1, \vv_2, \vw_1, \vw_2\}$ is linearly independent. 
\end{problem}
%% \medskip
%% \begin{solution}
%% \end{solution}

\newpage

%--  PROBLEM 2  ---------------------------------------------------
\begin{problem} %[cf~SA 3.4.14] 
Recall,\\[4pt]
{\bf Proposition 3.6.} If $V \leq \R^n$, then $(V^\bot)^\bot= V$.
\begin{proof}
Choose a basis $\{\vv_1,\dots, \vv_k\}$ for $V$, and consider the 
$k \times n$ matrix $A$ with rows $\vv_1,\dots, \vv_k$. 
By construction, $V = \vR(A)$. By Theorem 2.5, $V^\bot = R(A)^\bot = \vN(A)$, 
and $\vN(A)^\bot = R(A)$, so $(V^\bot)^\bot = V$.
\end{proof}
As mentioned in lecture, the inclusion $V \subseteq (V^\bot)^\bot$ follows directly from the 
definitions.  It is the reverse inclusion $(V^\bot)^\bot \subseteq V$ that requires proof.
Use Theorem 4.9 to show that $\vx \in (V^\bot)^\bot$ implies $\vx \in V$ (thus giving
an alternative proof of Prop~3.6).
\end{problem}
%% \medskip
%% \begin{solution}
%% \end{solution}


\newpage

%--  EXERCISE  ---------------------------------------------------
\noindent The next exercise is recommended but will not be graded.

\begin{exercise}[SA 3.4.15] 
Suppose $V \leq \R^n$. Let $\{\vv_1,\vv_2, \dots, \vv_k\}$ be a basis for $V$, and let 
$\vw_1,\vw_2, \dots, \vw_\ell \in V$ be vectors such that 
$\Span \{\vw_1, \vw_2, \dots, \vw_\ell\} = V$. Prove that $\ell \geq k$.
\end{exercise}
%% \medskip
%% \begin{solution}
%% \end{solution}

\newpage

%--  PROBLEM 4  ---------------------------------------------------
\begin{problem}[SA~3.4.16]
Prove the Proposition 4.4. ({\it Hint:} 3.4.15 and Prop~4.3 are helpful.)  \\[4pt]
{\bf Proposition 4.4.} 
Let $V \leq \R^n$ be a $k$-dimensional subspace. 
Then any $k$ vectors that span $V$ must be linearly independent, 
and any $k$ linearly independent vectors in $V$ must span $V$.
\begin{proof}\
~
  \vfill
\end{proof}
\end{problem}
\newpage


%--  PROBLEM 5  ---------------------------------------------------
\begin{problem}[SA~3.4.17] 
Let $V \leq \R^n$ be a subspace, 
and suppose $\{\vv_1,\dots, \vv_k\}\subseteq V$ is a linearly independent set of vectors.
Show that if $\dim V > k$, then there are 
vectors $\vv_{k+1},\dots, \vv_\ell \in V$ such that 
$\{\vv_{1},\dots, \vv_\ell\}$ forms a basis for $V$.
\end{problem}
%% \medskip
%% \begin{solution}
%% \end{solution}

\newpage

%--  PROBLEM  ---------------------------------------------------
\noindent The next exercise is recommended but will not be graded.
\begin{exercise}[SA 3.4.18]
Suppose $V$ and $W$ are subspaces of $\R^n$ and $W \leq V$. 
Prove that $\dim W \leq \dim V$. 
%({\it Hint:} Start with a basis for $W$ and apply Exercise 3.4.17.)
\end{exercise}
%% \medskip
%% \begin{solution}
%% \end{solution}

\newpage

%--  PROBLEM 5 ---------------------------------------------------
\begin{problem}[SA 3.4.19] 
Suppose $A$ is an $n \times n$ matrix, and let $\vv_1,\dots, \vv_n \in \R^n$. 
Suppose $\{A\vv_1,\dots, A\vv_n\}$ is linearly independent. 
Prove that $A$ is nonsingular.
\end{problem}
%% \medskip
%% \begin{solution}
%% \end{solution}

\newpage

%--  EXERCISE ---------------------------------------------------
\noindent The exercises on this page are recommended but will not be graded.
\begin{exercise}[SA 3.3.23, 3.4.20]
Let $U$ and $V$ be subspaces of $\R^n$ with $U \cap V =\{\vzero\}$.
\begin{enumerate}[(a)]
\item If $\{\vu_1, \dots, \vu_k\}$ is a basis for $U$ 
and $\{\vv_1, \dots, \vv_\ell\}$ is a basis for $V$, then  
$\{\vu_1, \dots, \vu_k, \vv_1, \dots, \vv_\ell\}$ is a basis for $U + V$.
\item $\dim(U + V) = \dim U + \dim V$.
\end{enumerate}
\end{exercise}
%% \medskip
%% \begin{solution}
%% \end{solution}

\vfill

%--  EXERCISE ---------------------------------------------------
\begin{exercise}[SA 3.4.21] 
Let $U$ and $V$ be subspaces of $\R^n$. 
Prove that 
\[
\dim(U + V) = \dim U + \dim V - \dim(U \cap V).
\] 
(Hint: Start with a basis for $U \cap V$, and use Exercise 3.4.17.)
\end{exercise}
%% \medskip
%% \begin{solution}
%% \end{solution}
\vskip8cm

\newpage

\section*{Section 3.6}

%--  PROBLEM 6 ---------------------------------------------------
\begin{problem}[SA 3.6.2abc]
Decide whether the following sets of vectors are linearly independent. 
(Justify your answers.)
\begin{enumerate}[a.]
\item
  $\left\{\begin{bmatrix*}[r]1&0\\0&1\end{bmatrix*},
          \begin{bmatrix*}[r]0&1\\1&0\end{bmatrix*},
          \begin{bmatrix*}[r]1&1\\1&-1\end{bmatrix*}\right\} \subseteq \R^{2\times 2}$
\item
  $\{f_1, f_2, f_3\} \subseteq \mathcal{P}_1$, where
  $f_1(t) = t$, $f_2(t) = t + 1$, $f_3(t) = t + 2$
\item
  $\{f_1, f_2, f_3\} \subseteq \mathcal{C}^\infty(\R)$, where
  $f_1(t) = 1$, $f_2(t) = \cos t$, $f_3(t) = \sin t$
\end{enumerate}
\end{problem}
%% \medskip
%% \begin{solution}
%% \end{solution}

\newpage

%--  PROBLEM 7 ---------------------------------------------------
\begin{problem}[SA 3.6.15a]
Let $g_1(t) = 1$ and $g_2(t) = t$. 
Using the inner product defined in Example~10(c), find
the orthogonal complement of $\Span \{g_1, g_2\}$ in
$\mathcal{P}_2 \subset \mathcal{C}^0([-1, 1])$.
\end{problem}
%% \medskip
%% \begin{solution}
%% \end{solution}

\newpage

%--  EXERCISE  ---------------------------------------------------
\noindent The next exercise is recommended but will not be graded.
\begin{exercise}
Let $V$ be the set of all functions mapping $\R$ to itself.  That is, $V$ is the
set of all real-valued functions with domain the set of real numbers.
(We often denote such functions by $f:\R \to \R$.)
Let $W\subseteq V$ and suppose that $W$ contains the constant
function $x\mapsto 0$ (taking every $x$ to 0)
as well as all of those functions $f \in V$ satisfying the
following condition: $f(x) = 0$ for at most finitely many real numbers $x$.
Prove or disprove: $W$ is a subspace of $V$.
\end{exercise}
%% \medskip
%% \begin{solution}
%% \end{solution}

\newpage
%--  EXERCISE  ---------------------------------------------------
\noindent The next exercise is just for fun. 
Think about it over dinner. You need not submit a solution.
\begin{exercise}
Fix an integer $n$.  Let $V = \{0, 1, 2, \dots, 2^n - 1\}$.  (Notice that the
size of the set $V$ is $2^n$.)  Each number in $V$ can be 
represented as a ``binary vector'' of length $n$ as shown in Table~\ref{tab:1}.
Let $F = \{0,1\}$ be the scalar field with two elements, where multiplication on
$F$ is defined as usual, and addition is defined modulo 2. That is,
\begin{align*}
0\cdot 0 &= 0, \quad 0\cdot 1 =0= 1\cdot 0, \quad 1\cdot 1 = 1,\\
0+0 &= 0 = 1+1, \; \text{ and } \; \quad 1+0=1=0+1.
\end{align*}
Using the binary representation given above, each number
in $V$ is a vector (of length $n$) with entries 0 and 1.
Show that $V$ is a vector space over the scalar field
$F$.  ({\it Hints:} Say how vector addition and scalar
multiplication should be defined, say what plays the role of the zero vector,
and check that, with these definitions, $V$ satisfies all the properties of
a vector space. In particular, for each vector $\vv\in V$, identify the vector
$\vw$ that satisfies $\vv + \vw = \vzero$.)

\begin{table}[h!]
  \begin{center}
\begin{tabular}{c|l}
  decimal & binary\\
  \hline
$0 $    & $\begin{bmatrix*}[r] 0&0&0&0&\dots& 0&0\end{bmatrix*}$\\[2pt]
$1 $    & $\begin{bmatrix*}[r] 1&0&0&0&\dots& 0&0\end{bmatrix*}$\\[2pt]
$2 $    & $\begin{bmatrix*}[r] 0&1&0&0&\dots& 0&0\end{bmatrix*}$\\[2pt]
$3 $    & $\begin{bmatrix*}[r] 1&1&0&0&\dots& 0&0\end{bmatrix*}$\\[2pt]
$4 $    & $\begin{bmatrix*}[r] 0&0&1&0&\dots& 0&0\end{bmatrix*}$\\
$\vdots$        &$\quad \vdots$\\
$2^n-2 $& $\begin{bmatrix*}[r]0&1&1&1&\cdots& 1&1\end{bmatrix*}$\\
$2^n-1 $& $\begin{bmatrix*}[r]1&1&1&1&\cdots&1&1\end{bmatrix*}$
\end{tabular}
  \caption{representation of integers as binary vectors} 
  \label{tab:1}
  \end{center}
\end{table}
\end{exercise}


%% \medskip
%% \begin{solution}
%% \end{solution}

\newpage

%--  EXERCISE  ---------------------------------------------------
\noindent The next exercise is just for fun.  Think about it over the break. You need not submit a solution.
\begin{exercise}
  Let $X$ be a nonempty set.  Consider the so called power set $\mathscr{P}(X)$
  of all subsets of $X$. 
  Can you find a way to turn $V = \mathscr{P}(X)$ into a vector space?
  ({\it Hints:} If each subset of $X$ is a vector,
  how could you define vector addition? ...and scalar multiplication?
  What would you take as the zero vector? Find appropriate definitions for
  these operations so that the defining vector space properties are
  satisfied. Thinking about the problem on the previous page might help.)
\end{exercise}

%% \medskip
%% \begin{solution}
%% \end{solution}


  %% \bibliographystyle{plain}
%% \bibliography{refs}

\end{document}
